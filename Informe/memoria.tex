%%%%%%%%%%%%%%%%%%%%%%%%%%%%%%%%%%%%%%%%%
% Masters/Doctoral Thesis 
% LaTeX Template
% Version 2.3 (25/3/16)
%
% This template has been downloaded from:
% http://www.LaTeXTemplates.com
%
% Version 2.x major modifications by:
% Vel (vel@latextemplates.com)
%
% This template is based on a template by:
% Steve Gunn (http://users.ecs.soton.ac.uk/srg/softwaretools/document/templates/)
% Sunil Patel (http://www.sunilpatel.co.uk/thesis-template/)
%
% Template license:
% CC BY-NC-SA 3.0 (http://creativecommons.org/licenses/by-nc-sa/3.0/)
%
%%%%%%%%%%%%%%%%%%%%%%%%%%%%%%%%%%%%%%%%%

%------------------------------------------------------------------------------
%	PACKAGES AND OTHER DOCUMENT CONFIGURATIONS
%------------------------------------------------------------------------------

\documentclass[
11pt, % The default document font size, options: 10pt, 11pt, 12pt
%oneside, % Two side (alternating margins) for binding by default, uncomment to switch to one side
%chapterinoneline,% Have the chapter title next to the number in one single line
%english, % ngerman for German
spanish,
singlespacing, % Single line spacing, alternatives: onehalfspacing or doublespacing
%draft, % Uncomment to enable draft mode (no pictures, no links, overfull hboxes indicated)
%nolistspacing, % If the document is onehalfspacing or doublespacing, uncomment this to set spacing in lists to single
%liststotoc, % Uncomment to add the list of figures/tables/etc to the table of contents
%toctotoc, % Uncomment to add the main table of contents to the table of contents
parskip, % Uncomment to add space between paragraphs
%nohyperref, % Uncomment to not load the hyperref package
headsepline, % Uncomment to get a line under the header
]{MastersDoctoralThesis} % The class file specifying the document structure



\usepackage[utf8]{inputenc} % Required for inputting international characters
\usepackage[T1]{fontenc} % Output font encoding for international characters

\usepackage{palatino} % Use the Palatino font by default
%,style=authoryear
\usepackage[backend=bibtex,natbib=true]{biblatex} % Use the bibtex backend with the authoryear citation style (which resembles APA)

\addbibresource{references.bib} % The filename of the bibliography

\usepackage[autostyle=true]{csquotes} % Required to generate language-dependent quotes in the bibliography

\usepackage{caption}
\usepackage{subcaption}

%------------------------
\usepackage{listings}

\usepackage[framemethod=tikz]{mdframed} % @Eric for License
\usepackage{enumitem} % @Eric for Lists 1. 1.1. ...

% Paquete para código en algún lenguaje de programación
\usepackage{listings} % @Eric for Syntax Highlighting

% Defino el sintaxhighlight para lenguaje C
\lstdefinestyle{customc}{
	frame=Ltb,
    framerule=0pt,
    aboveskip=0.5cm,
    framextopmargin=3pt,
    framexbottommargin=3pt,
    framexleftmargin=0.4cm,
    framesep=0pt,
    rulesep=.4pt,
    %backgroundcolor=\color{gray97},
    rulesepcolor=\color{black},
    %
    %stringstyle=\ttfamily,
    showstringspaces = false,
    basicstyle=\small\ttfamily,
    %commentstyle=\color{gray45},
    %keywordstyle=\bfseries,
    %
    numbers=left,
    numbersep=15pt,
    numberstyle=\tiny,
    numberfirstline = false,
    breaklines=true,
    %
    %     
  belowcaptionskip=1\baselineskip,
  %breaklines=true,
  %frame=L,
  xleftmargin=\parindent,
  language=C,
  %showstringspaces=false,
  %basicstyle=\footnotesize\ttfamily,
  keywordstyle=\bfseries\color{purple!80!black},
  commentstyle=\itshape\color{green!40!black},
  identifierstyle=\color{blue},
  stringstyle=\color{orange},
  %numbers=left, % where to put the line-numbers; values are (none, left, right)
  %numbersep=10pt,   % how far the line-numbers are from the code
  %numberstyle=\tiny % the style that is used for the line-numbers
}	
\lstset{escapechar=@,style=customc}	

% Defino el sintaxhighlight para lenguaje Smalltalk

\lstdefinelanguage{Smalltalk}{
  morekeywords={true,false,self,super,nil},
  sensitive=true,
  morecomment=[s]{"}{"},
  morestring=[d]',
  %style=SmalltalkStyle
}
%\lstdefinestyle{SmalltalkStyle}{
%  literate={:=}{{$\gets$}}1{^}{{$\uparrow$}}1
%} 




%\usepackage[hyphens]{url}
%\usepackage[hidelinks]{hyperref}
%\hypersetup{breaklinks=true}
\urlstyle{same}
%\usepackage{cite}

%--------------------------

\usepackage{color}

%
%------------------------------------------------------------------------------
%	MARGIN SETTINGS
%------------------------------------------------------------------------------

\geometry{
	paper=a4paper, % Change to letterpaper for US letter
	inner=2cm, % Inner margin
	outer=3.3cm, % Outer margin
	bindingoffset=2cm, % Binding offset
	top=1.5cm, % Top margin
	bottom=1.5cm, % Bottom margin
	%showframe,% show how the type block is set on the page
}

%------------------------------------------------------------------------------
%	INFORMACIÓN DE LA MEMORIA
%------------------------------------------------------------------------------

\thesistitle{sAPI (simpleAPI): diseño e implementación de una biblioteca para sistematizar
la programación de sistemas embebidos} % El títulos de la memoria, se usa en la carátula y se puede usar el cualquier lugar del documento con el comando \ttitle
\supervisor{MSc. Ing. Félix Gustavo Emilio Safar (UNQ)} % El nombre del director, se usa en la carátula y se puede usar el cualquier lugar del documento con el comando \supname
\degree{Magíster en Sistemas Embebidos} % Nombre del grado, se usa en la carátula y se puede usar el cualquier lugar del documento con el comando \degreename
\author{Esp. Ing. Eric Nicolás Pernia} % Tu nombre, se usa en la carátula y se puede usar el cualquier lugar del documento con el comando \authorname
\juradoUNO{Dr. Ing. Pablo Martín Gómez (UBA)} % Nombre y pertenencia del un jurado se usa en la carátula y se puede usar el cualquier lugar del documento con el comando \jur1name
\juradoDOS{Mg. Ing. Pablo Oscar Ridolfi (UTN FRBA)} % Nombre y pertenencia del un jurado se usa en la carátula y se puede usar el cualquier lugar del documento con el comando \jur2name
\juradoTRES{Ing. Juan Manuel Cruz (FIUBA,UTN-FRBA)} % Nombre y pertenencia del un jurado se usa en la carátula y se puede usar el cualquier lugar del documento con el comando \jur3name
\fechaINICIO{marzo de 2018}
\fechaFINAL{diciembre de 2018}

% Your subject area, this is not currently used anywhere in the template, print it elsewhere with \subjectname
\subject{Memoria del Trabajo Final de la Maestría en Sistemas Embebidos (MSE) de la UBA} 
\keywords{MSE, Sistemas Embebidos, CIAA} % Keywords for your thesis, this is not currently used anywhere in the template, print it elsewhere with \keywordnames
\university{Universidad de Buenos Aires} % Your university's name and URL, this is used in the title page and abstract, print it elsewhere with \univname
\faculty{{Facultad de Ingeniería}} % Your faculty's name and URL, this is used in the title page and abstract, print it elsewhere with \facname
\department{Departamento de Electrónica} % Your department's name and URL, this is used in the title page and abstract, print it elsewhere with \deptname
\group{{Laboratorio de Sistemas Embebidos}} % Your research group's name and URL, this is used in the title page, print it elsewhere with \groupname


\hypersetup{pdftitle=\ttitle} % Set the PDF's title to your title
\hypersetup{pdfauthor=\authorname} % Set the PDF's author to your name
\hypersetup{pdfkeywords=\keywordnames} % Set the PDF's keywords to your keywords


\newcaptionname{spanish}{\acknowledgementname}{Agradecimientos}
\newcaptionname{spanish}{\authorshipname}{Declaración de Autoría}
\newcaptionname{spanish}{\abbrevname}{Glosario}
\newcaptionname{spanish}{\byname}{por}

\renewcommand{\lstlistingname}{Algoritmo}% Listing -> Algorithm

% List of Listings -> List of Algorithms
\renewcommand{\lstlistlistingname}{Índice de \lstlistingname s}

\renewcommand{\listtablename}{Índice de Tablas}
\renewcommand{\tablename}{Tabla} 

\addtolength{\footnotesep}{2mm} % Espacio adicional en los footnotes

\begin{document}

% Use roman page numbering style (i, ii, iii, iv...) for the pre-content pages
\frontmatter 

% Default to the plain heading style until the thesis style is called for the
% body content
\pagestyle{plain}

%------------------------------------------------------------------------------
%	CARÁTULA
%------------------------------------------------------------------------------

\begin{titlepage}
\begin{center}

{\scshape\LARGE UNIVERSIDAD DE BUENOS AIRES\par}\vspace{0.1cm} % University name
{\scshape\LARGE FACULTAD DE INGENIERÍA\par}\vspace{0.1cm} % Faculty name
{\scshape\LARGE Maestría en Sistemas Embebidos\par}\vspace{1cm} % Thesis type

\includegraphics[width=.3\textwidth]{./Figures/logoFIUBA.png}
\vspace{0.7cm}

\textsc{\Large Memoria del Trabajo Final}\\[0.5cm] % Thesis type

{\huge \bfseries \ttitle\par}\vspace{0.4cm} % Thesis title

\vspace{0.7cm}
\LARGE\textbf{Autor:\\
\authorname}\\ % Author name

\vspace{1cm}

\large
\vspace{10px}
{Director:} \\
{\supname} % Supervisor name
 
\vspace{0.7cm}
Jurados:\\
\jurunoname\\
\jurdosname\\
\jurtresname
 
\vfill
\textit{Este trabajo fue realizado en las Ciudad Autónoma de Buenos Aires, entre \fechaINICIOname \hspace{1px} y \fechaFINALname.}
\end{center}
\end{titlepage}


%------------------------------------------------------------------------------
%	LICENCIA - LICENSE
%------------------------------------------------------------------------------
\newpage
\begin{center}

\begin{mdframed}[outerlinecolor=black, outerlinewidth=2pt, linecolor=gray, middlelinewidth=3pt, roundcorner=10pt]

 \begin{center}
  \includegraphics*{Figures/cc-by-sa.png}
 \end{center}

\emph{sAPI (simpleAPI): diseño e implementación de una biblioteca para
sistematizar la programación de sistemas embebidos} por Esp. Ing. Eric Nicolás
Pernia se distribuye bajo una \textbf{Licencia Creative Commons
Atribución-CompartirIgual 4.0 Internacional}.
Para ver una copia de esta licencia, visita:
\\
\url{http://creativecommons.org/licenses/by-sa/4.0/}.


\end{mdframed}

\end{center}

%------------------------------------------------------------------------------
%	RESUMEN - ABSTRACT 
%------------------------------------------------------------------------------

\begin{abstract}
\addchaptertocentry{\abstractname} % Add the abstract to the table of contents
%
%The Thesis Abstract is written here (and usually kept to just this page). 
%The page is kept centered vertically so can expand into the blank space above
%the title too\ldots
\centering

En esta memoria se presenta el diseño de una biblioteca para la programación de
sistemas embebidos portable entre plataformas de hardware y lenguajes de
programación. Se realizó una implementación de referencia en lenguaje C para
las plataformas del Proyecto CIAA. Se realizó un banco de pruebas de hardware,
junto a la utilización de testeo unitario e integración contínua para la
validación.

Además de la biblioteca, se creó una herramienta de código abierto para
desarrolladores que automatiza la implemetación de bibliotecas a partir de un
modelo que las describe. De esta manera se facilita la futura ampliación de la
biblioteca y su implementación en otras plataformas de hardware.

\end{abstract}

%------------------------------------------------------------------------------
%	CONTENIDO DE LA MEMORIA  - AGRADECIMIENTOS
%------------------------------------------------------------------------------

\begin{acknowledgements}
% Descomentando esta línea se puede agregar los agradecimientos al índice
%\addchaptertocentry{\acknowledgementname} 
\vspace{1.5cm}

Agradecimientos personales. \textbf{[OPCIONAL]} 

No olvidarse de agradecer al tutor.

No vale poner anti-agradecimientos (este trabajo fue posible a pesar de...)

\end{acknowledgements}

%------------------------------------------------------------------------------
%	LISTA DE CONTENIDOS/FIGURAS/TABLAS
%------------------------------------------------------------------------------
\renewcommand{\listtablename}{Índice de Tablas}

\tableofcontents % Prints the main table of contents

\listoffigures % Prints the list of figures

\listoftables % Prints the list of tables


%------------------------------------------------------------------------------
%	CONTENIDO DE LA MEMORIA  - DEDICATORIA
%------------------------------------------------------------------------------

% escribir acá si se desea una dedicatoria
\dedicatory{\textbf{Dedicado a... [OPCIONAL]}}  

%------------------------------------------------------------------------------
%	CONTENIDO DE LA MEMORIA  - CAPÍTULOS
%------------------------------------------------------------------------------

\mainmatter % Begin numeric (1,2,3...) page numbering

\pagestyle{thesis} % Return the page headers back to the "thesis" style

\renewcommand{\tablename}{Tabla} 

%------------------------------------------------------------------------------

% Define some commands to keep the formatting separated from the content 
\newcommand{\keyword}[1]{\textbf{#1}}
\newcommand{\tabhead}[1]{\textbf{#1}}
\newcommand{\code}[1]{\texttt{#1}}
\newcommand{\file}[1]{\texttt{\bfseries#1}}
\newcommand{\option}[1]{\texttt{\itshape#1}}
\newcommand{\grados}{$^{\circ}$}

%------------------------------------------------------------------------------

\newcommand{\titulo}[1]{\bigskip \noindent\textbf{#1} \\} % @Eric
\newcommand{\subtitulo}[1]{\bigskip \noindent\textit{#1} \\} % @Eric

%------------------------------------------------------------------------------

% Incluir los capítulos como archivos separados desde la carpeta Chapters
% Descomentar las líneas a medida que se escriben los capítulos

% Introducción General
\chapter{Introducción General}
\label{ChapterIntroGral}

En este capítulo se presenta el contexto y justificación de este trabajo final, la motivación para llevarlo a cabo, sus objetivos y alcance.
%(\ref{sec:contextoYJustificacion})
%(\ref{sec:motivacion})
%(\ref{sec:objetivosAlance})

%------------------------------------------------------------------------------
%	SECTION
%------------------------------------------------------------------------------
\section{Contexto y justificación}
\label{sec:contextoYJustificacion}

\emph{La idea de esta sección es presentar el tema de modo que cualquier persona que no conoce el tema pueda entender de qué se trata y por qué es importante realizar este trabajo y cuál es su impacto.}


En la actualidad la programación de plataformas basadas en microcontrolador se realiza su mayoría en lenguaje C utilizando bibliotecas para el manejo del núcleo de procesamiento (o los núcleos) y periféricos. En consecuencia, una biblioteca de C es parte integral de cualquier diseño de sistema embebido basado en microcontrolador. 

En este trabajo se presenta un diseño de biblioteca de programación de plataformas de Sistemas Embebidos basadas en microcontrolador de forma sencilla. La misma permite utilizar los modos más comunes de los periféricos típicos de un microcontrolador (tanto del núcleo de procesamiento como sus periféricos). Se expone además las características de implementación de referencia sobre la plataforma EDU-CIAA-NXP.







En la actualidad existe una enorme variedad plataformas de Sistemas Embebidos en el mercado, y si bien todas cuentan con dispositivos programables con características similares y periféricos compatibles, se observa que en la práctica son muy distintos. En consecuencia se debe invertir mucho tiempo en aprender a programar cada una de ellas, con sus particularidades antes de su utilización. Estas diferencias se deben a varios factores:

\begin{itemize}
   \item Los fabricantes de los dispositivos programables y empresas asociadas carecen de diseños estándar de arquitectura de hardware (tanto en núcleos de procesamiento, como periféricos). Si bien esto trae el beneficio de permitir elegir el dispositivo programable que más se adecúe a un proyecto, también es la principal causa de la necesidad de conocer en detalle la arquitectura en particular.
   \item Dichas empresas en su mayoría se limitan a ofrecer información de bajo nivel para programar el hardware directamente, o bien, sus propias bibliotecas escritas en lenguaje C, que están diseñadas con una gran dependencia de la arquitectura de hardware subyacente, es decir, carecen de abstracción del hardware.
\end{itemize}




Se están portando cada vez más lenguajes de programación a las plataformas de Sistemas Embebidos, que antes se reservaban para las computadoras de propósito general (como la PC).
Esto se debe a que las nuevas plataformas poseen más poder de procesamiento y debido a la complejidad creciente de las aplicaciones se exige más características y abstracción a los lenguajes de programación.

Existen algunos desarrollos de bibliotecas que logran una abstracción de hardware aceptable en varias plataformas, pero que están escritas en un único lenguaje de programación y para el ecosistema de plataformas que soportan la empresa o comunidad involucrada. Ninguna de ellas se ha adoptado como estándar de facto.





Debido a la amplia variedad de plataformas de hardware y lenguajes de programación, se propone en este proyecto la realización del diseño de una biblioteca modelada independientemente del lenguaje de programación y arquitectura del hardware, con la intención de convertirlo en una propuesta de estándar para la programación de Sistemas Embebidos. Se tendrá en cuenta en el modelado lograr una buena relación de compromiso entre el nivel de abstracción para independizarse del hardware y los conceptos que espera encontrar un programador de Sistemas embebidos.
Como el lenguaje más utilizado en la actualidad para programación de sistemas embebidos basados en microcontroladores continúa siendo el lenguaje C, se realizará una implementación de referencia para dicho lenguaje.

Dado que el presente autor cumple el rol de Coordinador General del proyecto de hardware y software abierto "Computadora Industrial Abierta Argentina (CIAA)" se realizará la implementación de referencia tomando las plataformas de hardware diseñadas en el marco de este proyecto como casos de validación del diseño de biblioteca a realizar. Esto permitirá a la comunidad de usuarios de las plataformas del Proyecto CIAA programar las diferentes plataformas utilizando la misma biblioteca, reduciendo tiempos de aprendizaje y desarrollo.

Este trabajo es parte de un grupo de iniciativas promovidas por la Universidad Nacional de Quilmes para colaborar en el marco del Proyecto CIAA, para facilitar el desarrollo y la enseñanza de Sistemas Embebidos en Argentina. Otras iniciativas incluyen: entorno de programación PLC Ladder (IDE4PLC), programación Java para CIAA (CIAA-HVM), Firmata4CIAA (para facilitar la programación de bloques gráficos en BYOB Snap! lenguaje para uso en escuelas secundarias) y CIAABOT IDE.

%------------------------------------------------------------------------------
%	SECTION
%------------------------------------------------------------------------------
\section{Motivación}
\label{sec:motivacion}


%------------------------------------------------------------------------------
\subsection{Lenguajes de programación y Hardware en Sistemas Embebidos}

Saraza...


%------------------------------------------------------------------------------
\subsection{Bibliotecas para microcontroladores ofrecidas por los fabricantes}

Saraza...

%------------------------------------------------------------------------------
\subsection{Proyecto CIAA}

Saraza...



En el mercado se encuentra una gran variedad de plataformas basadas en microcontrolador y aunque todas ellas poseen microcontroladores con características similares y periféricos compatibles, sin embargo, sus bibliotecas son muy diferentes. Esto se debe a que cada fabricante y/o empresas asociadas ofrece sus propias bibliotecas en lenguaje C las cuales están diseñadas fuertemente dependientes de cada arquitectura de cada microcontrolador que estas plataformas contienen.
Existen también muchas bibliotecas que logran una buena abstracción del hardware en varias plataformas, pero ninguna se ha adoptado como estándar general. Esto se debe a múltiples causas, entre ellas:


En el sector de la industria automotriz existe un estándar llamado AUTOSAR[] para la estandariza- ción de la  arquitectura de sistemas electrónicos que aún no ha logrado extenderse a otras industrias y cuya definición de bibliotecas propuesta es muy extensa y compleja de implementar y también para aprender a utilizar.
Bibliotecas de drivers basadas en POSIX[] (que podemos hallar en sistemas con en Linux Embebido como Raspberry Pi[]). Su abstracción ha sido muy útil en la estandarización de drivers para PCs con sistema operativo Unix-compatible. Sin embargo, es tan alejada del hardware físico que en la práctica provoca una muy baja utilización por parte de los profesionales electrónicos y afines que se dedican a la programación de sistemas embebidos.
Programadores hobbistas de sistemas embebidos han impulsado la  estandarización de la programación mediante una biblioteca conocida como Wiring[] disponible para múltiples plataformas (como la popular Arduino[]). Esta biblioteca si bien logra una gran facilidad de uso y rápido aprendizaje contiene algunas imprecisiones técnicas que provoca vicios indeseados en el aprendizaje de programación de microcontroladores. También carece de definición de una API para la utilización del periférico temporizador el cual es muy utilizado en un microcontrolador.
En la actualidad existen muchos fabricantes de microcontroladores que adquieren licencias para la fabricación de microcontroladores con núcleos de procesamiento de arquitecturas Cortex M[] diseñados por la empresa ARM[]. Para los mismos existe una biblioteca estándar llamada CMSIS{} para la programación del núcleo de procesamiento y controlador de interrupciones pero que no se extiende a los perifŕicos donde cada fabricante busca diferenciarse de sus competidores.
Otras empresas muy difundidas en el campo de la enseñanza en la programación como Microchip[] proveen bibliotecas muy dependientes del hardware y generadores automatizados de código que si bien en principio aceleran los tiempos de desarrollo, cuando una aplicación requiere modos más avanzados terminan dificultando la programación pues unas configuraciones pisan a las otras provocando que no funcione.

De los ejemplos anteriores se observa que para la realización de una biblioteca estándar que satisfaga a los diferentes usuarios se debe lograr un balance entre:

Extensión de la definición de la API.
Dependencia del hardware.
Nivel de abstracción.
Complejidad de aprendizaje y uso.
Periféricos y modos soportados.
Escalabilidad.

Por estos motivos se decide realizar la definición de una API de una biblioteca estándar para microcontroladores  en lenguaje C que supere todas estas dificultades.

%------------------------------------------------------------------------------
%	SECTION
%------------------------------------------------------------------------------
\section{Objetivos y alcance}
\label{sec:objetivosAlance}

En esta sección se definen los objetivos (sección \ref{subsec:objetivos}) y el alcance (sección \ref{subsec:alcance}) del presente Trabajo Final.

%------------------------------------------------------------------------------
\subsection{Objetivos}
\label{subsec:objetivos}

El objetivo de este proyecto es diseñar e implementar una biblioteca de
software para la programación de sistemas embebidos basados en
microcontroladores con las siguientes características: 

\begin{itemize}
   \item Estar modelada independientemente de los lenguajes de programación.
   \item Definir una interfaz de programación de aplicaciones (API) sencilla que abstraiga los modos de uso más comunes de los periféricos típicos que hallados en cualquier microcontrolador del mercado. 
   \item Ser totalmente portable entre diferentes arquitecturas de hardware sobre donde se ejecuta, manteniendo una API uniforme a lo largo de las mismas\footnote{Debe cumplir la función de capa de abstracción de hardware, o \textit{Hardware Abstraction Layer} (HAL),  en inglés}.
\end{itemize}

Dicha biblioteca se deberá implementar en lenguaje C para las plataformas del
Proyecto CIAA.

%------------------------------------------------------------------------------
\subsection{Alcance}
\label{subsec:alcance}

Este Trabajo Final incluye realización de:

\begin{itemize}
   \item Diseño de la biblioteca. Archivos de descripción de la biblioteca mediante diferentes diagramas y código independiente del lenguaje de programación.
   \item Implementación en lenguaje C de la biblioteca para las plataformas de hardware:
   \begin{itemize}
      \item CIAA-NXP.
      \item EDU-CIAA-NXP.
      \item CIAA-Z3R0.
      \item PicoCIAA.
   \end{itemize}
   \item Manual de instalación de las herramientas para utilizar la biblioteca con las plataformas de hardware citadas.
   \item Manual de referencia de la biblioteca.
   \item Ejemplos de utilización.
\end{itemize}


% Introducción Específica
\chapter{Introducción Específica}
\label{ChapterIntroEsp}

En este capítulo se explican los detalles para comprender las decisiones de diseño adoptadas. Se describen los trabajos previos que se tomaron como fundamento para el diseño de la biblioteca, se presentan las plataformas sobre las cuales se elige realizar la implementación de la misma y los requerimientos de este trabajo, junto a la planificación para su cumplimiento.

%En particular, en la sección \ref{sec:antecedentes}
%en la sección \ref{sec:ciaaBoards}
% así como las herramientas provistas por los fabricantes de sus respectivos microcontroladores. Considerando las secciones anteriores, 
%cumplimiento de los mismos (\ref{sec:planificacion}).

%------------------------------------------------------------------------------
%	SECTION
%------------------------------------------------------------------------------
\section{Antecedentes}
\label{sec:antecedentes}

Los primeros pasos para la idea de la realización de este trabajo surgen del trabajo final de la Carrera de Especialización en Sistemas Embebidos de la Universidad de Buenos Aires (CESE FIUBA) \cite{CESE}, realizado por el autor, titulado "Desarrollo de Firmware y Software para programar la CIAA en lenguaje JAVA con aplicación en entornos Industriales" \cite{CeseTesisEric} presentado en diciembre de 2015, donde como parte del mismo se implementó la programación de plataformas de sistemas embebidos en lenguaje Java \cite{Java}. 

En ese trabajo se llevó a cabo, entre otras cosas, la implementación de una biblioteca para acceder a los periféricos de un microcontrolador en lenguaje Java. Para llevarlo a cabo se realizó la definición de clases que modelan los periféricos GPIO, ADC y UART y se implementó el acceso al hardware como métodos nativos de Java, escritos en lenguaje C. Por este motivo se debió realizar una implementación de la biblioteca también lenguaje C. 
%Esta implementación logra un buen balance entre facilidad de uso, extensión y modos de uso definidos. 
Se implementó para las plataformas del proyecto CIAA EDU-CIAA-NXP \cite{EDUCIAA} y CIAA-NXP \cite{CIAANXP}.
% (plataformas que se describen en detalle en la sección \ref{sec:ciaaBoards}). 
Esta biblioteca se nombró sAPI, siglas de \emph{simple API} en alusión a que provee una API sencilla para la programación de microcontroladores.

%Desde su desarrollo, se ha utilizado como ejemplo de capa de abstracción de hardware en las asignaturas dictadas por el autor ''Programación de microprocesadores'' [] de la CESE FIUBA, y la asignatura ''Sistemas Digitales'' de la carrera Ingeniería en Automatización y Control Industrial (IACI[]) de la Universidad Nacional de Quilmes (UNQ).

A partir mazo de 2016 se decide utilizar la biblioteca como base para la enseñanza de la programación de periféricos de microcontroladores.
% en los Cursos Abiertos de Programación de Sistemas Embebidos (CAPSE[]) organizados por la ACSE. 
A lo largo de ese año se extendió la biblioteca de C para la plataforma EDU-CIAA-NXP de forma considerable para explicar la utilización los periféricos típicos de microcontroladores, logrando excelentes resultados en aprendizaje por parte de los alumnos tanto de niveles avanzados como quienes dan sus primeros pasos en el aprendizaje de programación de microcontroladores.

Además, la biblioteca sAPI en lenguaje C para la EDU-CIAA-NXP se puso a disposición de cualquier persona ya que se encuentra publicada de forma libre y gratuita por internet bajo una licencia BSD modificada \citep{BSD3clause} en el sitio de github del autor \citep{sAPIgit}.

Finalmente, en diciembre de 2016 se decide utilizar la biblioteca sAPI realizada en lenguaje C como biblioteca estándar para las plataformas del proyecto CIAA, distribuyéndola como parte del \emph{framework}\footnote{En el desarrollo de software, un \emph{framework} es una estructura conceptual y tecnológica de asistencia definida, normalmente, con artefactos o módulos concretos de software, que sirve de base para la organización y desarrollo de software. En este caso concreto se compone de bibliotecas de código C y makefiles para su compilación permitiendo organizar proyectos de software en lenguaje C.} ''Firmware v2'' \citep{ciaaFirmwareV2}. Este \emph{framework} combinó la biblioteca sAPI con \emph{framework} "Workspace" \citep{ws-ridolfi}, desarrollado por Pablo Ridolfi.

Esta plataforma ha sido adoptada por una gran cantidad de usuarios. Una prueba de ello es la encuesta realizada en octubre de 2018, titulada: "Tecnologías usadas en los Trabajos Finales del Posgrado en Sistemas Embebidos: 2015-2018" \citep{EncuestaCeseMse} donde en la sección 27, \emph{''Uso de material del Proyecto CIAA en los trabajos finales"} se observa que alrededor de la mitad de los trabajos finales de la CESE/MSE, utilizaron material generado en el marco del Proyecto CIAA. Y en particular, a partir de 2017 se produjo un cambio de tendencia y más del 60\% de los trabajos finales utilizaron material del Proyecto CIAA. Este cambio en la tendencia se observa que comienza en 2016 con la publicación de firmware v2 como se muestra en la gráfica de la figura \ref{fig:tendenciaMaterialCiaa}, extraída de dicho artículo.

\begin{figure}[!htbp]
\begin{center}  % [width=14cm,height=8cm] [width=\textwidth]
\includegraphics*[width=12cm]{Figures/TendenciaMaterialCiaa.png}
\par\caption{Evolución en el uso de material del Proyecto CIAA.}\label{fig:tendenciaMaterialCiaa}
\end{center}
\end{figure}

A principios de 2017, el autor llevó a cabo una profunda revisión y mejora de la biblioteca de C con el objetivo de extenderla a las demás plataformas del proyecto CIAA.
% En este rediseño colaboró Martín Ribelotta. 
Ese trabajo fue compilado en un artículo y publicado en el Congreso Argentino de Sistemas Embebidos (CASE) \citep{paperSapiCASE2017} en agosto de 2017. Siendo un antecedente fundamental para la realización del presente trabajo final, porque se realizó un estudio exhaustivo de las bibliotecas existentes en el mercado; dando como resultado los siguientes puntos a considerar en el diseño de una biblioteca de C para la programación de sistemas embebidos:

\begin{itemize}
\item
Extensión de la definición de la API.
\item
Dependencia del hardware.
\item
Nivel de abstracción.
\item
Complejidad de aprendizaje y uso.
\item
Periféricos y modos soportados.
\item
Escalabilidad.
\end{itemize}

Parte de este rediseño se aplicó a la biblioteca sAPI en lenguaje C para la EDU-CIAA-NXP.

En 2017 el autor desarrolló una plataforma más económica que la EDU-CIAA-NXP, nombrada ''CIAA-Z3R0'' \cite{CIAAZ3R0}, que salió al mercado en noviembre. En diciembre el autor publicó la primera versión de la biblioteca sAPI en lenguaje C para esta plataforma dándole soporte a algunos periféricos de la misma.
% (se descibe en detalle en la sección \ref{sec:CIAA-Z3R0})

Tomando la experiencia adquirida a lo largo de estos años, el autor presenta en esta memoria de trabajo final un diseño mejorado para la biblioteca, ampliando el mismo e independizándolo del lenguaje de programación. Asimismo, teniendo en cuenta las tareas repetitivas que requieren la implementación de una biblioteca para diferentes plataformas de sistemas embebidos se decide desarrollar herramientas para automatizar el proceso donde resulta posible.

 
%------------------------------------------------------------------------------
%	SECTION
%------------------------------------------------------------------------------
\section{Plataformas del Proyecto CIAA}
\label{sec:ciaaBoards}

Todos los desarrollos de hardware realizados en el marco del proyecto CIAA han sido publicados con licencia BSD de tres cláusulas con el espíritu de que sean utilizadas como base tanto para diseños abiertos como para diseños cerrados. Los mismos pueden descargarse del repositorio oficial del proyecto CIAA nombrado ''CIAA Hardware'' \citep{ciaaHW}.

Las plataformas sobre las cuales se decide realizar el presente trabajo son aquellas que han logrado pasar de la fase de prototipo y se pueden hallar como producto en el mercado. Cabe destacar que el proyecto CIAA no se beneficia económicamente de la venta de plataformas.
% y fomenta la libre competencia.

\subsection{CIAA-NXP}

La plataforma CIAA-NXP (figura \ref{fig:ciaaNxp}) fue la primer paltaforma desarrollada en el marco del proyecto CIAA. Consiste en una computadora para uso industrial basada en el microcontrolador NXP LPC4337 JDB144 \citep{LPC4337} \emph{dual-core} asimétrico, formado por un procesador Cortex-M4F y un Cortex-M0 (ambos de 32 bits), que corren con una frecuencia de sistema máxima de 204MHz; con 1 MB de memoria Flash y 136 KB de memoria SRAM.


\begin{figure}[!htbp]
\begin{center}  %[width=14cm,height=8cm]
\includegraphics*[width=\textwidth]{Figures/CIAA-NXP_Foto.png}
\par\caption{Plataforma CIAA-NXP.}\label{fig:ciaaNxp}
\end{center}
\end{figure}

Sus características destacables son:

\begin{itemize}
\item
Interfaces de entrada/salida: 8 entradas digitales (opto-aisladas 24VDC), 4 entradas analógicas (0-10V/4-20mA), 4 salidas digitales Open-Drain (24VDC), 4 salidas digitales a relé DPDT y 1 salida analógica (0-10V/4-20mA).
\item
Interfaces de comunicación: 1 Ethernet, 2 USB On-The-Go, 1 RS232, 1 RS485, 1 CAN, 1 SPI, 1 I2C.
\item
Uso de Linux: Posee memorias RAM y Flash externas que posibilitan ejecutar Linux sobre esta plataforma.
\item
Incluye \emph{debugger}: Esta plataforma incluye el circuito que permite depuración en tiempo real del programa que corre en la plataforma desde la PC. Se basa en el chip FTDI FT2232H \cite{FT2232H}.
\end{itemize}

%Es fabricada y distribuida al por mayor por EXO S.A. [], y distribuida al por menor por Electrocomponentes S.A. []. Se puede comprar en Argentina por AR\$ 11.310,04 (precio de noviembre 2018).

\subsection{EDU-CIAA-NXP}

En base al diseño de la CIAA-NXP los integrantes del proyecto CIAA realizan una versión educativa sin las interfaces y protecciones industriales, nombrada EDU-CIAA-NXP (figura \ref{fig:eduCiaa}).

\begin{figure}[!htbp]
\begin{center}  %[width=14cm,height=8cm]
\includegraphics*[width=\textwidth]{Figures/EDU-CIAA-NXP_Foto.png}
\par\caption{Plataforma EDU-CIAA-NXP.}\label{fig:eduCiaa}
\end{center}
\end{figure}

Utiliza el mismo microcontrolador que la CIAA-NXP, circuito de depuración, interfaz RS-485 y posee 1 USB OTG.

Los pines no utilizados por las interfaces anteriores se disponen en los conectores P1 y P2. En ellos incluye todos los periféricos típicos que podemos encontrar en los microcontroladores disponibles en el mercado (GPIO, ADC, DAC, TIMER, UART, SPI, I2C, etc.). Además posee 1 LED RGB, 3 LEDs y 4 pulsadores.

%Mediante la colaboración de la Red Universitaria de Sistemas Embebidos (RUSE) [] se han distribuido en 2015 entre 10 y 40 placas en Universidades de Argentina con carreras afines a la electrónica.

%Esta plataforma es fabricada y distribuida por Electrocomponentes S.A., en colaboración con Ernesto Mayer S.A. [], Assisi S.A. [] y Asembli S.A. [], y se puede comprar en Argentina por AR\$ 2.240,94 (en noviembre 2018). 

\subsection{PicoCIAA}

La PicoCIAA \citep{PicoCIAA} (figura \ref{fig:picoCiaa}) es una placa en formato mini PCI Express, pensada para ser utilizada como un módulo de cómputo y/o adquisición en una plataforma mayor.

Se basa en el microcontrolador NXP LPC54102J512BD64 \citep{LPC54102J512BD64}, otro microcontrolador \emph{dual-core} asimétrico formado por un procesador Cortex-M4F y un Cortex-M0+ (ambos de 32 bits), que corren con una frecuencia de sistema máxima de 100 MHz; con 512 KB de memoria Flash y 104 KB de SRAM .

\begin{figure}[!htbp]
\begin{center}  % [width=14cm,height=8cm] [width=\textwidth]
\includegraphics*[width=8cm]{Figures/PicoCIAA_Foto.png}
\par\caption{Plataforma PicoCIAA.}\label{fig:picoCiaa}
\end{center}
\end{figure}

Incluye circuito de depuración USB vía un microcontrolador NXP LPC11U35 \citep{LPC11U35} programable.

Posee diversos puertos de comunicación, entre ellos USB, mini PCI Express, UART, SPI, I2C y soporte para PWM y entradas y salidas digitales de propósito general. Su tamaño reducido (51 x 30 mm) es ideal en Single Board Computers.

%Esta plataforma fue diseñada por Pablo Ridolfi y fabricada y distribuida por Vicda Argentina S.A. []. Su precio en Argentina es de AR\$ 2.470,00 (precio en noviembre 2018).

\subsection{CIAA-Z3R0}
\label{sec:CIAA-Z3R0}

La CIAA-Z3R0 se diseñó para ser la plataforma más económica del proyecto CIAA. Esta plataforma es ideal para aplicaciones de bajo consumo (como sensores de IoT) y proyectos de robótica educativa. Está diseñada para ser utilizada como componente en un diseño mayor debido a su tamaño (19.8 x 51.8 mm), soldada a través de su borde de agujeros para montaje \emph{castellated}, o bien, conectándola mediante tiras de pines.
%\footnote{Este tipo de terminación de borde de pin del PCB permite soldar una placa sobre otra, para más información ver [].}

\begin{figure}[!htbp]
\begin{center}  %[width=14cm,height=8cm]
\includegraphics*[width=8cm]{Figures/CIAA-Z3R0_Foto.png}
\par\caption{Plataforma CIAA-Z3R0.}\label{fig:ciaaZero}
\end{center}
\end{figure}

Posee la mayoría de los periféricos que se encuentran en la EDU-CIAA-NXP pero utiliza un microcontrolador siete veces más económico que esta última, de la empresa Silicon Labs, modelo EFM32HG322F64 (QFP48) \citep{EFM32HG322F64} con núcleo ARM Cortex-M0+ a una frecuencia máxima de 25 MHz; 64 KB de memoria Flash y 8 KB de memoria SRAM, que es suficiente para que un alumno entre en el mundo de los microcontroladores modernos de 32 bits.

El \emph{debugger} se debe comprar por separado, sin embargo, mediante un único dispositivo de depuración se pueden programar muchas plataformas CIAA-Z3R0.

%Esta plataforma fue diseñada por el autor y fabricada y distribuida por Asembli S.A. con la gestión de Ariel Lutemberg y Gastón Lagoa. Su precio en Argentina es de AR\$ 650,00 (noviembre de 2018). 

\subsection{Material provisto por los fabricantes}

De las secciones anteriores, se advierte que las cuatro plataformas de hardware presentadas poseen núcleos de procesamiento diseñados por la empresa ARM \citep{ARM} los cuales son licenciados a diversos fabricantes.

ARM ofrece manuales acerca de sus núcleos de procesamiento y diversas bibliotecas para su programación.

Además, tres de estas paltaformas poseen microcontroladores fabricados por la empresa NXP (CIAA-NXP, EDU-CIAA-NXP con LPC4337 y PicoCIAA con LPC5410) y una por Silicon Labs (EFM32HG). 

Tanto NXP como Silicon Labs ofrecen hojas de datos, manuales del sistema y notas de aplicación y bibliotecas para sus microcontroladores.

Se exponen a continuación las principales bibliotecas provistas por las empresas ARM, NXP y Silicon Labs para estos microcontroladores.

\titulo{CMSIS de ARM}

CMSIS son las siglas de \emph{''Cortex Microcontroller Software Interface Standard''}, es decir, interfaz de software para microcontroladores Cortex. Mediante este conjunto de bibliotecas, la empresa ARM intenta estandarizar cómo se programan los microcontroladores que ofrecen las empresas proveedores de silicio licenciantes de sus núcleos de procesamiento Cortex-M. CMSIS se desarrolla públicamente en GitHub. 

Las principales bibliotecas que posee están realizadas en su mayoría lenguaje C y son:

\begin{itemize}
\item
CMSIS-CORE: define el arranque del sistema y acceso periféricos.
\item
CMSIS-RTOS: es una API de abtracción del RTOS\footnote{Siglas en inglés de Sistema Operativo de Tiempo Real.} que permite capas de software coherentes con componentes de \emph{middleware}\footnote{\emph{Middleware} son bibliotecas que asisten a una aplicación, por ejemplo, biblioteca para manejo de sistema de archivos y \emph{stracks} de protocolos.} y bibliotecas de bajo nivel.
\item
CMSIS-DSP: es una colección de funciones de procesamiento de señales digitales, optimizada para núcleos de procesamiento Cortex-M.
\item
CMSIS-Driver: interfaces genéricas de periféricos para \emph{middleware} y código de aplicación.
\end{itemize}

Además provee las siguientes herramientas:

\begin{itemize}
\item
CMSIS-Pack: define la estructura de un paquete de software que contiene componentes de software. Los componentes del software son fácilmente seleccionables, y se resaltan las dependencias de otros paquetes.
\item
CMSIS-SVD: son archivos que habilitan vistas detalladas a los periféricos del dispositivo, que muestran el estado actual de cada registro, y aseguran que la vista del depurador coincida con la implementación real de los periféricos del dispositivo.
\item
CMSIS-DAP: una interfaz estandarizada para el puerto de acceso de depuración de Cortex (DAP) y es utilizada por muchos kits de de desarrollo, siendo compatible con varios dispositivos de hardware para depuración.
\item
CMSIS-NN: es una colección de núcleos de redes neuronales eficientes desarrollada para maximizar el rendimiento y minimizar la huella de memoria de las redes neuronales para núcleos Cortex-M.
\end{itemize}

Para los tres microcontroladores de este trabajo existe soporte completo para la capa Core, incluyendo el núcleo de procesamiento y los drivers de periféricos.

\titulo{Mbed de ARM}

Mbed es una plataforma de drivers y sistema operativo para prototipado rápido, enfocada en dispositivos IoT basados en mcirocontroladores ARM. Es un proyecto de código abierto, también disponible en github \citep{MbedGit}, desarrollado colaborativamente por ARM y sus socios tecnológicos.

Provee abstracción del hardware pero se limita a plataformas del ecosistema ARM. A diferencia de CMSIS, en este caso se provee soporte para plataformas de hardware completas en lugar de solamente ocuparse del microcontrolador. De esta forma muchas empresas añaden soporte a mbed a sus kit de desarrollo con microcontroladores ARM Cortex-M. 

Existen kits de desarrollo que contienen los microcontroladores LPC4337 y EFM32 HG322 con soporte de mbed, los cuales son LPCXpresso4337 \citep{LPCXpresso4337board} y EFM32 USB-enabled Happy Gecko \citep{EFM32HGboard} respectivamente. Para el LPC54102 no existe un kit de desarrollo soportado por mbed, pero existe el kit LPCXpresso54114 \citep{LPCXpresso54114board} que posee un  microcontrolador de la misma familia.

Las plataformas soportadas por Mbed se programan mediante un entorno de desarrollo on line. Las diferentes bibliotecas para microcontroladores provistas en el marco de mbed están escritas en su mayoría en lenguaje C++.

\titulo{LPCOpen y MCUXpresso de NXP}

NXP provee para su línea de microcontroladores LPC las bibliotecas LPCOpen \citep{LPCOpenNXP}, que incluyen drivers para sus microcontroladores, bibliotecas \emph{middelware} de terceros y programas de ejemplo. 

Se puede descargar de forma gratuita de la web de NXP sin registrarse.

LPCOpen se compone de:

\begin{itemize}
\item
Biblioteca de drivers, que se divide en dos capas: una capa de drivers de \emph{chip} que contiene controladores optimizados para un dispositivo o familia específica, y una capa \emph{board} que contiene funciones específicas de un dado kit de desarrollo.
\item
\emph{Middleware}. Esta capa incluye: la biblioteca de objetos gráficos emWin, biblioteca de gráficos SWIM, \emph{stack} de redes de código abierto LWIP y bibliotecas USB (\emph{device} y \emph{host}).
\item
Uso de LPCOpen junto a un RTOS: Incluye ejemplos para utilizar LPCOpen con FreeRTOS.
\item
Ejemplos: incluye un extenso conjunto de ejemplos diseñados para ilustrar cómo usar las funciones de la biblioteca del controlador central y el \emph{middleware}.
\end{itemize}

Para la incialización del sistema utiliza la parte de CMSIS Core que inicializa cada uno de sus núcleos de procesamiento.

Si bien contiene ejemplos para comenzar a utilizar los microcontroladores LPC, sus drivers de periféricos están relacionados directamente con su arquitectura de hardware y proveen una muy baja abstracción de la misma. Además, tiene código duplicado de las bibliotecas de periféricos para cada núcleo de procesamiento de un mismo microcontrolador restándole mantenibilidad.

MCUXpresso \citep{MCUXpresso} de NXP es la evolución de LPCOpen luego de que NXP adquiera a Freescale,  agregando soporte a las líneas de microcontroladores Kinetis e i.MX RT. Requiere registrarse para tener acceso a la misma. Posee soporte únicamente para el microcontrolador LPC54102 de este trabajo.

\titulo{Bibliotecas de Silicon Labs}

Para el mcirocontrolador EFM32HG322 Silicon Labs ofrece las siguientes bibliotecas \citep{EmLibSiliconLabs}

\begin{itemize}
\item
Drivers CMSIS-CORE para EFM32 Happy Gecko.
\item
Biblioteca de periféricos EMLIB, que provee soporte de bajo nivel para periféricos proporcionando una API unificada para todos los MCU y SoC EFM32, EZR32 y EFR32 de Silicon Labs.
\item
Biblioteca EMDRV, conjunto de drivers de alto rendimiento energético específicos para periféricos en chip EFM32, EZR32 y EFR32. Los controladores suelen estar basados en DMA y utilizan todas las funciones disponibles de bajo consumo. La API ofrece funciones síncronas y asíncronas para la mayoría de estos drivers. Además son totalmente reentrantes y basadas en callbacks.
\item
\emph{Platform Middleware}: se compone de una biblioteca de sensado capacitivo (CSLIB), una biblioteca gráfica (GLIB, \emph{stack} USB \emph{device} para dispositivos Gecko y biblioteca de interfaz USBXpress.
\item
\emph{Board Support Package}: El BSP proporciona una API para controladores de una cierta plataforma, incluyendo control de E/S para botones, LED y funcionalidades de \emph{trace} los kits de desarrollo de EFM32, EZR32 y EFR32.
\item
Drivers para componentes de los kits de desarrollo: incluye pantallas, sensores y memorias.
\item
Programas de ejemplo.
\end{itemize}
 
%------------------------------------------------------------------------------
%	SECTION
%------------------------------------------------------------------------------
\section{Requerimientos}
\label{sec:requerimientos}

%\subsection{Diseño de la biblioteca}

%\subsection{Implementación}

%\subsection{Documentación y difusión}

Los requerimientos se establecieron en base a los objetivos expuestos en la sección \ref{sec:objetivosAlance}, reuniones con desarrolladores del Proyecto CIAA, alumnos (de grado de UNQ, posgrado de FIUBA y cursos CAPSE) y el director. El espíritu de los mismos es proponer una solución que permita la estandarización de la programación de microcontroladores. Estos son:

\begin{enumerate}
   \item Fecha de finalización: 19/11/2018.
   \item Diseño de la biblioteca.
      \begin{enumerate}[1]
         \item Realizar un diseño independiente del hardware y lenguaje de programación, teniendo en cuenta los conceptos familiares al programador de Sistemas Embebidos.
         \item Debe estar modelada con objetos y contar con una descripción mediante diagramas UML.
         \item Debe modelar al menos:
            \begin{enumerate}[1]
               \item CORE: un núcleo de procesamiento.
               \item GPIO: periférico que consiste en un único pin de entrada/salida de propósito general (pin), así como un grupo de pines (port).
               \item ADC: periférico conversor analógico-digital.
               \item DAC: periférico conversor digital-analógico.
               \item TIMER: periférico temporizador.CORE: un núcleo de procesamiento.
               \item RTC: periférico reloj de tiempo real.CORE: un núcleo de procesamiento.
               \item UART: periférico de comunicación serial asincrónico.
               \item SPI: periférico interfaz serie sincrónica.
               \item I2C: periférico de comunicación serie entre circuitos integrados.
            \end{enumerate}
      \end{enumerate}
   \item Implementación de la biblioteca.
      \begin{enumerate}[1]
         \item Utilizar un sistema de control de versiones con repositorios on line.
         \item Programar en lenguaje C la biblioteca para cada plataforma de hardware particular utilizando como plantilla los archivos generados.
         \item Desarrollar ejemplos de utilización para las diferentes plataformas.
      \end{enumerate}
   \item Documentación y difusión.
      \begin{enumerate}[1]
         \item Confeccionar un manual de referencia de la API de la biblioteca.
         \item Desarrollar un tutorial de instalación de las herramientas para utilizar la biblioteca con las plataformas de hardware citadas.
         \item Publicación on line del código fuente.
         \item Informar a la comunidad del Proyecto CIAA y a la comunidad de programadores de
   Sistemas Embebidos.
      \end{enumerate}
\end{enumerate}

% \subsection{Entregables principales del trabajo}
% 
% Se listan a continuación los entregables principales del proyecto:
% 
% 1. Archivos de descripción de la biblioteca mediante diferentes diagramas y código
% independiente del lenguaje de programación.
% 2. Código fuente de la implementación en lenguaje C de la biblioteca para las plataformas de hardware:
%    - CIAA-NXP.
%    - EDU-CIAA-NXP.
%    - CIAA-Z3R0.
%    - PicoCIAA.
% 3. Ejemplos de utilización.
% 4. Informe final.
% 5. Manual de referencia de la API de la biblioteca.
% 6. Tutorial de instalación de las herramientas para utilizar la biblioteca con las plataformas de hardware citadas.

 
%------------------------------------------------------------------------------
%	SECTION
%------------------------------------------------------------------------------
\section{Planificación}
\label{sec:planificacion}

Diagrama de Gannt
 

% Diseño
\chapter{Diseño}
\label{ChapterDiseno}

En este capítulo se presenta el diseño de la biblioteca sAPI con un enfoque \emph{top-down}. Se exponen los principios de diseño adoptados, la descripción de una aplicación donde se utiliza la biblioteca, la arquitectura general de la biblioteca, el modelo abstracto de un módulo de biblioteca, el diseño de los módulos de biblioteca de una plataforma de hardware genérica, su verificación y un diseño de archivo de texto para la descripción de una cierta plataforma concreta.

%------------------------------------------------------------------------------
%	SECTION
%------------------------------------------------------------------------------
\section{Descripción general}
\label{sec:descripGralDiseno}

El sistema propuesto para la creación de bibliotecas de software para la programación de plataformas de hardware se ilustra en la figura \ref{fig:sapi_gen}.

\begin{figure}[!htbp]
\begin{center}  % [width=14cm,height=8cm] [width=\textwidth]
\includegraphics*[width=14cm]{Figures/sapi_gen.pdf}
\par\caption{Diagrama del sistema diseñado.}\label{fig:sapi_gen}
\end{center}
\end{figure}

Se diseñó un modelo de plataforma de hardware (en adelante \emph{board} por sus siglas en inglés). Este modelo describe una plataforma de hardware completa y se debe instanciar para cada nueva plataforma de hardware.

Por otra parte, se desarrollaron módulos de biblioteca (en adelante módulos sAPI) independientes del hardware y lenguajes de programación. Estos definen propiedades y métodos describen las propiedades y métodos de cada uno de los periféricos de un cierto SoC, definiendo entonces, la API de la biblioteca.

Además, se debe proveer la implementación de cada uno de los métodos de los módulos sAPI (en adelante \emph{drivers}). Estos \emph{drivers} deben estar escritos en un cierto lenguaje de programación y son dependientes de la arquitectura del hardware.

También, se diseñó un generador de biblioteca sAPI. Este generador se debe definir para cada lenguaje de programación. Para este trabajo final se desarrolló solamente el generador de lenguaje C, sin embargo, el mismo se realizó de forma modular para que pueda ser fácilmente adaptado a otros lenguajes. 

Mediante el generador de C, utilizando una instancia de \emph{board} y \emph{drivers} específicos, junto con los módulos sAPI, se genera entonces una biblioteca sAPI en lenguaje C para una plataforma particular.

En las siguientes secciones de detalla el diseño e implementación de cada una de las entidades descriptas. Las mismas se modelaron utilizaron los conceptos del paradigma de la programación orientada a objetos, que son especialmente útiles para describir módulos de software que encapsulan funcionalidad. Estos se describen mediante diagramas UML de clases de forma independiente del lenguaje de programación.


% Board
%------------------------------------------------------------------------------
%	SECTION
%------------------------------------------------------------------------------
\section{Modelo de plataforma de hardware}
\label{sec:modelHardware}

En la figura \ref{fig:ModelBoard} se muestra el diagrama que describe una \emph{board} y las partes que la componen.

\begin{figure}[!htbp]
\begin{center}  % [width=14cm,height=8cm] [width=\textwidth]
\includegraphics*[width=14cm]{Figures/Board.pdf}
\par\caption{Diagrama de clases de \emph{board}.}\label{fig:ModelBoard}
\end{center}
\end{figure}

Estas partes son:

\begin{itemize}
\item
Componente: representa los componentes dentro de cierta placa, por ejemplo, circuitos integrados, botones, leds, conectores, etc. 
\item
Terminal: describe un terminal de conexión que posee cada componente.
\item
Conexión: modela el mapa de conexiones entre componentes. Cada conexión se compone de los dos terminales conectados.
\end{itemize}

%------------------------------------------------------------------------------
\subsection{Componente}

Un componente incluye nombre, documentación, una lista de terminales de conexión y la cantidad de terminales. En la figura \ref{fig:Component} se expone el diagrama de clases. Existen los siguientes tipos de componentes:

\begin{itemize}
\item
\emph{SystemOnChip}: modela un sistema completo en un chip.
\item
Componente con driver: modela a componentes que necesitan un driver, por ejemplo, LEDs, botones, sensores y memorias montados en la placa.
\item
Conector: representa los conectores físicos de la placa, como ser, tira de pines, borneras, etc. Su importancia en el modelo radica en la descripción de a qué terminales del SoC se conectan cada uno.
\end{itemize}

\begin{figure}[!htbp]
\begin{center}  % [width=14cm,height=8cm] [width=\textwidth]
\includegraphics*[width=11cm]{Figures/Component.pdf}
\par\caption{Diagrama de clases de componente.}\label{fig:Component}
\end{center}
\end{figure}

%------------------------------------------------------------------------------
\subsection{Modelo de SoC}
\label{sec:modelSoC}

Un SoC describe un sistema completo dentro de un chip. Este término se utiliza para describir tanto microcontroladores (que incluyen núcleos de procesamiento, memorias y diversos periféricos), como sistemas que incluyen además lógica programable (FPGA) o módulos analógicos de radiofrecuencia complejos como ser Bluetooth y Wi-Fi.

En la figura \ref{fig:ModelSoC} se muestra el modelo de SoC, el cual se compone de núcleos de procesamiento (\emph{Core}), periféricos (\emph{Peripheral}) y memorias (\emph{Memory}).

\begin{figure}[!htbp]
\begin{center}  % [width=14cm,height=8cm] [width=\textwidth]
\includegraphics*[width=8cm]{Figures/SoC.pdf}
\par\caption{Diagrama de clases de SoC.}\label{fig:ModelSoC}
\end{center}
\end{figure}

%------------------------------------------------------------------------------
\subsection{Modelo de IP core}
\label{sec:modelSoC}

Para modelar núcleos de procesamiento, periféricos y memorias se utiliza el concepto de \emph{IP core}. Un núcleo de propiedad intelectual es un bloque de lógica o datos reutilizable, para definir un diseño de hardware de forma abstracta. Normalmente se utilizan en electrónica como componentes en un diseño de hardware ya sea en una FPGA o ASIC. Define una interfaz de señales y comportamiento interno.

En la figura \ref{fig:IPCore} se muestra el modelo de \emph{IPCore} y sus relaciones.

\begin{figure}[!htbp]
\begin{center}  % [width=14cm,height=8cm] [width=\textwidth]
\includegraphics*[width=14cm]{Figures/IPCore.pdf}
\par\caption{Diagrama de clases de \emph{IPcore}.}\label{fig:IPCore}
\end{center}
\end{figure}

Un \emph{IPCore} define una dirección base, que se utiliza para localizarlo en el mapa de memoria. También define una interfaz que se compone de una lista ordenada de señales.

Las clases \emph{Memory} e \emph{IPCoreExtended} heredan de la clase \emph{IPCore} agregándole funcionalidades particulares. 

\emph{Memory} modela una memoria interna del SoC, por ejemplo, RAM o Flash. Tiene las propiedades: tipo de memoria, tamaño en bytes y tipo de acceso permitido (lectura, lectura/escritura).

\emph{IPCoreExtended} contiene registros e interrupciones. Los registros se modelan tanto para fines de documentación, como para poder realizar \emph{tests} sobre valores de los mismos. Las interrupciones son necesarias para conocer cuales tenemos disponibles físicamente y a qué evento responden. De la clase \emph{IPCoreExtended} heredan \emph{Core} y \emph{Peripheral}. La primera modela un núcleo de procesamiento y la segunda un periférico. 

\emph{Peripheral} contiene una lista ordenada de \emph{Locations} (ubicaciones). Una ubicación define un conjunto de conexiones entre señales del periférico y terminales del SoC. Esto permite modelar las posibles configuraciones de pines de un dado periférico.

%------------------------------------------------------------------------------
\subsection{Archivos para la descripción de una plataforma de hardware}

Para definir una plataforma de hardware se utilizan archivos de texto en formato \emph{JSON} [], de esta manera se puede definir cada una de las entidades presentadas por separado y referenciarlas. 

Como ejemplo, se expone a continuación los archivos propuestos para una pequeña plataforma de hardware ficticia nombrada ''fake-board-v1'' (se ilustra en la figura \ref{fig:fakeBoard}). 

\begin{figure}[!htbp]
\begin{center}  % [width=14cm,height=8cm] [width=\textwidth]
\includegraphics*[width=8cm]{Figures/FakeBoard-01.pdf}
\par\caption{Fake-board-v1.}\label{fig:fakeBoard}
\end{center}
\end{figure}

Archivo de descripción de \emph{board} nombrado \emph{fakeBoard.json}:

\begin{verbatim}
{
  "type": "Board",
  "name": "fake-board-v1",
  "components": [
    {
      "type": "@/vendor1/soc/fakeSoc.json",
      "name": "SOC0"
    },
    {
      "type": "@/vendor2/phy/fakePhy232.json",
      "name": "PHY0"
    },
    {
      "name": "RS232",
      "type": "@/vendor3/connector/fakeDb9Male.json"
    }
  ],
  "connections": [
    [ "#/SOC0/P1", "#/PHY0/RX_LVL" ],
    [ "#/SOC0/P2", "#/PHY0/TX_LVL" ],
    [ "#/SOC0/P3", "#/PHY0/GND" ],
    [ "#/PHY0/RX_232", "#/RS232/P2" ],
    [ "#/PHY0/TX_232", "#/RS232/P3" ],
    [ "#/PHY0/GND", "#/RS232/P5" ]
  ]
}
\end{verbatim}

''@'' indica una referencia a un archivo externo con el path incluido, mientras que ''\#'' indica una referencia a un terminal de un componente.

Archivo de descripción de \emph{SoC} nombrado \emph{fakeSoc.json}:

\begin{verbatim}
{
  "type": "SystemOnChip",
  "name": "fake-board-v1",
  "cores": [
    {
      "type": "@/arch/arm/armv7m.json",
      "name": "CORTEXM4"
    }
  ],
  "peripherals": [
    {
      "type": "@/vendor1/ipcore/uart.json",
      "name": "UART0"
    }
  ],
  "memories": [
    {
      "type": "@/vendor1/mem/sram8kb.json",
      "name": "RAM"
    },
    {
      "type": "@/vendor1/mem/flash32kb.json",
      "name": "FLASH"
    }
  ],
  "terminals": [ "P1", "P2", "P3" ]
}
\end{verbatim}

Archivo de descripción de un chip de capa física para RS-232  nombrado \emph{fakePhy232.json}:

\begin{verbatim}
{
  "type": "ComponentWithDriver",
  "name": "phy-rs232",
  "terminals": [ 
    "RX_LVL", "TX_LVL", "GND",
    "RX_232", "TX_232"
  ],
\end{verbatim}

\pagebreak 
\begin{verbatim}
  "drivers": [
    {
      "lang":"c",
      "files": [
        "@/driver/c/vendor2/phy/fakePhy232.h",
        "@/driver/c/vendor2/phy/fakePhy232.c"      
      ]      
    }
  ]
}
\end{verbatim}

Archivo de descripción de un conector \emph{DB-9} nombrado \emph{fakeDb9Male.json}:

\begin{verbatim}
{
  "type": "Connector",
  "name": "DB9-male",
  "terminals": [ 
    "P1", "P2", "P3", "P4", "P5", "P6", "P7", "P8", "P9"
  ]
}
\end{verbatim}

Cabe destacar que el ejemplo presentado es válido solamente para describir la estructura de los archivos y filosofía de diseño adoptada, sin embargo, los archivos de implementación poseen más propiedades y niveles de anidamiento en sus entidades.

% Modulos sAPI
%------------------------------------------------------------------------------
%	SECTION
%------------------------------------------------------------------------------
\section{Modelo abstracto de biblioteca}
\label{sec:modelLibrary}

%------------------------------------------------------------------------------
\subsection{Módulo de biblioteca}

%------------------------------------------------------------------------------
\subsection{Biblioteca}
 
% Generador de sAPI
%------------------------------------------------------------------------------
%	SECTION
%------------------------------------------------------------------------------
\section{Generador de sAPI en lenguaje C}
\label{sec:modelGenerator}

El generador de sAPI en lenguaje C toma como entrada el una instancia de \emph{board} y \emph{drivers} específicos, junto con los \emph{módulos sAPI}, y genera como resultado una estructura de archivos y carpetas con la biblioteca sAPI en lenguaje C para una plataforma particular. También permite generación de documentación a partir del mismo modelo.

Para su diseño se debió estandarizar la forma de generar un módulo de código en lenguaje C.
%(sección \ref{sec:moduloEnC}). 
Una vez realizado, se modelaron las clases para generar un módulo en lenguaje C a partir de los modelos desciptos.
%(\ref{sec:modelModuloEnC}). 

%------------------------------------------------------------------------------
\subsection{Módulo de sAPI en lenguaje C}
\label{sec:moduloEnC}

Un módulo de código en lenguaje C posee 

%------------------------------------------------------------------------------
\subsection{Modelo de módulo en lenguaje C}
\label{sec:modelModuloEnC}

%------------------------------------------------------------------------------
\subsubsection{Generación automática de código fuente}
\label{sec:genCIndep}

Generación de código en lenguaje C independiente del hardware.


%------------------------------------------------------------------------------
\subsection{Generación de documentación en base al modelo}
\label{sec:genDoc}



%------------------------------------------------------------------------------
\subsection{Implementación del generador}

Se elige para la implementación el lenguaje \emph{Javascrip} [] y el entorno de ejecución \emph{NodeJS} []. El motivo principal de esta decisión es permitir en un futuro realizar la definición de plataformas y generación de código desde una plataforma web. Estas herramientas permiten generar aplicaciones de escritorio, web o móviles con mínimos esfuerzos.

Para llevarlo a cabo se implementaron en lenguaje \emph{Javascrip} todas las clases presentadas en las secciones \ref{sec:modelHardware} y \ref{sec:modelLibrary}, así como las clases que definen al generador de la biblioteca en lenguaje C.
 

%------------------------------------------------------------------------------
% Section
%------------------------------------------------------------------------------
\section{Verificación del diseño de la biblioteca}

Para la verificación del cumplimiento de los requerimientos del grupo \emph{REQ.2}, se llevó a cabo una revisión por pares, esto es, se envió la descripción de la API a colegas sometiéndola a sus revisiones. 

El proceso de revisión por pares se llevó a cabo de forma iterativa e incremental, es decir, se sostuvieron reuniones iniciales, luego se envió una versión inicial, se discutieron posibles mejoras y cambios, se llevaron a cabo los cambios y se volvieron a someter a revisión. Este proceso se llevó a cabo al menos dos veces por cada módulo con dos revisores distintos. El autor agradece enormemente el trabajo de revisión llevado a cabo por Martín Ribelotta y Alejandro Celery, ambos profesionales con muchísima experiencia.



%------------------------------------------------------------------------------
%	SECTION
%------------------------------------------------------------------------------
\section{Diseño de archivo de texto para la descripción de una plataforma de hardware}
\label{sec:modelFileDesign}

%------------------------------------------------------------------------------
%\subsection{}


% Implementación
\chapter{Implementación}
\label{ChapterImplementacion}

%------------------------------------------------------------------------------
%	SECTION
%------------------------------------------------------------------------------
\subsection{Archivos de descripción de las plataformas del proyecto CIAA}

Para las plataformas del proyecto CIAA se 

\input{Chapters/Implementacion/CIAA_NXP_BoardFile}
\input{Chapters/Implementacion/EDU_CIAA_NXP_BoardFile}
\input{Chapters/Implementacion/PicoCIAA_BoardFile}
%------------------------------------------------------------------------------
\subsection{CIAA-Z3R0}


%------------------------------------------------------------------------------
%	SECTION
%------------------------------------------------------------------------------
\section{Generación automática de código fuente}
\label{sec:codeGeneration}

Generación de código en lenguaje C independiente del hardware.

%------------------------------------------------------------------------------
%\subsection{}


%------------------------------------------------------------------------------
\subsection{Clasificación de módulos de hardware}

Modulos instanciables modulos fijos.

Periféricos virtuales: estos corresponden a periféricos que no contiene el MCU o SoC de la plataforma pero se pueden implementar por software para suplirlos, por ejemplo, un I2C implementado por software que para funcionar utiliza GPIOs.


%------------------------------------------------------------------------------
%	SECTION
%------------------------------------------------------------------------------
\section{Implementación del código C dependiende del hardware}
\label{sec:codeImplementation}

%------------------------------------------------------------------------------
%\subsection{}

En la implementación para las plataformas del proyecto CIAA, del código dependiente del hardware, que forma parte de la biblioteca sAPI, se utilizaron las bibliotecas de drivers provistas por los fabricantes. En particular, para los microcontroladores de la empresa NXP se utilizó LPCOpen, en la version 3.02 para el LPC4337 y versión 3.04 para el LPC54102. Para el microcontrolador EFM32HG322 de Siicon Labs se utilizó la biblioteca EMLIB versión 5.1.2 (todas en sus últimas versiones al momento de la realización de este trabajo).

Por otra parte, se reutilizó parte del código desarrollado en versiones anteriores de la biblioteca sAPI [] para las plataformas EDU-CIAA-NXP y CIAA-Z3R0 y parte del código de PicoAPI [] para la implementación en la plataforma PicoCIAA.

Una aplicación de embebidos típica que utiliza la biblioteca sAPI, se puede combinar con un sistema operativo de tiempo real, \emph{stracks} y \emph{middelware} resultando una arquitectura de capas de software típica en aplicaciones de sistemas embebidos como se ilustra en la figura \ref{fig:sapiCapas2}.

\begin{figure}[!htbp]
\begin{center}  % [width=14cm,height=8cm] [width=\textwidth]
\includegraphics*[width=10.4cm]{Figures/sapiCapas2.png}
\par\caption{Arquitectura de una aplicación que utiliza la biblioteca.}\label{fig:sapiCapas2}
\end{center}
\end{figure}


%------------------------------------------------------------------------------
%	SECTION
%------------------------------------------------------------------------------
\section{Ejemplos de utilización}
\label{sec:libExamples}

%En esta sección se exponen los ejmplos de uso de la biblioteca realizados y los criterios para realizarlos independientes de las plataforma de hardware.




%------------------------------------------------------------------------------
%	SECTION
%------------------------------------------------------------------------------
\subsection{Documentación y difusión}
\label{sec:documentation}

Verificación del cumplimiento de los requerimientos del grupo \emph{REQ.4.}

REQ.4. Documentación y difusión.

Producción de código y su documentación: escribir la documentación a la par del código fuente.

Documentación general en lenguaje \emph{Markdown} \citep{MARKDOWN}, que permite exportar el contenido a \emph{html}, \LaTeX, \emph{pdf}, entre otros.

Definir un documento de estilo del código fuente.

%------------------------------------------------------------------------------
\titulo{Manual de referencia de la API}

REQ.4.1. Confeccionar un manual de referencia de la biblioteca.


%------------------------------------------------------------------------------
\titulo{Tutoriales de instalación y uso}

REQ.4.2. Desarrollar un tutorial de instalación de las herramientas para utilizar la biblioteca.


%------------------------------------------------------------------------------
\titulo{Difusión}

Difusión a la comunidad del Proyecto CIAA y Embebidos32

REQ.4.3. Publicación on line del código fuente.

REQ.4.4. Informar a la comunidad del Proyecto CIAA y a la comunidad de programadores de Sistemas Embebidos.


% Ensayos y Resultados
\chapter{Ensayos y Resultados}
\label{ChapterEnsayosYResultados}

\emph{Pruebas funcionales del hardware: La idea de esta sección es explicar cómo se hicieron los ensayos, qué resultados se obtuvieron y analizarlos.}

%------------------------------------------------------------------------------
%	SECTION
%------------------------------------------------------------------------------
\section{Testeo Unitario}
\label{sec:unitTest}


%------------------------------------------------------------------------------
%	SECTION
%------------------------------------------------------------------------------
\section{Banco de pruebas de hardware}
\label{sec:testBench}


%------------------------------------------------------------------------------
%	SECTION
%------------------------------------------------------------------------------
\section{Integración contínua}
\label{sec:ci}


%------------------------------------------------------------------------------
%	SECTION
%------------------------------------------------------------------------------
\section{Utilización de la biblioteca para la enseñanza de programación de Sistemas Embebidos}
\label{sec:teach}


----------------------------------------------

CASOS DE USO (Paper sAPI)

Desde la primer versión de la biblioteca sAPI realizada en el marco del proyecto de Java[] en 2015 se ha utilizado como ejemplo de capa de abstracción de hardware en las asignaturas dictadas por el autor. Estas asignaturas son: el  curso de posgrado “Programación de microproce- sadores”[] de la Carrera Especialización en Sistemas Embebidos (CESE[]) de la FI-UBA. Donde el autor se ha desempeñado como docente a cargo del curso durante tres ediciones, y la asignatura “Sistemas Digitales” de la carrera Ingeniería en Automatización y Control Industrial (IACI[]) de la Universidad Nacional de Quilmes (UNQ) donde el autor se desempeña en la actualidad como instructor.

A partir mazo de 2016 se decide utilizar la biblioteca como en el marco de los Cursos Abiertos de Programación de Sistemas Embebidos (CAPSE[]) organizados por la ACSE[]. De esta forma se ha extendido la biblioteca de forma considerable para explicar la utilización de todos los periféricos típicos de microcontroladores con excelentes resultados en cuanto a aprendizaje por parte de los alumnos tanto de niveles avanzados como quienes dan sus primeros pasos en el aprendizaje de programación de microcontroladores.
Además, la biblioteca sAPI se puso a disposición de cualquier persona ya que se encuentra publicada de forma libre y gratuita por internet bajo una licencia BSD modificada[] en el sitio de github del autor[] y ha sido adoptada por una gran cantidad de usuarios. 

Finalmente, en diciembre de 2016 se decide utilizar la biblioteca como biblioteca estándar para las plataformas del Proyecto CIAA. Esto llevó a una profunda revisión y mejora de la misma y todavía se está trabajando en la actualidad.


----------------------------------------------


CASOS DE USO (paper tools)

Estas herramientas se han utilizado en múltiples, entre
ellos, el curso "Sistemas digitales" de la carrera Ingeniería en
Automatización y Control Industrial en la UNQ, donde el
autor actualmente se desempeña como Profesor Instructor
desde 2014. El plan de estudio de este curso incluye
programación avanzada en lenguaje C para
microcontroladores, con temas tales como modularización de
código, máquinas de estado finito, sistemas operativos en
tiempo real que usan programación cooperativa y apropiativa
en microcontroladores; el curso de posgrado "Programación
de microprocesadores" dentro de la CESE de FI-UBA donde
el autor ha sido profesor a cargo de tres ediciones. Este curso
incluye: programación básica de microcontroladores en
lenguaje C, modularización, máquinas de estados finitos.
También en cursos organizados por ACSE []. En estos
cursos las herramientas junto con la secuencia didáctica
propuesta fueron intensamente utilizadas, mejorando esta
secuencia entre el autor y Pablo Gómez. En los cursos de
ACSE se distinguen dos grupos, "Cursos Abiertos de
Programación de Sistemas Embebidos" (CAPSE), orientados a
cualquier persona interesada en aprender la programación de
Sistemas Embebidos; y cursos impartidos al "Instituto
Nacional de Escuelas Técnicas" (INET) [], enfocados en la
capacitación de docentes de Escuelas Técnicas Secundarias,
para posteriormente poder retransmitir a sus alumnos en todo
el país. En ambos casos, se han observado muy buenos
resultados de aprendizaje. En el caso de los cursos CAPSE, se
impartieron cuatro cohortes entre 2016 y 2017; en la primera
cohorte, el 46\% de los estudiantes completaron todos los
niveles con un promedio de 67\% de índice de aprobación por
nivel; mientras que en la cuarta cohorte, el 73\% completó
todos los niveles, con una tasa de aprobación promedio de
81\% por nivel. En el caso de los cursos INET, con dos
cohortes en 2017, el 68\% de los estudiantes completaron todos
los niveles, con una tasa promedio de aprobación del 74\% por
nivel. Además, después de completar sus estudios, algunos de
estos estudiantes se inscribieron posteriormente en la CESE
FI-UBA para continuar profundizando su aprendizaje.
Además, se han realizado varios talleres en escuelas
secundarias y universidades de todo el país. Vale la pena
mencionar los múltiples tutoriales y workshop en el "Simposio
Argentino de Sistemas Embebidos" (SASE) [].
Todas estas herramientas son de código abierto, publicadas de
forma gratuita en la cuenta web de github del Proyecto CIAA
y han sido adoptadas por un número importante de usuarios y
docentes.

% Conclusiones
\chapter{Conclusiones} % Main chapter title
\label{ChapterConclusiones} % Change X to a consecutive number; for referencing
                            % this chapter elsewhere, use \ref{ChapterX}

%----------------------------------------------------------------------------------------
%	SECTION
%----------------------------------------------------------------------------------------

\section{Trabajo realizado}

\emph{La idea de esta sección es resaltar cuáles son los principales aportes del trabajo realizado y cómo se podría continuar. Debe ser especialmente breve y concisa. Es buena idea usar un listado para enumerar los logros obtenidos.}

%----------------------------------------------------------------------------------------
%	SECTION
%----------------------------------------------------------------------------------------
\section{Próximos pasos}

\emph{Acá se indica cómo se podría continuar el trabajo más adelante.}
 

%------------------------------------------------------------------------------
%	CONTENIDO DE LA MEMORIA  - APÉNDICES
%------------------------------------------------------------------------------

% indicativo para indicarle a LaTeX los siguientes "capítulos" son apéndices
\appendix 

% Incluir los apéndices de la memoria como archivos separadas desde la carpeta
% Appendices
% Descomentar las líneas a medida que se escriben los apéndices

%\include{Appendices/AppendixA}
%\include{Appendices/AppendixB}
%\include{Appendices/AppendixC}

%------------------------------------------------------------------------------
%	BIBLIOGRAPHY
%------------------------------------------------------------------------------

\Urlmuskip=0mu plus 1mu\relax
\raggedright
\printbibliography[heading=bibintoc]

%------------------------------------------------------------------------------

\end{document}  
