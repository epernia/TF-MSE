%------------------------------------------------------------------------------
%	SECTION
%------------------------------------------------------------------------------
\section{Validación}
\label{sec:validacion}

La validación de la biblioteca se llevó a cabo mediante la realización de ejemplos funcionales, los cuales fueron utilizados por alumnos y docentes, como se describe en las siguientes secciones.

%------------------------------------------------------------------------------
\subsection{Ejemplos de uso de la biblioteca}
\label{sec:libExamples}

Los ejemplos fueron diseñados independientes de la arquitectura, de esta manera con que exista al menos un periférico de cada tipo en la plataforma particular se podrán ejecutar sin cambios. Éstos son:

\begin{itemize}
\item
GPIO
\begin{itemize}
\item \emph{gpio\_switche\_led}: pulsador controlando el estado de un LED.
\item \emph{gpio\_blinky}: LED destellando con retardo bloqueante.
\item \emph{gpio\_blinky\_switch}: LED destellando con retardo no bloqueante que permite leer el estado del pulsador en simultáneo.
\item \emph{gpio\_led\_sequences}: secuencia de LEDs usando retardo no bloqueante.
\item \emph{dht11\_temp\_humidity}: sensor de humedad y temperatura.
\item \emph{keypad\_7segment}: teclado matricial y display 7 segmentos.
\item \emph{lcd}: display LCD alfanumérico.
\end{itemize}
\item
ADC
\begin{itemize}
\item \emph{adc}: ADC.
\item \emph{dac}: DAC.
\item \emph{adc\_dac}: ADC y DAC.
\end{itemize}
\item
UART
\begin{itemize}
\item \emph{uart\_echo}: eco por UART, el periférico responde lo mismo que se le envía.
\item \emph{uart\_receive\_string\_blocking}: esperar hasta recibir un cierto patrón de \emph{String} o expire el tiempo de \emph{timeout}.
\item \emph{uart\_receive\_string\_non\_blocking}: esperar hasta recibir un cierto patrón de \emph{String} de manera no bloqueante.
\item \emph{bluetooth\_ble\_hm10}: módulo \emph{Bluetoth} BLE HM10 con comandos AT.
\item \emph{wifi\_esp8266}: módulo \emph{Wi-Fi} ESP01 (ESP8266) con comandos AT.
\item \emph{wifi\_esp8266\_02\_thingspeak}: módulo ESP01 (ESP8266), envío de datos al servidor IoT \emph{ThingSpeak}.
\end{itemize}
\item
SPI: \emph{eeprom\_at25fs010}: memoria EEPROM SPI.
\item
I2C
\begin{itemize}
\item \emph{imu\_mpu9250}: IMU MPU9250 de 9 grados de libertad (giróscopo, acelerómetro, magnetómetro) y temperatura.
\item \emph{magnetometer\_hmc5883l}: magnetómetro HMC5883L. 
\item \emph{magnetometer\_qmc5883l}: magnetómetro QMC5883L.
\end{itemize}
\item
RTC: \emph{rtc}: establecimiento y lectura de fecha/hora.
\item
TIMER
\begin{itemize}
\item \emph{pwm\_led}: PWM aplicado a LEDs.
\item \emph{pwm\_rgb}: tres salidas PWM aplicadas a un LED RGB.
\item \emph{servo}: PWM para control de servomotor angular.
\item \emph{ultrasonic\_sensor\_hcsr04}: sensor ultrasónico de distancia HC-SR04 (modo entrada de captura).
\end{itemize}
\end{itemize}

Para su utilización se debe revisar la la documentación de la plataforma particular, observando la ubicación física de cada periférico definido en la biblioteca y sus terminales de conexión. Con el afán de facilitar la labor, se realizó un diagrama gráfico que mapea periféricos definidos en la biblioteca, a ubicación y conexiones para cada plataforma, un ejemplo para la plataforma CIAA-Z3R0 se expone en la figura \ref{fig:mapaSapiBoard}.

\begin{figure}[!htbp]
\begin{center}  % [width=14cm,height=8cm] [width=\textwidth]
\includegraphics*[width=14cm]{Figures/mapaSapiBoard.pdf}
\par\caption{Diagrama de biblioteca sAPI para la plataforma CIAA-Z3R0.}\label{fig:mapaSapiBoard}
\end{center}
\end{figure}

En todas las plataformas se utilizó un diodo LED y un pulsador (SW) ubicando sus pines como IO0 e IO1, respectivamente. Además, en los periféricos se observa el ''\#'' para indicar el valor de \emph{location} del periférico definido en la biblioteca, en este caso particular se dan dos posibles pares de pines por los que se puede utilizar el periférico \emph{UART0}.

%------------------------------------------------------------------------------

\subsection{Utilización de la biblioteca}
\label{sec:teach}

Desde marzo de 2017 el autor mantiene el repositorio \emph{cese-edu-ciaa-template} \citep{repoCESE} para la utilización de alumnos y docentes de la CESE de FIUBA (posgrado) e IACI de UNQ \citep{IACI} (grado) en sus diferentes asignaturas. 

Cada nueva funcionalidad de la  biblioteca se volcó en ese repositorio, de forma de mantener actualizados a los usuarios. Se prestó atención en mantener la coherencia de la API a lo largo del ciclo lectivo, evitando volcar los cambios que afecten la compatibilidad con versiones previas. En casos muy particulares, donde los cambios se justificaban se informó a todos los usuarios por correo electrónico.

Los alumnos han utilizado la biblioteca tanto para trabajos prácticos de materias, como para sus trabajos finales de carrera, con resultados excelentes. El entusiasmo de los mismos se plasmó, en algunos casos, en la contribución de \emph{drivers} realizados sobre la biblioteca, los cuales fueron incluidos al repositorio. Sus experiencias y propuestas, junto con las de sus respectivos docentes se han volcado en revisiones de la biblioteca.

%----------------------------------------------
%
%CASOS DE USO (Paper sAPI)
%
%Desde la primer versión de la biblioteca sAPI realizada en el marco del proyecto de Java[] en 2015 se ha utilizado como ejemplo de capa de abstracción de hardware en las asignaturas dictadas por el autor. Estas asignaturas son: el  curso de posgrado “Programación de microproce- sadores”[] de la Carrera Especialización en Sistemas Embebidos (CESE[]) de la FI-UBA. Donde el autor se ha desempeñado como docente a cargo del curso durante tres ediciones, y la asignatura “Sistemas Digitales” de la carrera Ingeniería en Automatización y Control Industrial (IACI[]) de la Universidad Nacional de Quilmes (UNQ) donde el autor se desempeña en la actualidad como instructor.
%
%A partir mazo de 2016 se decide utilizar la biblioteca como en el marco de los Cursos Abiertos de Programación de Sistemas Embebidos (CAPSE[]) organizados por la ACSE[]. De esta forma se ha extendido la biblioteca de forma considerable para explicar la utilización de todos los periféricos típicos de microcontroladores con excelentes resultados en cuanto a aprendizaje por parte de los alumnos tanto de niveles avanzados como quienes dan sus primeros pasos en el aprendizaje de programación de microcontroladores.
%Además, la biblioteca sAPI se puso a disposición de cualquier persona ya que se encuentra publicada de forma libre y gratuita por internet bajo una licencia BSD modificada[] en el sitio de github del autor[] y ha sido adoptada por una gran cantidad de usuarios. 
%
%Finalmente, en diciembre de 2016 se decide utilizar la biblioteca como biblioteca estándar para las plataformas del Proyecto CIAA. Esto llevó a una profunda revisión y mejora de la misma y todavía se está trabajando en la actualidad.
%
%
%----------------------------------------------
%
%
%CASOS DE USO (paper tools)
%
%Estas herramientas se han utilizado en múltiples, entre
%ellos, el curso "Sistemas digitales" de la carrera Ingeniería en
%Automatización y Control Industrial en la UNQ, donde el
%autor actualmente se desempeña como Profesor Instructor
%desde 2014. El plan de estudio de este curso incluye
%programación avanzada en lenguaje C para
%microcontroladores, con temas tales como modularización de
%código, máquinas de estado finito, sistemas operativos en
%tiempo real que usan programación cooperativa y apropiativa
%en microcontroladores; el curso de posgrado "Programación
%de microprocesadores" dentro de la CESE de FI-UBA donde
%el autor ha sido profesor a cargo de tres ediciones. Este curso
%incluye: programación básica de microcontroladores en
%lenguaje C, modularización, máquinas de estados finitos.
%También en cursos organizados por ACSE []. En estos
%cursos las herramientas junto con la secuencia didáctica
%propuesta fueron intensamente utilizadas, mejorando esta
%secuencia entre el autor y Pablo Gómez. En los cursos de
%ACSE se distinguen dos grupos, "Cursos Abiertos de
%Programación de Sistemas Embebidos" (CAPSE), orientados a
%cualquier persona interesada en aprender la programación de
%Sistemas Embebidos; y cursos impartidos al "Instituto
%Nacional de Escuelas Técnicas" (INET) [], enfocados en la
%capacitación de docentes de Escuelas Técnicas Secundarias,
%para posteriormente poder retransmitir a sus alumnos en todo
%el país. En ambos casos, se han observado muy buenos
%resultados de aprendizaje. En el caso de los cursos CAPSE, se
%impartieron cuatro cohortes entre 2016 y 2017; en la primera
%cohorte, el 46\% de los estudiantes completaron todos los
%niveles con un promedio de 67\% de índice de aprobación por
%nivel; mientras que en la cuarta cohorte, el 73\% completó
%todos los niveles, con una tasa de aprobación promedio de
%81\% por nivel. En el caso de los cursos INET, con dos
%cohortes en 2017, el 68\% de los estudiantes completaron todos
%los niveles, con una tasa promedio de aprobación del 74\% por
%nivel. Además, después de completar sus estudios, algunos de
%estos estudiantes se inscribieron posteriormente en la CESE
%FI-UBA para continuar profundizando su aprendizaje.
%Además, se han realizado varios talleres en escuelas
%secundarias y universidades de todo el país. Vale la pena
%mencionar los múltiples tutoriales y workshop en el "Simposio
%Argentino de Sistemas Embebidos" (SASE) [].
%Todas estas herramientas son de código abierto, publicadas de
%forma gratuita en la cuenta web de github del Proyecto CIAA
%y han sido adoptadas por un número importante de usuarios y
%docentes.

