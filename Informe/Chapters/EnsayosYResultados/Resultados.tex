%------------------------------------------------------------------------------
%	SECTION
%------------------------------------------------------------------------------
\section{Resultados}

En esta sección se 

%------------------------------------------------------------------------------
\subsection{Características de la implementación de sAPI en lenguaje C}
\label{sec:codeImplemC}

Un usuario de la biblioteca sAPI puede elegir el nivel de abstracción deseado a la hora de programar su aplicación. De menor a mayor abstracción tendremos programas:

\begin{enumerate}
\item
Dependientes del chip: utilizando la API genérica y los nombres de los pines y periféricos del \emph{chip}, definidos por el fabricante (opcionalmente con el agregado de funciones específicas de periféricos de cierto \emph{chip}).
\item
Dependientes de la placa: mediante la API genérica y los nombres de la serigrafía de la placa, incluyendo pines y periféricos del chip, así como otros componentes de la placa (por ejemplo, otros chips, conectores, puertos de comunicación).
\item
Portables entre placas compatibles: usando la API genérica y los nombres genéricos para todos.
\item
Totalmente portables: Mediante la API genérica y es responsabilidad del usuario agregar un archivo de nombres, por ejemplo, remap.h donde elija los nombres de los nombres genéricos para todo lo que va a usar.
\end{enumerate}

VER SI PONGO ALGUNOS EJEMPLOS

