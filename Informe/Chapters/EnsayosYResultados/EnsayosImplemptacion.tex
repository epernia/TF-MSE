%------------------------------------------------------------------------------
% Section
%------------------------------------------------------------------------------
\section{Verificación de los requerimientos de implementación de la biblioteca}

Para el cumplimiento de los requerimientos del grupo \emph{REQ3}, se utilizaron las herramientas y servicios que se ilustran en la figura \ref{fig:ciTools}. Todas estas herramientas y servicios seleccionados son de código abierto y uso gratuito. Además se definió un flujo de trabajo para el desarrollo y las contribuciones.

\begin{figure}[!htbp]
\begin{center}  % [width=14cm,height=8cm] [width=\textwidth]
\includegraphics*[width=14cm]{Figures/ciTools.pdf}
\par\caption{Metodología y herramientas.}\label{fig:ciTools}
\end{center}
\end{figure}

%------------------------------------------------------------------------------
\subsection{Herramientas de producción de software para sistemas embebidos en lenguaje \emph{C}}

\begin{itemize}
\item
IDE\footnote{Siglas en inglés de Entorno de Desarrollo Integrado.}: \emph{GNU MCU Eclipse} \citep{GNUMCUEclipse}, una distribución de Eclipse IDE con \emph{plugins} para la programación en lenguaje C o C++ para las arquitecturas ARM y RISC-V. Permite tanto compilar como depurar el programa.
\item
Compilación y depuración:
\begin{itemize}
\item
\emph{Make} \citep{Make}, para gestión de dependencias. Los proyectos en C son todos proyectos en base a \emph{Makefile}.
\item
\emph{GNU ARM Embedded} \citep{Toolchain}, herramientas para compilación cruzada para procesadores Arm Cortex-M y Cortex-R.
\item
\emph{OpenOCD} \citep{OpenOCD}, herramienta para depuración, programación y \emph{boundary-scan} para sistemas embebidos.
\end{itemize}
\item
\emph{Tests}: \emph{Minut} \citep{Minut}, es una biblioteca mínima de pruebas unitarias para el lenguaje C.
\end{itemize}

%------------------------------------------------------------------------------
\subsection{Herramientas para producción de software en lenguaje \emph{JavaScript}}

\begin{itemize}
\item
IDE: \emph{Visual Studio Code} \citep{vscode}, permite su utilización con múltiples lenguajes de programación, incluyendo \emph{JavaScript}.
\item
Entorno de ejecución: \emph{Node.js}, permite la ejecución de programas en lenguaje \emph{JavaScript}.
\item
\emph{Tests}:
\begin{itemize}
\item
\emph{Mocha.js} \citep{Mocha}, es un \emph{framework} de pruebas de \emph{JavaScript} que se ejecuta sobre \emph{Node.js}.
\item
\emph{Chai.js} \citep{Chai}, es una biblioteca de aserciones, la cual se puede utilizar con cualquier \emph{framework} de pruebas de \emph{Javascript}.
\end{itemize}
\end{itemize}

%------------------------------------------------------------------------------
\subsection{Servicios en la nube}

Se utilizan cinco repositorios de código alojados en la nube utilizando \emph{GitHub}, uno que incluye la definición de los módulos de biblioteca independientes del lenguaje de programación, los módulos que definen las plataformas y el generador de la biblioteca; y otros cuatro para mantener la versión de biblioteca sAPI para cada plataforma del proyecto CIAA que abarca este trabajo.

Cada uno de estos repositorios se integra con \emph{Travis CI} \citep{TravisCI}, que es un sistema de integración continua, gratuito para proyectos de código abierto, y en la nube. Permite construir y probar programas de forma automatizada. Ante cada \emph{push} o \emph{pull request} en \emph{GitHub}, \emph{Travis CI} ejecuta automáticamente el \emph{pipeline} definido para cada repositorio. 

En el repositorio del generador de módulos de sAPI \emph{Travis CI} realiza las siguientes tareas:

\begin{itemize}
\item
Crear un entorno de ejecución con el sistema operativo Linux \citep{GNULINUX}.
\item
Clonar el repositorio.
\item
Descargar e instalar el \emph{Node.js}.
\item
Descargar e instalar los paquetes de \emph{Node.js} requeridos por el proyecto (dependencias).
\item
Ejecutar los test unitarios y generar reporte de resultados.
\end{itemize}

\begin{figure}[!htbp]
\begin{center}  % [width=14cm,height=8cm] [width=\textwidth]
\includegraphics*[width=14cm]{Figures/TravisMochaChaiTests.png}
\par\caption{Resultado de ejecución de pruebas unitarias sobre el generador de sAPI en lenguaje \emph{JavaScript} con \emph{Travis CI}.}\label{fig:testTravisJS}
\end{center}
\end{figure}

%mochaChaiTests

En los repositorios de cada plataforma \emph{Travis CI} se encarga de realizar las siguientes tareas:

\begin{itemize}
\item
Crear un entorno de ejecución con el sistema operativo Linux.
\item
Clonar el repositorio.
\item
Descargar e instalar el \emph{toolchain} \emph{GNU ARM Embedded}, necesario compilar código en lenguaje C para plataformas con microcontroladores que incluyen núcleos ARM Cortex-M y R.
\item
Ejecutar el \emph{script} de \emph{Bash} \citep{Bash} denominado \emph{test-build-all.sh}. Este \emph{script} compila todos los programas en lenguaje C del repositorio (mediante \emph{Make} y el \emph{toolchain}), recogiendo la información asociada a la compilación exitosa, o no, de de cada programa, como se expone en la figura \ref{fig:testTravisC}. De esta forma, realiza un ensayo para comprobar que todos los programas siguen compilado ante cambios en el código fuente en el repositorio.
\end{itemize}

\begin{figure}[!htbp]
\begin{center}  % [width=14cm,height=8cm] [width=\textwidth]
\includegraphics*[width=14cm]{Figures/travisBuildC.png}
\par\caption{Resultado de compilación de programas en C con \emph{Travis CI}.}\label{fig:testTravisC}
\end{center}
\end{figure}

Los resultados de \emph{Travis CI} son plasmados dentro de una pequeña imagen conocida como ''medalla'' en los repositorios (figura \ref{fig:travisBadge}). Éstas indican de forma inmediata el estado correcto (\emph{build passing}), o no (\emph{build failing}) del ensayo llevado a cabo por \emph{Travis CI}. 

\begin{figure}[!htbp]
\begin{center}  % [width=14cm,height=8cm] [width=\textwidth]
\includegraphics*[width=14cm]{Figures/travisBadge.png}
\par\caption{Medalla de \emph{Travis CI} en el repositorio.}\label{fig:travisBadge}
\end{center}
\end{figure}

Todos los ensayos mencionados pueden ser ejecutados también de forma local antes de enviar los cambios al repositorio.

%------------------------------------------------------------------------------
\subsection{Ensayos funcionales sobre las plataformas de hardware}

Como \emph{Travis CI} es un servicio en la nube, no permite ejecutar programas de \emph{tests} corriendo sobre las plataformas de hardware. Para realizarlo, se debe montar un servidor propio con \emph{Travis CI} y conectarle las cuatro plataformas, contando con alta disponibilidad. Debido a los costos de mantenimiento de esta infraestructura, se realizaron ensayos sobre el hardware a demanda. Esto es, se conectan las cuatro placas en una computadora de desarrollo en forma local y se ejecuta un \emph{script} denominado \emph{tests-c-unit.sh}. Este \emph{script} realiza las siguientes tareas, repitiéndolas para cada plataforma de hardware:

\begin{itemize}
\item
\emph{pull} del repositorio de la plataforma para actualizar el repositorio local con los cambios del repositorio remoto.
\item
Compilar usando \emph{Make} y el \emph{toolchain} \emph{GNU ARM Embedded}, cada uno de los programas que implementan \emph{tests} unitarios. Estos programas contienen la biblioteca de aserciones sencilla para C denominada \emph{Minut}. Se compilan para ser ejecutados en la memoria RAM del microcontrolador para evitar el desgaste de su memoria Flash.
\item
Descargar con \emph{Make} y \emph{OpenOCD}, cada programa a la plataforma (en RAM), ejecutarlo y almacenar el estado devuelto por el mismo, exponiendo los resultados de las pruebas por terminal. Los programas con \emph{tests} utilizan \emph{semihosting}\footnote{\emph{Semihosting} es un mecanismo que permite que el código que se ejecuta en un microcontrolador ARM se comunique y use las facilidades de entrada/salida en una computadora \emph{host} que ejecuta un depurador.} \citep{semihosting}.
\end{itemize}

%------------------------------------------------------------------------------
\subsection{Flujo de desarrollo}

Se decide utilizar el siguiente flujo de trabajo:

\begin{itemize}
\item
Se debe abrir un \emph{issue} en el repositorio de \emph{GitHub} correspondiente por cada característica a diseñar/mejorar.
\item
Cada desarrollador debe mantener actualizado su propio \emph{Fork} (copia en su propia cuenta de \emph{GitHub}) del repositorio con cual desea colaborar.
\item
Para enviar una contribución al repositorio oficial debe hacerlo mediante un \emph{Pull Request} a la rama \emph{develop}.
\item
La contribución se integra (\emph{merge}) a la rama \emph{develop} solamente si pasan todos los \emph{tests} en \emph{Travis C}. Esta tarea la realiza un encargado del repositorio.
\item
La contribución se integra en la rama \emph{master} (rama principal) solamente después de pasar correctamente ensayos funcionales en hardware.
\end{itemize}






%------------------------------------------------------------------------------
%\titulo{Testeo Unitario en JavaScript}


%------------------------------------------------------------------------------
%\titulo{Testeo Unitario en JavaScript}


%------------------------------------------------------------------------------
%\titulo{Testeo Unitario en C}


%------------------------------------------------------------------------------
%\titulo{Integración continua}



%------------------------------------------------------------------------------
%\subsubsection{Banco de pruebas de hardware}
%\label{sec:testBench}





%------------------------------------------------------------------------------
%\subsection{Verificación de los archivos de descripción de una plataforma}

%Verificación de generación del modelo a partir de archivos json y viseversa.

%------------------------------------------------------------------------------
%\subsection{Verificación del generador de código C}

%------------------------------------------------------------------------------
%\subsection{Ensayos funcionales}

%Verificación del funcionamiento de cada uno de los periféricos en cada una de las plataformas.


%REQ.3.2. Programar en lenguaje C la biblioteca para cada plataforma de hardware particular utilizando como plantilla los archivos generados.

