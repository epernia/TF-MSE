%------------------------------------------------------------------------------
% Section
%------------------------------------------------------------------------------
\section{Verificaciónes de la implementación de la biblioteca}

Se exponen las verificaciones realizadas para el cumplimiento de los requerimientos del grupo \emph{REQ3} sobre la implementación de la biblioteca.

%------------------------------------------------------------------------------
\subsection{Metodología y herramientas utilizadas}

Para el cumplimiento del requerimiento \emph{REQ.3.1}, se utilizó el  siguiente esquema de trabajo:

figura \ref{fig:ciTools}.

\begin{figure}[!htbp]
\begin{center}  % [width=14cm,height=8cm] [width=\textwidth]
\includegraphics*[width=14cm]{Figures/ciTools.pdf}
\par\caption{Metodología y herramientas.}\label{fig:ciTools}
\end{center}
\end{figure}




  utilizaron las siguientes herramientas:


en cumplimiento con \emph{REQ.3.3}. Los mismos

repositorios \emph{Git} en el sitio \emph{Github}. 


Además se realizaron \emph{tests} unitarios sobre el generador de bibliotecas en lenguaje \emph{JavaScript} y del código generado en lenguaje \emph{C} dependiente de la arquitectura 


REQ.3.1. Utilizar un sistema de control de versiones on line, desarrollar tests unitarios y de integración.

Sistema de control de versiones: utilización de 

\emph{Workflow} de desarrollo mediante \emph{Fork} y \emph{Pull Requests} de github. Se debe abrir un \emph{issue} en \emph{Github} por cada característica a diseñar/mejorar.


%------------------------------------------------------------------------------
\titulo{Testeo Unitario en JavaScript}


%------------------------------------------------------------------------------
\titulo{Testeo Unitario en JavaScript}


%------------------------------------------------------------------------------
\titulo{Testeo Unitario en C}


%------------------------------------------------------------------------------
\titulo{Integración continua}



%------------------------------------------------------------------------------
%\subsubsection{Banco de pruebas de hardware}
%\label{sec:testBench}





%------------------------------------------------------------------------------
\subsection{Verificación de los archivos de descripción de una plataforma}

Verificación de generación del modelo a partir de archivos json y viseversa.

%------------------------------------------------------------------------------
\subsection{Verificación del generador de código C}

%------------------------------------------------------------------------------
\subsection{Ensayos funcionales}

Verificación del funcionamiento de cada uno de los periféricos en cada una de las plataformas.


REQ.3.2. Programar en lenguaje C la biblioteca para cada plataforma de hardware particular utilizando como plantilla los archivos generados.

