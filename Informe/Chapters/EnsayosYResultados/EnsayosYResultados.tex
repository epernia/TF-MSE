\emph{Pruebas funcionales del hardware: La idea de esta sección es explicar cómo se hicieron los ensayos, qué resultados se obtuvieron y analizarlos.}


%----------------------------------------------------------------------------------------
%	SECTION
%----------------------------------------------------------------------------------------
\section{Análisis del software}
 
La idea de esta sección es resaltar los problemas encontrados, los criterios utilizados y la justificación de las decisiones que se hayan tomado.

Se puede agregar código o pseudocódigo dentro de un entorno lstlisting con el siguiente código:

\begin{verbatim}
\begin{lstlisting}[caption= "un epígrafe descriptivo"]

	las líneas de código irían aquí...
	
\end{lstlisting}
\end{verbatim}

A modo de ejemplo:

\begin{lstlisting}[caption=Pseudocódigo del lazo principal de control.]  % Start your code-block

#define MAX_SENSOR_NUMBER 3
#define MAX_ALARM_NUMBER  6
#define MAX_ACTUATOR_NUMBER 6

uint32_t sensorValue[MAX_SENSOR_NUMBER];		
FunctionalState alarmControl[MAX_ALARM_NUMBER];	//ENABLE or DISABLE
state_t alarmState[MAX_ALARM_NUMBER];						//ON or OFF
state_t actuatorState[MAX_ACTUATOR_NUMBER];			//ON or OFF

void vControl() {

	initGlobalVariables();
	
	period = 500 ms;
		
	while(1) {

		ticks = xTaskGetTickCount();
		
		updateSensors();
		
		updateAlarms();
		
		controlActuators();
		
		vTaskDelayUntil(&ticks, period);
	}
}
\end{lstlisting}





%------------------------------------------------------------------------------
%	SECTION
%------------------------------------------------------------------------------
\section{Testeo Unitario}
\label{sec:unitTest}


%------------------------------------------------------------------------------
%	SECTION
%------------------------------------------------------------------------------
\section{Banco de pruebas de hardware}
\label{sec:testBench}


%------------------------------------------------------------------------------
%	SECTION
%------------------------------------------------------------------------------
\section{Integración continua}
\label{sec:ci}


%------------------------------------------------------------------------------
%	SECTION
%------------------------------------------------------------------------------
\section{Utilización de la biblioteca para la enseñanza de programación de Sistemas Embebidos}
\label{sec:teach}


%----------------------------------------------
%
%CASOS DE USO (Paper sAPI)
%
%Desde la primer versión de la biblioteca sAPI realizada en el marco del proyecto de Java[] en 2015 se ha utilizado como ejemplo de capa de abstracción de hardware en las asignaturas dictadas por el autor. Estas asignaturas son: el  curso de posgrado “Programación de microproce- sadores”[] de la Carrera Especialización en Sistemas Embebidos (CESE[]) de la FI-UBA. Donde el autor se ha desempeñado como docente a cargo del curso durante tres ediciones, y la asignatura “Sistemas Digitales” de la carrera Ingeniería en Automatización y Control Industrial (IACI[]) de la Universidad Nacional de Quilmes (UNQ) donde el autor se desempeña en la actualidad como instructor.
%
%A partir mazo de 2016 se decide utilizar la biblioteca como en el marco de los Cursos Abiertos de Programación de Sistemas Embebidos (CAPSE[]) organizados por la ACSE[]. De esta forma se ha extendido la biblioteca de forma considerable para explicar la utilización de todos los periféricos típicos de microcontroladores con excelentes resultados en cuanto a aprendizaje por parte de los alumnos tanto de niveles avanzados como quienes dan sus primeros pasos en el aprendizaje de programación de microcontroladores.
%Además, la biblioteca sAPI se puso a disposición de cualquier persona ya que se encuentra publicada de forma libre y gratuita por internet bajo una licencia BSD modificada[] en el sitio de github del autor[] y ha sido adoptada por una gran cantidad de usuarios. 
%
%Finalmente, en diciembre de 2016 se decide utilizar la biblioteca como biblioteca estándar para las plataformas del Proyecto CIAA. Esto llevó a una profunda revisión y mejora de la misma y todavía se está trabajando en la actualidad.
%
%
%----------------------------------------------
%
%
%CASOS DE USO (paper tools)
%
%Estas herramientas se han utilizado en múltiples, entre
%ellos, el curso "Sistemas digitales" de la carrera Ingeniería en
%Automatización y Control Industrial en la UNQ, donde el
%autor actualmente se desempeña como Profesor Instructor
%desde 2014. El plan de estudio de este curso incluye
%programación avanzada en lenguaje C para
%microcontroladores, con temas tales como modularización de
%código, máquinas de estado finito, sistemas operativos en
%tiempo real que usan programación cooperativa y apropiativa
%en microcontroladores; el curso de posgrado "Programación
%de microprocesadores" dentro de la CESE de FI-UBA donde
%el autor ha sido profesor a cargo de tres ediciones. Este curso
%incluye: programación básica de microcontroladores en
%lenguaje C, modularización, máquinas de estados finitos.
%También en cursos organizados por ACSE []. En estos
%cursos las herramientas junto con la secuencia didáctica
%propuesta fueron intensamente utilizadas, mejorando esta
%secuencia entre el autor y Pablo Gómez. En los cursos de
%ACSE se distinguen dos grupos, "Cursos Abiertos de
%Programación de Sistemas Embebidos" (CAPSE), orientados a
%cualquier persona interesada en aprender la programación de
%Sistemas Embebidos; y cursos impartidos al "Instituto
%Nacional de Escuelas Técnicas" (INET) [], enfocados en la
%capacitación de docentes de Escuelas Técnicas Secundarias,
%para posteriormente poder retransmitir a sus alumnos en todo
%el país. En ambos casos, se han observado muy buenos
%resultados de aprendizaje. En el caso de los cursos CAPSE, se
%impartieron cuatro cohortes entre 2016 y 2017; en la primera
%cohorte, el 46\% de los estudiantes completaron todos los
%niveles con un promedio de 67\% de índice de aprobación por
%nivel; mientras que en la cuarta cohorte, el 73\% completó
%todos los niveles, con una tasa de aprobación promedio de
%81\% por nivel. En el caso de los cursos INET, con dos
%cohortes en 2017, el 68\% de los estudiantes completaron todos
%los niveles, con una tasa promedio de aprobación del 74\% por
%nivel. Además, después de completar sus estudios, algunos de
%estos estudiantes se inscribieron posteriormente en la CESE
%FI-UBA para continuar profundizando su aprendizaje.
%Además, se han realizado varios talleres en escuelas
%secundarias y universidades de todo el país. Vale la pena
%mencionar los múltiples tutoriales y workshop en el "Simposio
%Argentino de Sistemas Embebidos" (SASE) [].
%Todas estas herramientas son de código abierto, publicadas de
%forma gratuita en la cuenta web de github del Proyecto CIAA
%y han sido adoptadas por un número importante de usuarios y
%docentes.

%------------------------------------------------------------------------------
%	SECTION
%------------------------------------------------------------------------------
\section{Documentación y difusión}
\label{sec:documentation}


%------------------------------------------------------------------------------
\subsection{Generación automática de manual de referencia de la API en base al modelo}



%------------------------------------------------------------------------------
\subsection{Tutoriales de instalación y uso}



%------------------------------------------------------------------------------
\subsection{Difusión a la comunidad del Proyecto CIAA y Embebidos32}


