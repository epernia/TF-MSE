%------------------------------------------------------------------------------
%	SECTION
%------------------------------------------------------------------------------
\section{Motivación}
\label{sec:motivacion}

Este trabajo es parte de un grupo de iniciativas realizadas por el autor (docente e investigador en la Universidad Nacional de Quilmes), para colaborar en el marco del Proyecto CIAA, con el objetivo de facilitar el desarrollo y la enseñanza de Sistemas Embebidos en Argentina.
%Otras iniciativas incluyen: entorno de programación PLC Ladder (IDE4PLC) [], programación Java para CIAA (CIAA-HVM) [], Firmata4CIAA [] (para facilitar la programación de bloques gráficos en BYOB Snap! lenguaje para uso en escuelas secundarias). 
% @Eric ver si pongo las cosas que hice, en ese caso agregar la bib

El proyecto CIAA nació en 2013 como una iniciativa conjunta entre el sector académico y el industrial, representados por la ACSE \citep{ACSE} y CADIEEL \citep{CADIEEL} respectivamente. Consiste en una comunidad argentina de desarrolladores de herramientas de software y hardware abierto y usuarios que desde sus inicios intenta impulsar el desarrollo tecnológico nacional proveyendo herramientas abiertas de software y hardware, para mejorar la situación industrial, darle visibilidad a la electrónica y generar cambios estructurales en la forma en que se generan, comparten y utilizan los conocimientos en Argentina.

% @Eric: mover a bib
%\footnote{Asociación Civil para la investigación, promoción y desarrollo de los Sistemas electrónicos Embebidos. Sitio web: \url{http://www.sase.com.ar/asociacion-civil-sistemas-embebidos}}
%\footnote{Cámara Argentina de Industrias Electrónicas, Electromecánicas y Luminotécnicas. Sitio web: \url{http://www.cadieel.org.ar/}}

Entre los principales logros del proyecto se destacan:

\begin{itemize}
\item
Se creó una comunidad de desarrolladores de software y hardware abierto con participación de profesionales, docentes Universitarios, alumnos y docentes de escuelas Secundarias a nivel nacional.
\item
Se desarrollaron múltiples diseños de referencia de plataformas de hardware para sistemas embebidos. Estas plataformas fueron diseñadas para diferentes casos de uso, entre ellos: educativo, industrial, sistema críticos y alta capacidad de cómputo.
\item
Se logró la colaboración de empresas nacionales para la fabricación y comercialización de la mayoría de estas plataformas.
\item
Se crearon bibliotecas, \emph{frameworks} y entornos de programación para las plataformas, contando con múltiples lenguajes de programación.
\item
En colaboración con universidades, se insertó la plataforma educativa EDU-CIAA-NXP en las universidades con carreras de electrónica o afines, a lo largo y a lo ancho del país. De esta forma se logró modernizar las herramientas que utilizan los alumnos en su aprendizaje, alcanzando el estado del arte.
\item
Se han dictado múltiples cursos para docentes y alumnos de escuelas secundarias, docentes y alumnos universitarios y público interesado.
\end{itemize}

El autor de este trabajo final cumple actualmente el rol de coordinador general del proyecto CIAA. Tanto en el trabajo realizado para la implementación de bibliotecas para diferentes plataformas en el marco de este proyecto, como en el trabajo profesional, ha notado problemática de falta de estadarización en las bibliotecas de sistemas embebidos detallada en la sección \ref{sec:contextoYJustificacion}.

Se considera de importancia estratégica para el proyecto CIAA que todas las plataformas de hardware diseñadas en el marco del mismo se programen utilizando la misma biblioteca para permitir a la comunidad de usuarios reducir tiempos de aprendizaje y desarrollo.

Como el lenguaje más utilizado en la actualidad para programación de sistemas embebidos basados en microcontroladores continúa siendo el lenguaje C, se decide realizar la implementación de la biblioteca en este lenguaje.
