%------------------------------------------------------------------------------
%	SECTION
%------------------------------------------------------------------------------
\section{Objetivos y alcance}
\label{sec:objetivosAlance}

En esta sección se definen los objetivos y el alcance del presente trabajo final.
%(sección \ref{subsec:objetivos}) 
%(sección \ref{subsec:alcance})

%------------------------------------------------------------------------------
\subsection{Objetivos}
\label{subsec:objetivos}

El objetivo de este proyecto es diseñar e implementar una biblioteca de software para la programación de sistemas embebidos basados en microcontroladores con las siguientes características: 

\begin{itemize}
\item
Estar modelada independientemente de los lenguajes de programación.
\item
Definir una interfaz de programación de aplicaciones (API) sencilla que abstraiga los modos de uso más comunes de los periféricos típicos hallados en cualquier microcontrolador del mercado. 
\item
Ser totalmente portable entre diferentes arquitecturas de hardware sobre dónde se ejecuta, manteniendo una API uniforme a lo largo de las mismas, cumpliendo la función de capa de abstracción de hardware.
\item
Debe ser útil tanto para enseñanza de programación de sistemas embebidos como para uso a nivel industrial.
\end{itemize}

La implementación de referencia se realizará en lenguaje C para las plataformas del Proyecto CIAA.

%------------------------------------------------------------------------------
\subsection{Alcance}
\label{subsec:alcance}

Este Trabajo Final incluye realización de:

\begin{itemize}
\item
Diseño de la biblioteca, archivos de descripción de la biblioteca mediante diferentes diagramas y código independiente del lenguaje de programación.
\item
Implementación en lenguaje C de la biblioteca para las plataformas de hardware:
\begin{itemize}
\item CIAA-NXP.
\item EDU-CIAA-NXP.
\item CIAA-Z3R0.
\item PicoCIAA.
\end{itemize}
\item
Manual de instalación de las herramientas para utilizar la biblioteca con las plataformas de hardware citadas.
\item
Manual de referencia de la biblioteca.
\item
Ejemplos de utilización.
\end{itemize}

%------------------------------------------------------------------------------
%	SECTION
%------------------------------------------------------------------------------
\section{Metodología de trabajo}

Para el desarrollo de este trabajo se elige utilizar las siguientes prácticas y herramientas:

\begin{itemize}
\item
Sistema de control de versiones: utilización de repositorios \emph{Git} \citep{GIT} en el sitio \emph{Github} \citep{GITHUB}. 
\item
\emph{Workflow} de desarrollo mediante \emph{Fork} y \emph{Pull Requests} de github. Se debe abrir un \emph{issue} en \emph{Github} por cada característica a diseñar/mejorar.
\item
Producción de código y su documentación: escribir la documentación a la par del código fuente.
\item
Definir un documento de estilo del código fuente.
\item
Documentación general en lenguaje \emph{Markdown} \citep{MARKDOWN}, que permite exportar el contenido a \emph{html}, \LaTeX, \emph{pdf}, entre otros.
\item
Desarrollo de \emph{tests} unitarios y de integración.
\end{itemize}
