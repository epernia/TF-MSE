%------------------------------------------------------------------------------
%	SECTION
%------------------------------------------------------------------------------
\section{Objetivos y alcance}
\label{sec:objetivosAlance}

En esta sección se definen los objetivos (sección \ref{subsec:objetivos}) y el alcance (sección \ref{subsec:alcance}) del presente Trabajo Final.

%------------------------------------------------------------------------------
\subsection{Objetivos}
\label{subsec:objetivos}

El objetivo de este proyecto es diseñar e implementar una biblioteca de
software para la programación de sistemas embebidos basados en
microcontroladores con las siguientes características: 

\begin{itemize}
   \item Estar modelada independientemente de los lenguajes de programación.
   \item Definir una interfaz de programación de aplicaciones (API) sencilla que abstraiga los modos de uso más comunes de los periféricos típicos que hallados en cualquier microcontrolador del mercado. 
   \item Ser totalmente portable entre diferentes arquitecturas de hardware sobre donde se ejecuta, manteniendo una API uniforme a lo largo de las mismas\footnote{Debe cumplir la función de capa de abstracción de hardware, o \textit{Hardware Abstraction Layer} (HAL),  en inglés}.
\end{itemize}

Dicha biblioteca se deberá implementar en lenguaje C para las plataformas del
Proyecto CIAA.

%------------------------------------------------------------------------------
\subsection{Alcance}
\label{subsec:alcance}

Este Trabajo Final incluye realización de:

\begin{itemize}
   \item Diseño de la biblioteca. Archivos de descripción de la biblioteca mediante diferentes diagramas y código independiente del lenguaje de programación.
   \item Implementación en lenguaje C de la biblioteca para las plataformas de hardware:
   \begin{itemize}
      \item CIAA-NXP.
      \item EDU-CIAA-NXP.
      \item CIAA-Z3R0.
      \item PicoCIAA.
   \end{itemize}
   \item Manual de instalación de las herramientas para utilizar la biblioteca con las plataformas de hardware citadas.
   \item Manual de referencia de la biblioteca.
   \item Ejemplos de utilización.
\end{itemize}
