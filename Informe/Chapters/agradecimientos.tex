En principio agradezco a mi novia, Soledad Kruszyn, y a mis padres, Claudia Petruzzi y Jorge Pernia, cuyo apoyo y esfuerzo hizo posible el trabajo realizado.

A Ariel Lutemberg, por otorgar la beca que me dió la oportunidad de cursar la especialización y maestría, tanto por la confianza depositada, como por permitirme entrar en contacto con mucha gente que comparte el entusiasmo en la temática de sistemas embebidos.

A Félix Safar, director de este trabajo y del programa de investigación en el que participo en la Universidad Nacional de Quilmes, y al departamento de ciencia y tecnología, dirigido por Alejandra Zinni por el apoyo brindado.

A Martín Ribelotta, cuya experiencia ha sido un aporte fundamental durante la realización de la biblioteca.

A Pablo Rifolfi, por promover el uso de la sAPI como biblioteca oficial de las plataformas del proyecto CIAA durante su gestión como coordinador general del proyecto.

A José Juarez, Pablo Gómez, Juan Manuel Cruz y otra entome cantidad de docentes que confiaron en el criterio del autor, colaborando activamente en el uso de las bibliotecas en sus clases.

A todos mis compañeros de la maestría, con quienes he compartido las asignaturas, en especial Alejandro Celery y Leandro Lanzieri, con quienes hemos formado un excelente equipo de estudio.

Finalmente, a la comunidad de usuarios del proyecto CIAA, los alumnos de los cursos CAPSE, CESE e IACI, que han adoptado estas herramientas con grán entusiasmo, motivando al autor a continuar con la mejora continua de las mismas.
