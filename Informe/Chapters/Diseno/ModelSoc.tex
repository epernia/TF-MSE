%------------------------------------------------------------------------------
%	SECTION
%------------------------------------------------------------------------------
\section{Modelo de SoC}
\label{sec:modelSoC}

Un SoC describe un sistema completo dentro de un chip. En la jerga electrónica se utiliza para describir tanto microcontroladores (que incluyen núcleos de procesamiento, memorias y diversos periféricos), como sistemas que incluyen además lógica programable (FPGA) o módulos analógicos de radiofrecuencia complejos como ser Bluetooth y Wi-Fi. 

En la figura \ref{fig:ModelSoC} se muestra el modelo de SoC, el cual se compone de núcleos de procesamiento, periféricos y memorias.

\begin{figure}[!htbp]
\begin{center}  % [width=14cm,height=8cm] [width=\textwidth]
\includegraphics*[width=14cm]{Figures/SoC.pdf}
\par\caption{Diagrama de clases de SoC.}\label{fig:ModelSoC}
\end{center}
\end{figure}
