%------------------------------------------------------------------------------
\subsection{Módulo GPIO}

El módulo GPIO modela tanto un único pin de entrada/salida de propósito general (pin), así como a un conjunto de pines (puerto). 

\titulo{Propiedades de pin}

\begin{itemize}
\item
\emph{mode}, del tipo \emph{PinMode\_t}, cuyos valores posibles son: \emph{DISABLE} (deshabilidato), \emph{INPUT} (pin configurado como entrada) y \emph{OUTPUT} u \emph{OUTPUT\_PUSHPULL} (pin configurado como salida, modo \emph{push-pull}), \emph{OUTPUT\_OPEN\_DRAIN} (pin como salida, en modo drenador abierto).
\item
\emph{pull}, del tipo \emph{PinPull\_t} cuyos valores posibles son: \emph{DISABLE} (modo por defecto, pin flotante), \emph{PULL\_UP} (con resistencia de \emph{pull-up}), \emph{PULL\_DOWN} (con resistencia de \emph{pull-down}), \emph{PULL\_BOTH} (con ambas resistencias).
\item
\emph{value}, del tipo \emph{bool\_t}. Representa el valor a escribir o leer de un pin.
\end{itemize}

\titulo{Métodos de pin}

Los métodos básicos de configuración y uso de un pin son:

\begin{verbatim}
 • init( PinMode_t mode | PinPull_t pull ) : PinStatus_t
 • read() : bool_t
 • write( bool_t value )
 • deinit() : PinStatus_t
\end{verbatim}

Notar que los parámetros de configuración del método \emph{init()} se acumulan mediante el operador $|$. 

Contiene además el método \emph{toggle()} intercambia el valor del pin si el mismo está configurado como salida.:

\begin{verbatim}
 • toggle()
\end{verbatim}

Para establecer una interrupción en un pin ante cierto evento define los métodos: 

\begin{verbatim}
 • hasInterrupt() : bool_t
 • interrupt( bool_t enable )
 • eventCallbackSet( PinEvent_t evt, Callback_t c )
 • eventCallbackClear( PinEvent_t evt )
\end{verbatim}

Los posibles eventos son: \emph{HIGH\_LEVEL} (interrupción por nivel alto), \emph{LOW\_LEVEL} (interrupción por nivel bajo), \emph{RISING\_EDGE} (interrupción por flanco ascendente), \emph{FALLING\_EDGE} (interrupción por flanco descendente) y \emph{BOTH\_EDGE} (interrupción por flanco ascendente y descendente). Nótese que según la plataforma particular existen las posibilidades de que algunas pines fijos solamente soporten interrupción, o que exista una cantidad de interrupciones de pin y se pueda elegir a qué pines se le asignan, es por eso que se define el método \emph{hasInterrupt()} que retorna true si un dado pin tiene capacidad de configurar interrupción. 

\titulo{Propiedad de puerto}

\emph{value}, del tipo \emph{uitn32\_t}. Representa el valor a escribir o leer en un puerto.

\titulo{Métodos de puerto}

Para la configuración y uso de un puerto se definen los métodos básicos:

\begin{verbatim}
 • init( List_t config ) : PinStatus_t
 • read() : uint32_t
 • write( uint32_t value )
 • deinit() : PinStatus_t
\end{verbatim}

El método \emph{init()} de puerto requiere una lista de configuraciones.
Los métodos \emph{read()} y \emph{write()} permiten leer y escribir el puerto completo.

Se agregan además, el método \emph{pins()}, devuelve una lista de pines que componen el puerto, y los métodos \emph{set()} y \emph{reset()} que permiten escribir el puerto afectado por una máscara de bits que se le pasa como parámetro:

\begin{verbatim}
 • pins(): List_t
 • set( uint32_t mask )
 • reset( uint32_t mask )
\end{verbatim}
