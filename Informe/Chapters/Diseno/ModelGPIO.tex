%------------------------------------------------------------------------------
\subsection{Módulo GPIO}

El módulo GPIO modela tanto un único pin de entrada/salida de propósito general (pin), así como a un conjunto de pines (puerto). 

\titulo{Tipos de datos}

\begin{itemize}
\item
\emph{Pin\_t}, modela un pin.
\item
\emph{Port\_t}, representa un puerto.
\item
\emph{PortValue\_t}, representa el valor de todos los pines del puerto.
\end{itemize}

La cantidad de pines que forman un puerto depende de la arquitectura.

\titulo{Propiedades de pin}

\begin{itemize}
\item
\emph{mode}, del tipo \emph{PinMode\_t}, cuyos valores posibles son: \emph{DISABLE} (deshabilidato), \emph{INPUT} (pin configurado como entrada) y \emph{OUTPUT} u \emph{OUTPUT\_PUSHPULL} (pin configurado como salida, modo \emph{push-pull}), \emph{OUTPUT\_OPEN\_DRAIN} (pin como salida, en modo drenador abierto).
\item
\emph{pull}, del tipo \emph{PinPull\_t} cuyos valores posibles son: \emph{NONE} (modo por defecto, pin flotante), \emph{PULL\_UP} (con resistencia de \emph{pull-up}), \emph{PULL\_DOWN} (con resistencia de \emph{pull-down}), \emph{PULL\_BOTH} (con ambas resistencias).
\item
\emph{value}, del tipo \emph{bool\_t}. Representa el valor a escribir o leer de un pin.
\end{itemize}

\titulo{Métodos de pin}

Los métodos básicos de configuración y uso de un pin son:

\begin{verbatim}
   • init( PintInit_t init )
   • read() : bool_t
   • write( bool_t value )
   • deInit()
\end{verbatim}

%Speed. %Specifies the speed for the selected pins. %Possible values are:
%GPIO_SPEED_LOW
%GPIO_SPEED_MEDIUM
%GPIO_SPEED_FAST
%GPIO_SPEED_HIGH

El parámetro \emph{init} se debe completar con un valor posible de \emph{mode}, y opcionalmente, un valor de \emph{pull} separados con $|$, por ejemplo:

\begin{verbatim}
   • init( INPUT )
   • init( INPUT | PULL_UP )
\end{verbatim}

Contiene además el método \emph{toggle()} intercambia el valor del pin si el mismo está configurado como salida.:

\begin{verbatim}
   • toggle()
\end{verbatim}

Para establecer una interrupción en un pin ante cierto evento define los métodos: 

\begin{verbatim}
   • eventCallbackSet( PinEvent_t evt, Callback_t func )
   • eventCallbackClear( PinEvent_t evt )
   • interrupt( bool_t enable ) : PinStatus_t
 \end{verbatim} 

\emph{eventCallbackSet()} establece el evento ante el cual se interrumpe, junto a una función de \emph{callback} que se ejecutará ante su ocurrencia. \emph{eventCallbackClear()} permite remover el \emph{callback}. Los posibles eventos son: \emph{HIGH\_LEVEL} (interrupción por nivel alto), \emph{LOW\_LEVEL} (interrupción por nivel bajo), \emph{RISING\_EDGE} (interrupción por flanco ascendente), \emph{FALLING\_EDGE} (interrupción por flanco descendente) y \emph{BOTH\_EDGES} (interrupción por flanco ascendente y descendente); todos del tipo \emph{PinEvent\_t}.

\emph{interrupt()} permite habilitar o deshabilitar la interrupción de hardware del pin ante la ocurrencia del evento. Según la plataforma particular, existe la posibilidad que soporte interrupción en algunos pines fijos, o que exista una cierta cantidad de interrupciones de pin y se pueda elegir a qué pines se le asignan, es por eso que el método \emph{interrupt()} retorna un valor del tipo \emph{PinStatus\_t}. Este tipo de datos define los valores \emph{STATUS\_OK}, si pudo configurar la interrupción, \emph{ERROR\_NOT\_SUPPORT}, si el pin no soporta capacidad de interrupción, y \emph{ERROR\_NO\_MORE\_AVAILABLE}, si ya se configuraron todas las interrupciones de hardware disponibles. 

\titulo{Propiedades y métodos de puerto}

Puerto define la propiedad \emph{value}, del tipo \emph{PortValue\_t}. Representa el valor a escribir o leer en un puerto.

Para la configuración y uso de un puerto se definen los métodos:

\begin{verbatim}
   • init( List_t config )
   • read() : PortValue_t
   • write( PortValue_t value )
   • deInit()
\end{verbatim}

El método \emph{init()} de puerto requiere una lista de configuraciones. Los métodos \emph{read()} y \emph{write()} permiten leer y escribir el puerto completo.

Se agregan además los métodos:

\begin{verbatim}
   • pins(): List_t
   • set( PortValue_t mask )
   • reset( PortValue_t mask )
\end{verbatim}

El método \emph{pins()}, devuelve una lista de pines que componen el puerto. Los métodos \emph{set()} y \emph{reset()} permiten escribir el puerto afectado por una máscara de bits que se le pasa como parámetro.

% Para implementacion ver:

% https://github.com/epernia/sAPI/blob/develop/sapi1Project/libs/sapi_soc_peripherals/inc/sapi_gpio.h

% https://github.com/epernia/sAPI/blob/develop/sapi1Project/libs/sapi_soc_peripherals/src/sapi_gpio.c

% https://github.com/epernia/sAPI/blob/develop/toMergeIn_sapi1Project/sapi_soc_peripherals/PERIPH/GPIO/gpioInterruptLPCOpenExample.c.txt

% Ejemplos de micropython de uart interrupt de gigliotti

