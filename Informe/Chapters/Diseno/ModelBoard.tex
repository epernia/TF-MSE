%------------------------------------------------------------------------------
%	SECTION
%------------------------------------------------------------------------------
\section{Modelo de plataforma de hardware}
\label{sec:modelHardware}

En la figura \ref{fig:ModelBoard} se muestra el diagrama que describe una \emph{board} y las partes que la componen.

\begin{figure}[!htbp]
\begin{center}  % [width=14cm,height=8cm] [width=\textwidth]
\includegraphics*[width=14cm]{Figures/Board.pdf}
\par\caption{Diagrama de clases de \emph{board}.}\label{fig:ModelBoard}
\end{center}
\end{figure}

Estas partes son:

\begin{itemize}
\item
Componente: representa los componentes dentro de cierta placa, por ejemplo, circuitos integrados, botones, leds, conectores, etc. 
\item
Terminal: describe un terminal de conexión que posee cada componente.
\item
Conexión: modela el mapa de conexiones entre componentes. Cada conexión se compone de los dos terminales conectados.
\end{itemize}

%------------------------------------------------------------------------------
\subsection{Componente}

Un componente incluye nombre, documentación, una lista de terminales de conexión y la cantidad de terminales. En la figura \ref{fig:Component} se expone el diagrama de clases. Existen los siguientes tipos de componentes:

\begin{itemize}
\item
\emph{SystemOnChip}: modela un sistema completo en un chip.
\item
Componente con driver: modela a componentes que necesitan un driver, por ejemplo, LEDs, botones, sensores y memorias montados en la placa.
\item
Conector: representa los conectores físicos de la placa, como ser, tira de pines, borneras, etc. Su importancia en el modelo radica en la descripción de a qué terminales del SoC se conectan cada uno.
\end{itemize}

\begin{figure}[!htbp]
\begin{center}  % [width=14cm,height=8cm] [width=\textwidth]
\includegraphics*[width=11cm]{Figures/Component.pdf}
\par\caption{Diagrama de clases de componente.}\label{fig:Component}
\end{center}
\end{figure}

%------------------------------------------------------------------------------
\subsection{Modelo de SoC}
\label{sec:modelSoC}

Un SoC describe un sistema completo dentro de un chip. Este término se utiliza para describir tanto microcontroladores (que incluyen núcleos de procesamiento, memorias y diversos periféricos), como sistemas que incluyen además lógica programable (FPGA) o módulos analógicos de radiofrecuencia complejos como ser Bluetooth y Wi-Fi.

En la figura \ref{fig:ModelSoC} se muestra el modelo de SoC, el cual se compone de núcleos de procesamiento (\emph{Core}), periféricos (\emph{Peripheral}) y memorias (\emph{Memory}).

\begin{figure}[!htbp]
\begin{center}  % [width=14cm,height=8cm] [width=\textwidth]
\includegraphics*[width=8cm]{Figures/SoC.pdf}
\par\caption{Diagrama de clases de SoC.}\label{fig:ModelSoC}
\end{center}
\end{figure}

%------------------------------------------------------------------------------
\subsection{Modelo de IP core}
\label{sec:modelSoC}

Para modelar núcleos de procesamiento, periféricos y memorias se utiliza el concepto de \emph{IP core}. Un núcleo de propiedad intelectual es un bloque de lógica o datos reutilizable, para definir un diseño de hardware de forma abstracta. Normalmente se utilizan en electrónica como componentes en un diseño de hardware ya sea en una FPGA o ASIC. Define una interfaz de señales y comportamiento interno.

En la figura \ref{fig:IPCore} se muestra el modelo de \emph{IPCore} y sus relaciones.

\begin{figure}[!htbp]
\begin{center}  % [width=14cm,height=8cm] [width=\textwidth]
\includegraphics*[width=14cm]{Figures/IPCore.pdf}
\par\caption{Diagrama de clases de \emph{IPcore}.}\label{fig:IPCore}
\end{center}
\end{figure}

Un \emph{IPCore} define una dirección base, que se utiliza para localizarlo en el mapa de memoria. También define una interfaz que se compone de una lista ordenada de señales.

Las clases \emph{Memory} e \emph{IPCoreExtended} heredan de la clase \emph{IPCore} agregándole funcionalidades particulares. 

\emph{Memory} modela una memoria interna del SoC, por ejemplo, RAM o Flash. Tiene las propiedades: tipo de memoria, tamaño en bytes y tipo de acceso permitido (lectura, lectura/escritura).

\emph{IPCoreExtended} contiene registros e interrupciones. Los registros se modelan tanto para fines de documentación, como para poder realizar \emph{tests} sobre valores de los mismos. Las interrupciones son necesarias para conocer cuales tenemos disponibles físicamente y a qué evento responden. De la clase \emph{IPCoreExtended} heredan \emph{Core} y \emph{Peripheral}. La primera modela un núcleo de procesamiento y la segunda un periférico. 

\emph{Peripheral} contiene una lista ordenada de \emph{Locations} (ubicaciones). Una ubicación define un conjunto de conexiones entre señales del periférico y terminales del SoC. Esto permite modelar las posibles configuraciones de pines de un dado periférico.

%------------------------------------------------------------------------------
\subsection{Archivos para la descripción de una plataforma de hardware}

Para definir una plataforma de hardware se utilizan archivos de texto en formato \emph{JSON} [], de esta manera se puede definir cada una de las entidades presentadas por separado y referenciarlas. 

Como ejemplo, se expone a continuación los archivos propuestos para una pequeña plataforma de hardware ficticia nombrada ''fake-board-v1'' (se ilustra en la figura \ref{fig:fakeBoard}). 

\begin{figure}[!htbp]
\begin{center}  % [width=14cm,height=8cm] [width=\textwidth]
\includegraphics*[width=8cm]{Figures/FakeBoard-01.pdf}
\par\caption{Fake-board-v1.}\label{fig:fakeBoard}
\end{center}
\end{figure}

Archivo de descripción de \emph{board} nombrado \emph{fakeBoard.json}:

\begin{verbatim}
{
  "type": "Board",
  "name": "fake-board-v1",
  "components": [
    {
      "type": "@/vendor1/soc/fakeSoc.json",
      "name": "SOC0"
    },
    {
      "type": "@/vendor2/phy/fakePhy232.json",
      "name": "PHY0"
    },
    {
      "name": "RS232",
      "type": "@/vendor3/connector/fakeDb9Male.json"
    }
  ],
  "connections": [
    [ "#/SOC0/P1", "#/PHY0/RX_LVL" ],
    [ "#/SOC0/P2", "#/PHY0/TX_LVL" ],
    [ "#/SOC0/P3", "#/PHY0/GND" ],
    [ "#/PHY0/RX_232", "#/RS232/P2" ],
    [ "#/PHY0/TX_232", "#/RS232/P3" ],
    [ "#/PHY0/GND", "#/RS232/P5" ]
  ]
}
\end{verbatim}

''@'' indica una referencia a un archivo externo con el path incluido, mientras que ''\#'' indica una referencia a un terminal de un componente.

Archivo de descripción de \emph{SoC} nombrado \emph{fakeSoc.json}:

\begin{verbatim}
{
  "type": "SystemOnChip",
  "name": "fake-board-v1",
  "cores": [
    {
      "type": "@/arch/arm/armv7m.json",
      "name": "CORTEXM4"
    }
  ],
  "peripherals": [
    {
      "type": "@/vendor1/ipcore/uart.json",
      "name": "UART0"
    }
  ],
  "memories": [
    {
      "type": "@/vendor1/mem/sram8kb.json",
      "name": "RAM"
    },
    {
      "type": "@/vendor1/mem/flash32kb.json",
      "name": "FLASH"
    }
  ],
  "terminals": [ "P1", "P2", "P3" ]
}
\end{verbatim}

Archivo de descripción de un chip de capa física para RS-232  nombrado \emph{fakePhy232.json}:

\begin{verbatim}
{
  "type": "ComponentWithDriver",
  "name": "phy-rs232",
  "terminals": [ 
    "RX_LVL", "TX_LVL", "GND",
    "RX_232", "TX_232"
  ],
  "drivers": [
    {
      "lang":"c",
      "files": [
        "@/driver/c/vendor2/phy/fakePhy232.h",
        "@/driver/c/vendor2/phy/fakePhy232.c"      
	  ]      
    }
  ]
}
\end{verbatim}

Archivo de descripción de un conector \emph{DB-9} nombrado \emph{fakeDb9Male.json}:

\begin{verbatim}
{
  "type": "Connector",
  "name": "DB9-male",
  "terminals": [ 
    "P1", "P2", "P3", "P4", "P5", "P6", "P7", "P8", "P9"
  ]
}
\end{verbatim}

Cabe destacar que el ejemplo presentado es válido solamente para describir la estructura de los archivos y filosofía de diseño adoptada, sin embargo, los archivos de implementación poseen más propiedades y niveles de anidamiento en sus entidades.
