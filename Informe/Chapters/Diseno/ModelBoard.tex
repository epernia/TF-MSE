%------------------------------------------------------------------------------
%	SECTION
%------------------------------------------------------------------------------
\section{Modelo de plataforma de hardware}
\label{sec:modelHardware}

En la figura \ref{fig:ModelBoard} se muestra el diagrama que describe una \emph{board} y las partes que la componen.

\begin{figure}[!htbp]
\begin{center}  % [width=14cm,height=8cm] [width=\textwidth]
\includegraphics*[width=14cm]{Figures/Board.pdf}
\par\caption{Diagrama de clases de \emph{board}.}\label{fig:ModelBoard}
\end{center}
\end{figure}

Estas partes son:

\begin{itemize}
\item
Componente: representa los componentes dentro de cierta placa, por ejemplo, circuitos integrados, botones, leds, conectores, etc. 
\item
Terminal: describe un terminal de conexión que posee cada componente.
\item
Conexión: modela el mapa de conexiones entre componentes. Cada conexión se compone de los dos terminales conectados.
\end{itemize}

%------------------------------------------------------------------------------
\subsection{Componente}

Un componente incluye nombre, documentación, una lista de terminales de conexión y la cantidad de terminales. En la figura \ref{fig:Component} se expone el diagrama de clases. Existen los siguientes tipos de componentes:

\begin{itemize}
\item
\emph{SystemOnChip}: modela un sistema completo en un chip.
\item
Componente con driver: modela a componentes que necesitan un driver, por ejemplo, LEDs, botones, sensores y memorias montados en la placa.
\item
Conector: representa los conectores físicos de la placa, como ser, tira de pines, borneras, etc. Su importancia en el modelo radica en la descripción de a qué terminales del SoC se conectan cada uno.
\end{itemize}

\begin{figure}[!htbp]
\begin{center}  % [width=14cm,height=8cm] [width=\textwidth]
\includegraphics*[width=14cm]{Figures/Component.pdf}
\par\caption{Diagrama de clases de componente.}\label{fig:Component}
\end{center}
\end{figure}
