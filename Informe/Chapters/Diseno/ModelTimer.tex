%------------------------------------------------------------------------------
\subsection{Módulo Timer}

Este módulo modela un periférico contador/temporizador. Este periférico se utiliza para medición de tiempo, generación de retardos, conteo de eventos, generación de señales, etc.

Es el periférico que más variedad de modos tiene según la arquitectura. Debido a esto la documentación de implementación de la biblioteca para cada plataforma debe indicar qué modos soporta de forma clara.





Duración: TIME.

323 Mazidi

TIMER: modela un periférico Timer/Counter.

Time/Counterl Capture Event OOOC
Time/Counterl Compare Match A OOOE
Time/Counterl Compare Match B 001 0
Time/Counterl Overflow 001 2


\emph{Tick\_t}, representa un valor de conteo de un lapso de tiempo adimencional lapso de tiempo en ticks (se almacena internamente como \emph{uint64\_t}).

Como evento posible se define: \emph{CONVERSION\_COMPLETE} que genera una interrupción cuando se completa la conversión. El método \emph{interrupt()} habilita  o deshabilita todas las interrupciones del periférico.

% Para implementacion ver:

% https://github.com/epernia/sAPI/blob/develop/toMergeIn_sapi1Project/sapi_soc_peripherals/inc/sapi_timer.h

% la nueva sapi en la carpeta dev


%De notas.txt:
%
%
%Para TIMER en general:
%----------------------
%
%    Timer(4, prescaler=624, period=13439, mode=UP, div=1)
%
%The information means that this timer is set to run at the peripheral
%clock speed divided by 624+1, 
%
%it will count from 0 up to 13439, at which point it triggers an interrupt, 
%and then starts counting again from 0.
%
%These numbers are set to make the timer trigger at 10 Hz: the source frequency
%of the timer is 84MHz (found by running ``tim.source_freq()``) so we
%get 84MHz / 625 / 13440 = 10Hz.
%
%Timer counter
%-------------
%
%So what can we do with our timer?  The most basic thing is to get the
%current value of its counter::
%
%    >>> tim.counter()
%    21504
%
%This counter will continuously change, and counts up.
%
%---------------------------------------------------------
%
%
%Para TIMER PWM:
%---------------
%
%	enum POLARITY{ ACTIVE_LOW=0, ACTIVE_HIGH=1 };
%
%	float analogFrequency;  //defaults to 100,000 Hz
%	float analogMax;        //defaults to 3.3V
%
%	int setPeriod(unsigned int period_ns);
%	 unsigned int getPeriod();
%	 int setFrequency(float frequency_hz);
%	 float getFrequency();
%	 int setDutyCycle(unsigned int duration_ns);
%	 int setDutyCycle(float percentage);
%	 unsigned int getDutyCycle();
%	 float getDutyCyclePercent();
%
%	 int setPolarity(PWM::POLARITY);
%	 void invertPolarity();
%	 PWM::POLARITY getPolarity();
%
%	virtual void setAnalogFrequency(float frequency_hz) { this->analogFrequency = frequency_hz; }
%	virtual int calibrateAnalogMax(float analogMax); //must be between 3.2 and 3.4
%	virtual int analogWrite(float voltage);
%
%	virtual int run();
%	virtual bool isRunning();
%	virtual int stop();
%
%	virtual ~PWM();
%private:
%	float period_nsToFrequency(unsigned int);
%	unsigned int frequencyToPeriod_ns(float);
%};
%
%---------------------------------------------------------
%
%Para TIMER Waveform generation output signal:
%----------------------------------------------
%
%   // Generate a periodic sine wave in an array of 100 values - using ints
%   unsigned int waveform[100];
%   float gain = 50.0f;
%   float phase = 0.0f;
%   float bias = 50.0f;
%   float freq = 2.0f * 3.14159f / 100.0f;
%   for (i=0; i<100; i++){
%      waveform[i] = (unsigned int)(bias + (gain * sin((i * freq) + phase)));
%   }
%   
%   
%   
%tickConfig( TICK_MS(1) ); o tickStart( TICK_MS(1) );
%tickEnabeling( ENABLE );
%tickSetCallback( function );
%
%typedef tick_t uint64_t;
%
%#define TICK_MS(t) ((typedef)(t))
%#define TICK_S(t)  (((typedef)(t))*1000)



