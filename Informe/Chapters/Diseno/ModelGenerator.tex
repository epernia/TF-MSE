%------------------------------------------------------------------------------
%	SECTION
%------------------------------------------------------------------------------
\section{Generador de sAPI}
\label{sec:modelGenerator}

El generador de sAPI toma como entrada el una instancia de \emph{board} y \emph{drivers} específicos, junto con los \emph{módulos sAPI}, y genera como resultado una estructura de archivos y carpetas con la biblioteca sAPI en lenguaje C para una plataforma particular. También permite generación de documentación a partir del mismo modelo.

Para su diseño se estudió una forma sistematizada de generar un módulo de código en lenguaje C. Un módulo de biblioteca en lenguaje C se compone de al menos dos archivos, uno con extensión .h (\emph{header}) que contiene declaraciones y se utiliza para incluir el código del módulo en otro archivo; y otro de extensión .c (\emph{c code}) que contiene las implementaciones. 

En la implementación de los módulos de biblioteca sAPI se utilizan tres archivos por módulo. Un archivo \emph{module.h} que contiene la siguiente estructura:

\begin{verbatim}
  Evitar inclusión multiple{
    Inclusiones de dependencias de funciones publicas
    CPlusPlus (para cuando se usa este módulo desde C++) {
      Macros de definición de constantes publicas.
      Macros "estilo función" publicas.
      Definiciones de tipos de datos públicos.
      Prototipos de funciones publicas.
    }
  }
\end{verbatim}

Un archivo \emph{module\_private.h} con:

\begin{verbatim}
  Inclusiones de dependencias de funciones privadas.
  Macros de definición de constantes privadas.
  Macros "estilo función" privadas.
  Definiciones de tipos de datos privados.
  Prototipos de funciones privadas.
\end{verbatim}

Y un archivo \emph{module.c} con:

\begin{verbatim}
  Definiciones de variables globales públicas externas (extern).
  Definiciones de variables globales públicas.
  Definiciones de variables globales privadas (static).
  Implementaciones de funciones públicas.
  Implementaciones de funciones privadas.
\end{verbatim}

En los módulos de sAPI no se utilizan variables globales públicas y variables públicas externas para mantenerlos desacoplados.

Con esta estructura se definen las clases para formar un módulo de biblioteca en lenguaje C.

Además, se definen las clases para generar en lenguaje C todos los tipos de datos, constantes y modificadores de acceso y objetos definidos en los módulos de biblioteca sAPI. A partir de todas estas clases se formó el generador de código en lenguaje C y documentación.

Para la implementación tanto del generador de biblioteca sAPI, como de las clases que definen el modelo se utilizó el lenguaje \emph{JavaScript} \citep{JavaScript} y el entorno de ejecución \emph{NodeJS} \citep{NodeJS}. El motivo principal de esta decisión es permitir en un futuro realizar la definición de plataformas y generación de código desde una plataforma web. Estas herramientas permiten generar aplicaciones de escritorio, web o móviles con mínimos esfuerzos.

De esta forma, para utilizar el generador se debe primero definir los archivos dependientes de la plataforma y luego ejecutar el generador desde una terminal con NodeJS como se esquematiza en figura \ref{fig:fullSapiGen}, que resume el producto de este trabajo final.

\begin{figure}[!htbp]
\begin{center}  % [width=14cm,height=8cm] [width=\textwidth]
\includegraphics*[width=14.5cm]{Figures/sapi_gen2.pdf}
\par\caption{Generador de la biblioteca sAPI.}\label{fig:fullSapiGen}
\end{center}
\end{figure}

Tomado los campos de descripción que incluye cada entidad del modelo genera la documentación en lenguaje \emph{Markdown}. 

%------------------------------------------------------------------------------
\section{Características de la implementación de sAPI en lenguaje C}
\label{sec:codeImplemC}

Un usuario de la biblioteca sAPI puede elegir el nivel de abstracción deseado a la hora de programar su aplicación. De menor a mayor abstracción tendremos programas:

\begin{enumerate}
\item
Dependientes del chip: utilizando la API genérica y los nombres de los pines y periféricos del \emph{chip}, definidos por el fabricante (opcionalmente con el agregado de funciones específicas de periféricos de cierto \emph{chip}).
\item
Dependientes de la placa: mediante la API genérica y los nombres de la serigrafía de la placa, incluyendo pines y periféricos del chip, así como otros componentes de la placa (por ejemplo, otros chips, conectores, puertos de comunicación).
\item
Portables entre placas compatibles: usando la API genérica y los nombres genéricos para todos.
\item
Totalmente portables: Mediante la API genérica y es responsabilidad del usuario agregar un archivo de nombres, por ejemplo, \emph{remap.h} donde elija los nombres de los nombres genéricos para todo lo que va a usar.
\end{enumerate}
