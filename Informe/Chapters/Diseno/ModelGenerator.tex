%------------------------------------------------------------------------------
%	SECTION
%------------------------------------------------------------------------------
\section{Generador de sAPI en lenguaje C}
\label{sec:modelGenerator}

El generador de sAPI en lenguaje C toma como entrada el una instancia de \emph{board} y \emph{drivers} específicos, junto con los \emph{módulos sAPI}, y genera como resultado una estructura de archivos y carpetas con la biblioteca sAPI en lenguaje C para una plataforma particular. También permite generación de documentación a partir del mismo modelo.

Para su diseño se debió estandarizar la forma de generar un módulo de código en lenguaje C.
%(sección \ref{sec:moduloEnC}). 
Una vez realizado, se modelaron las clases para generar un módulo en lenguaje C a partir de los modelos desciptos.
%(\ref{sec:modelModuloEnC}). 

%------------------------------------------------------------------------------
\subsection{Módulo de sAPI en lenguaje C}
\label{sec:moduloEnC}

Un módulo de código en lenguaje C posee 

%------------------------------------------------------------------------------
\subsection{Modelo de módulo en lenguaje C}
\label{sec:modelModuloEnC}

%------------------------------------------------------------------------------
\subsubsection{Generación automática de código fuente}
\label{sec:genCIndep}

Generación de código en lenguaje C independiente del hardware.


%------------------------------------------------------------------------------
\subsection{Generación de documentación en base al modelo}
\label{sec:genDoc}



%------------------------------------------------------------------------------
\subsection{Implementación del generador}

Se elige para la implementación el lenguaje \emph{Javascrip} [] y el entorno de ejecución \emph{NodeJS} []. El motivo principal de esta decisión es permitir en un futuro realizar la definición de plataformas y generación de código desde una plataforma web. Estas herramientas permiten generar aplicaciones de escritorio, web o móviles con mínimos esfuerzos.

Para llevarlo a cabo se implementaron en lenguaje \emph{Javascrip} todas las clases presentadas en las secciones \ref{sec:modelHardware} y \ref{sec:modelLibrary}, así como las clases que definen al generador de la biblioteca en lenguaje C.
