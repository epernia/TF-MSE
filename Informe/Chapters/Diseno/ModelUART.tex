%------------------------------------------------------------------------------
\subsection{Módulo UART}

El módulo UART modela un periférico de comunicación serie transmisor/receptor asincrónico universal.

\titulo{Propiedades de UART}

\begin{itemize}
\item
\emph{baudRate}, del tipo \emph{int\_t}, tasa de baudios. Depende de la arquitectura pero en general se permiten: 1200, 2400, 4800, 9600, 19200, 38400, 57600 o 115200.
\item
\emph{dataBits}, del tipo \emph{UartDataBits\_t}, que puede tomar los valores \emph{DATABITS5}, \emph{DATABITS6}, \emph{DATABITS7}, \emph{DATABITS8} o \emph{DATABITS9}.
\item
\emph{parity}, del tipo \emph{UartParity\_t}, con valores posibles \emph{NONE}, \emph{EVEN} o \emph{ODD}.
\item
\emph{stopBits}, del tipo \emph{UartStopBits\_t}, con valores \emph{STOPBITS1}, \emph{STOPBITS2} o \emph{STOPBITS1\_5}.
\item
\emph{flowControl}, del tipo \emph{UartFlowCtrl\_t}, cuyos valores posibles son \emph{NONE}, \emph{CTS}, \emph{RTS} o \emph{RTS\_CTS}.
\item
\emph{enableTransmitter}, del tipo \emph{bool\_t}, se utiliza para habilitar/deshabilitar el transmisor de la UART.
\item
\emph{enableReceiver}, del tipo \emph{bool\_t}, permite habilitar/deshabilitar el receptor de la UART.
\item
\emph{location}, del tipo \emph{Location\_t}.
\item
\emph{transmitValue}, del tipo \emph{byte\_t}. Representa el valor a transmitir.
\item
\emph{receiveValue}, del tipo \emph{byte\_t}. Representa el último valor recibido. 
\end{itemize}

\titulo{Métodos de UART}

Inicialización y restablecimiento de un periférico UART:

\begin{verbatim}
   • init( UartInit_t baudRate | 
           dataBits | 
           parity | 
           stopBits | 
           flowControl |
           enableTransmitter | 
           bool_t enableReceiver |
           location
         )
   • deInit()
\end{verbatim}

Todos los valores del método \emph{init()} son opcionales. En caso de no inicializarlos los valores por defecto son:

\begin{verbatim}
   • baudRate = 9600
   • dataBits = DATABITS8
   • parity = NONE
   • stopBits = STOPBITS1
   • flowControl = NONE
   • loacation = LOCATION0
   • enableTransmitter = TRUE
   • enableReceiver = TRUE
\end{verbatim}

Se puede transmitir o recibir un byte usando el módulo UART tanto por encuesta como por interrupción mediante los métodos:

\begin{verbatim}
   • status() : UartStatus_t
   • send( const byte_t value )
   • receive() : byte_t
   • eventCallbackSet( ChannelEvent_t evt, Callback_t c )
   • eventCallbackClear( ChannelEvent_t evt )
   • interrupt( bool_t enable )
\end{verbatim}
  
Los posibles valores de \emph{UartStatus\_t} son:

%\emph{TRANSMIT\_COMPLETE}, \emph{RECEIVE\_COMPLETE}, \emph{TRANSMIT\_DATA\_OVERRUN}, \emph{RECEIVE\_FRAME\_ERROR}, \emph{RECEIVE\_DATA\_OVERRUN},  \emph{RECEIVE\_PARITY\_ERROR}, \emph{TRANSMIT\_TIMEOUT} y \emph{RECEIVE\_TIMEOUT}

\begin{verbatim}
   • TRANSMITTER_READY
   • TRANSMIT_COMPLETE
   • RECEIVE_COMPLETE
   • ERROR_TRANSMIT_DATA_OVERRUN
   • ERROR_RECEIVE_FRAMER
   • ERROR_RECEIVE_DATA_OVERRUN
   • ERROR_RECEIVE_PARITY
   • ERROR_TIMEOUT
\end{verbatim}

Y los posibles eventos:

%\emph{TRANSMIT\_COMPLETE}, \emph{RECEIVE\_COMPLETE}, \emph{TRANSMIT\_ERROR}, \emph{RECEIVE\_ERROR} 

\begin{verbatim}
   • TRANSMITTER_READY
   • TRANSMIT_COMPLETE
   • RECEIVE_COMPLETE
   • ERROR_TRANSMIT
   • ERROR_RECEIVE
\end{verbatim}

Existe además el método \emph{sendBreak()} que envía una \emph{break condition} al bus. Esto significa que mantiene la línea de transmisión en nivel bajo durante un tiempo mayor al requerido para enviar un caracter.

\begin{verbatim}
   • sendBreak()
\end{verbatim}

\titulo{Métodos de UART de alto nivel}

Métodos de lectura de byte:

\begin{verbatim}
   • readByte() : byte_t
   • readByteSync( inOut byte_t value, 
                   Time_t timout
                 ) : UartStatus_t
   • readByteAsync( Callback_t sucess, 
                    Callback_t error 
                  )
\end{verbatim}

Métodos de escritura de byte:

\begin{verbatim}
   • writeByte( const byte_t value )
   • writeByteSync( const byte_t value, 
                    Time_t timout 
                  ) : UartStatus_t
   • writeByteAsync( const byte_t value, 
                     Callback_t sucess,
                     Callback_t error 
                   )
\end{verbatim}

Métodos de lectura de \emph{String}:

\begin{verbatim}
   • readString( inOut String_t data,
                 inOut size_t dataSize,
                 String_t terminator
               )
   • readStringSync( inOut String_t data,
                     inOut size_t dataSize,
                     String_t terminator,
                     Time_t timout 
                  ) : UartStatus_t
   • readStringAsync( inOut String_t data,
                      inOut size_t dataSize,
   	                  String_t terminator,
                      Callback_t sucess, 
                      Callback_t error 
                    ) 
\end{verbatim}

Métodos de escritura de \emph{String}:                  
                
\begin{verbatim}  
   • writeString( const String_t data
                  inOut size_t dataSize
                )
   • writeStringSync( const String_t data,
                      inOut size_t dataSize,
                      Time_t timout
                    ) : UartStatus_t
   • writeStringAsync( const String_t value, 
                      inOut size_t dataSize,
                      Callback_t sucess,
                      Callback_t error
                     )
\end{verbatim}

Métodos de lectura de arreglo de bytes:

\begin{verbatim}
   • readByteArray( inOut ByteArray_t data,
                    inOut size_t dataSize
                  )
   • readByteArraySync( inOut ByteArray_t data,
                        inOut size_t dataSize,
                        Time_t timout
                      ) : UartStatus_t
   • readByteArrayAsync( inOut ByteArray_t data, 
                         inOut size_t dataSize,
                         Callback_t sucess,
                         Callback_t error 
                       )
\end{verbatim}

Métodos de escritura de arreglo de bytes:

\begin{verbatim}
  • writeByteArray( const ByteArray_t data,
                    inOut size_t dataSize
                  )
  • writeByteArraySync( const ByteArray_t data,
                        inOut size_t dataSize,
                        Time_t timout
                      ) : UartStatus_t
  • writeByteArrayAsync( const ByteArray_t data, 
                         inOut size_t dataSize,
                         Callback_t sucess,
                         Callback_t error
                       )
\end{verbatim}

En todos los métodos de lectura se le pasa como primer parámetro la variable donde escribirá los datos recibidos. En la lectura de \emph{String} se le debe pasar un parámetro \texttt{terminator} que es un \texttt{String} que indica como debe ser el fin de cadena a buscar. Cuando se lee un \emph{String} o \emph{Byte Array} se almacena la cantidad de datos leídos en la variable \emph{dataSize} que se le pasa como parámetro.
