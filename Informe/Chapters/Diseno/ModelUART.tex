%------------------------------------------------------------------------------
\subsection{Módulo UART}

El módulo UART modela un periférico de comunicación serie transmisor/receptor asincrónico universal.

\titulo{Propiedades de UART}

\begin{itemize}
\item
\emph{baudRate}, del tipo \emph{uint32\_t}, tasa de baudios. Depende de la arquitectura pero en general se permiten: 1200, 2400, 4800, 9600, 19200, 38400, 57600 o 115200.
\item
\emph{dataBits}, del tipo \emph{UartDataBits\_t}, que puede tomar los valores \emph{DATABITS\_5}, \emph{DATABITS\_6}, \emph{DATABITS\_7}, \emph{DATABITS\_8}. o \emph{DATABITS\_9}
\item
\emph{parity}, del tipo \emph{UartParity\_t}, con valores posibles \emph{NONE}, \emph{EVEN} o \emph{ODD}.
\item
\emph{stopBits}, del tipo \emph{UartStopBits\_t}, con valores \emph{STOPBITS\_1}, \emph{STOPBITS\_2} o \emph{STOPBITS\_1\_5}.
\item
\emph{flowControl}, del tipo \emph{UartFlowCtrl\_t}, cuyos valores posibles son \emph{NONE}, \emph{CTS}, \emph{RTS} o \emph{RTS\_CTS}.
\item
\emph{enableTransmitter}, del tipo \emph{bool\_t}, se utiliza para habilitar/deshabilitar el transmisor de la UART.
\item
\emph{enableReceiver}, del tipo \emph{bool\_t}, permite habilitar/deshabilitar el receptor de la UART.
\item
\emph{location}, del tipo \emph{Location\_t}.
\item
\emph{transmitValue}, del tipo \emph{uint8\_t}. Representa el valor a transmitir.
\item
\emph{receiveValue}, del tipo \emph{uint8\_t}. Representa el último valor recibido. 
\end{itemize}

\titulo{Métodos de UART}

Inicialización y restablecimiento de un periférico UART:

\begin{verbatim}
 • init( uint32_t baudRate | 
         dataBits | 
         parity | 
         stopBits | 
         flowControl |
         enableTransmitter | 
         bool_t enableReceiver |
         location
       )
 • deInit()
\end{verbatim}

Todos los parámetros del método \emph{init()} son opcionales. Los valores por defecto son:

\begin{verbatim}
 • baudRate = 9600
 • dataBits = DATABITS_8
 • parity = NONE
 • stopBits = STOPBITS_1
 • flowControl = NONE
 • loacation = LOCATION0
 • enableTransmitter = TRUE
 • enableReceiver = TRUE
\end{verbatim}

Se puede transmitir o recibir un byte usando el módulo UART tanto por encuesta como por interrupción. Para ello se utilizan las siguientes funciones:

\begin{verbatim}
 • transmitterStatus() : UartStatus_t
 • send( uint8_t value )
 • receiverStatus() : UartStatus_t
 • receive() : uint8_t
 • eventCallbackSet( ChannelEvent_t evt, Callback_t c )
 • eventCallbackClear( ChannelEvent_t evt )
 • interrupt( bool_t enable )
\end{verbatim}

Los posibles valores de \emph{UartStatus\_t} son:

%\emph{TRANSMIT\_COMPLETE}, \emph{RECEIVE\_COMPLETE}, \emph{TRANSMIT\_DATA\_OVERRUN}, \emph{RECEIVE\_FRAME\_ERROR}, \emph{RECEIVE\_DATA\_OVERRUN},  \emph{RECEIVE\_PARITY\_ERROR}, \emph{TRANSMIT\_TIMEOUT} y \emph{RECEIVE\_TIMEOUT}

\begin{verbatim}
 • TRANSMIT_COMPLETE
 • RECEIVE_COMPLETE
 • TRANSMIT_DATA_OVERRUN
 • RECEIVE_FRAME_ERROR
 • RECEIVE_DATA_OVERRUN
 • RECEIVE_PARITY_ERROR
 • TRANSMIT_TIMEOUT
 • RECEIVE_TIMEOUT
\end{verbatim}

Y los posibles eventos:

%\emph{TRANSMIT\_COMPLETE}, \emph{RECEIVE\_COMPLETE}, \emph{TRANSMIT\_ERROR}, \emph{RECEIVE\_ERROR} 

\begin{verbatim}
 • TRANSMIT_COMPLETE
 • RECEIVE_COMPLETE
 • TRANSMIT_ERROR
 • RECEIVE_ERROR
\end{verbatim}

Existe además el método \emph{sendBreak()} que envía una \emph{break condition} al bus. Esto significa que mantiene la línea de transmisión en nivel bajo durante un tiempo mayor al requerido para enviar un caracter.

\begin{verbatim}
 • sendBreak()
\end{verbatim}

\titulo{Métodos de UART de alto nivel}

Métodos de lectura y escritura de byte:

\begin{verbatim}
 • readByte() : uint8_t
 • readByteSync( uint8_t value, 
                 Time_ms_t timout
               ) : UartStatus_t
 • readByteAsync( Callback_t sucess, 
                  Callback_t error 
                )
 • writeByte( uint8_t value )
 • writeByteSync( uint8_t value, 
                  Time_ms_t timout 
                ) : UartStatus_t
 • writeByteAsync( uint8_t value, 
                   Callback_t sucess,
                   Callback_t error 
                 )
\end{verbatim}

Métodos de lectura y escritura de \emph{String}:

\begin{verbatim}
 • readString( String_t data, 
               String_t terminator
             )
 • readStringSync( String_t data, 
                   String_t terminator,
                   Time_ms_t timout 
                 ) : UartStatus_t
 • readStringAsync( String_t data, 
 	                String_t terminator,
                    Callback_t sucess, 
                    Callback_t error 
                  ) 
 • writeString( String_t data )
 • writeStringSync( String_t data,
                    Time_ms_t timout
                  ) : UartStatus_t
 • writeStringAsync( String_t value, 
                     Callback_t sucess,
                     Callback_t error )
\end{verbatim}

Métodos de lectura y escritura de vector de bytes:

\begin{verbatim}
 • readByteArray( ByteArray_t data, 
                  uint32_t dataSize
                )
 • readByteArraySync( ByteArray_t data, 
                      uint32_t dataSize,
                      Time_ms_t timout
                    ) : UartStatus_t
 • writeByteArrayAsync( ByteArray_t data, 
                        uint32_t dataSize,
                        Callback_t sucess,
                        Callback_t error 
                      )
 • writeByteArray( ByteArray_t data, 
                   uint32_t dataSize
                 )
 • writeByteArraySync( ByteArray_t data, 
                       uint32_t dataSize, 
                       Time_ms_t timout
                     ) : UartStatus_t
 • writeByteArrayAsync( ByteArray_t data, 
                        uint32_t dataSize,
                        Callback_t sucess,
                        Callback_t error
                      )
\end{verbatim}

En todos los métodos de lectura se le pasa como primer parámetro la variable donde escribirá el, o los, datos recibidos. En la lectura de \emph{String} se le debe pasar un parámetro \texttt{terminator} que es un \texttt{String} que indica como debe ser el fin de cadena a buscar. Cuando se lee un \emph{Byte Array} almacena la cantidad de datos leidos en la variable \emph{dataSize} que se le pasa como parámetro.
