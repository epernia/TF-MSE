%------------------------------------------------------------------------------
\subsection{Módulo SPI}

Modela un bus de serie para interfaz con periféricos (bus SPI). El periférico SPI proporciona comunicación serial sincrónica \emph{full duplex} entre dispositivos maestros y esclavos. Se usa comúnmente para la comunicación con periféricos externos como memorias flash, sensores, relojes en tiempo real (RTC), etc.

\titulo{Propiedades de SPI}

\begin{itemize}
\item
\emph{mode}, del tipo \emph{SpiMode\_t}, que representa el modo de funcionamiento del periférico SPI, con valores \emph{MASTER} o \emph{SLAVE}.
\item
\emph{clockFrequency}, del tipo \emph{int\_t}, es la frecuencia de reloj (en Hz) del BUS SPI (solo tiene sentido en modo \emph{MASTER}).
\item
\emph{clockPolarity}, del tipo \emph{SpiClockParity\_t}, define el nivel cuando se considera activo el clock, con valores \emph{POLARITY\_LOW} o \emph{POLARITY\_HIGH}.
\item
\emph{clockPhase}, del tipo \emph{SpiClockPhase\_t}, establece en que flanco de reloj se muestrean los datos, cuyos valores posibles son \emph{PHASE\_EDGE1} o \emph{PHASE\_EDGE2}.
\item
\emph{dataBits}, del tipo \emph{SpiDataBits\_t}, que puede tomar los valores \emph{DATABITS8} o \emph{DATABITS16}.
\item
\emph{dataOrder}, del tipo \emph{SpiDataOrder\_t}, representa el orden con el que se mueven los bits en el registro de desplazamiento, es decir si se mueven primero el bit menos significativo o el más significativo: \emph{MSB\_FIRST} o \emph{LSB\_FIRST}.
\item
\emph{location}, del tipo \emph{Location\_t}.
\item
\emph{transmitValue}, del tipo \emph{ByteArray\_t}. Representa el valor a transmitir.
\item
\emph{receiveValue}, del tipo \emph{ByteArray\_t}. Representa el último valor recibido. 
\end{itemize}

\titulo{Métodos de SPI}

Inicialización y restablecimiento de un periférico SPI:

\begin{verbatim}
   • init()
   • init( SpiInit_t init )
   • init( SpiInit_t init, int_t clockFrequency )
   • deInit()
\end{verbatim}

Los valores posibles del parámetro \emph{init} se forman con valores de:

\begin{verbatim}
   • mode | clockPolarity | clockPhase | dataOrder |
     dataBits | location
\end{verbatim}

Son todos opcionales, con valores por defecto:

\begin{verbatim}
   • mode = MASTER
   • clockFrequency = 100000
   • clockPolarity = POLARITY_HIGH
   • clockPhase = PHASE_EDGE1
   • dataBits = DATABITS8
   • dataOrder MSB_FIRST
   • location = LOCATION0
\end{verbatim}

En caso de seleccionar modo \emph{MASTER} se debe configurar la frecuencia de reloj y al menos un pin como selector de esclavo. La frecuencia de reloj puede configurarse usando el método \emph{init()} de dos parámetros, o bien, mediante:

\begin{verbatim}
   • clockFrequencySet( int_t clockFrequency )
\end{verbatim}

Utilizando \emph{initSSPin()} se configura un pin como selector de esclavo:

\begin{verbatim}
   • initSSPin( Pin_t pin )
\end{verbatim}

Luego para elegir que esclavo habilitar o deshabilitar se utiliza:

\begin{verbatim}
   • slaveSelect( Pin_t pin, bool_t enable )
\end{verbatim}

Permite realizar envío, recepción o transferencia de datos (que envía y recibe simultáneamente). Estas operaciones pueden realizarse por encuesta o interrupción con los siguientes métodos:

\begin{verbatim}
   • status() : SpiStatus_t
   • send( const ByteArray_t data )
   • receive( inOut ByteArray_t receive )
   • transfer( const ByteArray_t send, 
               inOut ByteArray_t receive
             ) : SpiStatus_t
   • eventCallbackSet( SpiEvent_t evt, Callback_t func )
   • eventCallbackClear( SpiEvent_t evt )
   • interrupt( bool_t enable )
\end{verbatim}

Los posibles valores de \emph{SpiStatus\_t} son:

\begin{verbatim}
   • READY
   • BUSY
   • TRANSFER_COMPLETE
   • ERROR
   • ERROR_TIMEOUT
\end{verbatim}

Eventos (tipo \emph{SpiEvent\_t}):

\begin{verbatim}
   • TRANSFER_COMPLETE
   • TRANSFER_ERROR
\end{verbatim}

\titulo{Métodos de SPI de alto nivel}

% Los métodos de escritura y lectura de a un byte no tienen sentido pues si el tamaño de datos no es 8, entonces siempre es byte array.

Lectura de arreglo de bytes:

\begin{verbatim}
   • readByteArray( inOut ByteArray_t data,
                    inOut size_t dataSize
                  )
   • readByteArraySync( inOut ByteArray_t data,
                        inOut size_t dataSize,
                        Time_t timeout
                      ) : SpiStatus_t
   • readByteArrayAsync( inOut ByteArray_t data, 
                         inOut size_t dataSize,
                         Callback_t sucess,
                         Callback_t error 
                       )
\end{verbatim}

Escritura de arreglo de bytes:

\begin{verbatim}
   • writeByteArray( const ByteArray_t data,
                     inOut size_t dataSize
                   )
   • writeByteArraySync( const ByteArray_t data,
                         inOut size_t dataSize,
                         Time_t timeout
                       ) : SpiStatus_t
   • writeByteArrayAsync( const ByteArray_t data, 
                          inOut size_t dataSize,
                          Callback_t sucess,
                          Callback_t error 
                        )
\end{verbatim}

\pagebreak

Transferencia de arreglo de bytes:

\begin{verbatim}
   • transferByteArray( const ByteArray_t send,
                        inOut size_t sendSize,
                        inOut ByteArray_t receive,
                        inOut size_t dataSize
                      )
   • transferByteArraySync( const ByteArray_t send,
                            inOut size_t sendSize,
                            inOut ByteArray_t receive,
                            inOut size_t dataSize,
                            Time_t timeout
                          ) : SpiStatus_t
   • transferByteArrayAsync( const ByteArray_t send,
                             inOut size_t sendSize,
                             inOut ByteArray_t receive,
                             Callback_t sucess,
                             Callback_t error 
                           )
\end{verbatim}

% Para la implementación:

% https://github.com/epernia/sAPI/blob/develop/toMergeIn_sapi1Project/sapi_soc_peripherals/inc/sapi_spi.h

%  https://github.com/epernia/sAPI/blob/develop/toMergeIn_sapi1Project/sapi_soc_peripherals/src/sapi_spi.c

%https://www.mikroe.com/ebooks/programming-dspic-microcontrollers-in-pascal/master-and-slave-modes
