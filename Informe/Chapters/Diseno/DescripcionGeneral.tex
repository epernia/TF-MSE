%------------------------------------------------------------------------------
%	SECTION
%------------------------------------------------------------------------------
\section{Principios de diseño}
\label{sec:principiosDiseno}

Se utilizan las siguientes premisas para guiar el diseño de la biblioteca sAPI:

\begin{itemize}
\item
Modelado a nivel de \emph{chip} (ya sea MCU\footnote{Abreviatura de microcontrolador.} o SoC\footnote{Abreviatura de sistema en un chip.}):
\begin{itemize}
\item
Dar soporte a la programación de núcleos de procesamiento y periféricos, utilizando los modos más comunes de funcionamiento de los mismos.
\item
Debe poseer un nivel de abstracción suficiente para independizarse del hardware pero manteniendo la identidad y conceptos claves de cada cada periférico que compone el chip.
\item
Especificar de manera explícita las dependencias entre módulos de la biblioteca y el uso de recursos físicos en cada implementación particular.
\end{itemize}
\item
A nivel de biblioteca:
\begin{itemize}
\item
Utilización de nombres sencillos para facilitar el aprendizaje y uso.
\item
Mantener baja la extensión de la definición de la API.
\item
Especificar las dependencias entre módulos de la biblioteca.
\end{itemize}
\end{itemize}

%------------------------------------------------------------------------------
%	SECTION
%------------------------------------------------------------------------------
\section{Arquitectura de una aplicación}
\label{sec:arqApp}

La biblioteca debe aislar la aplicación de usuario del hardware subyacente formando una capa de abstracción del hardware. De esta manera, una aplicación que utiliza la biblioteca contiene al menos las capas de software descriptas en la figura \ref{fig:sapiCapas1}.

\begin{figure}[!htbp]
\begin{center}  % [width=14cm,height=8cm] [width=\textwidth]
\includegraphics*[width=10.4cm]{Figures/sapiCapas1.png}
\par\caption{Biblioteca sAPI como capa de abstracción del hardware.}\label{fig:sapiCapas1}
\end{center}
\end{figure}

Según la implementación de la biblioteca para una plataforma de hardware particular, se puede aprovechar bibliotecas de drivers existentes provistas por el fabricante para su interfaz con el hardware. Además, en una aplicación de embebidos típica, se combinará con un sistema operativo de tiempo real, \emph{stracks} y \emph{middelware} resultando en las capas de software de la figura \ref{fig:sapiCapas2}.

\begin{figure}[!htbp]
\begin{center}  % [width=14cm,height=8cm] [width=\textwidth]
\includegraphics*[width=10.4cm]{Figures/sapiCapas2.png}
\par\caption{Arquitectura de una aplicación que utiliza la biblioteca.}\label{fig:sapiCapas2}
\end{center}
\end{figure}

%------------------------------------------------------------------------------
%	SECTION
%------------------------------------------------------------------------------
\section{Arquitectura de la biblioteca sAPI}
\label{sec:arqSapi}

La biblioteca sAPI modela todo el hardware que necesita acceder tanto la aplicación de usuario como las otras capas superiores de software. Utiliza un diseño modular que facilita la reutilización de sus partes. Estos módulos pueden agruparse en:

\begin{itemize}
\item
Plataforma de hardware (en adelante \emph{board}): son los módulos que describen cierta placa de sistema embebido completa.
\item
Periféricos externos conectados a la \emph{board}: son los módulos que modelan, por ejemplo, un chip conectado al I2C o al SPI.
\item
Módulos abstractos: representan módulos que agregan un mayor nivel de abstracción, los cuales permiten realizar tareas más avanzadas como, por ejemplo, buffers.
\end{itemize}

Los periféricos externos y módulos abstractos utilizan los módulos en el grupo \emph{board} y en consecuencia no dependen del hardware subyacente.
