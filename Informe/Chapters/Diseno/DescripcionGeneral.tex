%------------------------------------------------------------------------------
%	SECTION
%------------------------------------------------------------------------------
\section{Descripción general}
\label{sec:descripGralDiseno}

El sistema propuesto para la creación de bibliotecas de software para la programación de plataformas de hardware se ilustra en la figura \ref{fig:sapi_gen}.

\begin{figure}[!htbp]
\begin{center}  % [width=14cm,height=8cm] [width=\textwidth]
\includegraphics*[width=14cm]{Figures/sapi_gen.pdf}
\par\caption{Diagrama del sistema diseñado.}\label{fig:sapi_gen}
\end{center}
\end{figure}

Se diseñó un modelo de plataforma de hardware (en adelante \emph{board} por sus siglas en inglés). Este modelo describe una plataforma de hardware completa y se debe instanciar para cada nueva plataforma de hardware.

Por otra parte, se desarrollaron módulos de biblioteca (en adelante módulos sAPI) independientes del hardware y lenguajes de programación. Estos definen propiedades y métodos describen las propiedades y métodos de cada uno de los periféricos de un cierto SoC, definiendo entonces, la API de la biblioteca.

Además, se debe proveer la implementación de cada uno de los métodos de los módulos sAPI (en adelante \emph{drivers}). Estos \emph{drivers} deben estar escritos en un cierto lenguaje de programación y son dependientes de la arquitectura del hardware.

También, se diseñó un generador de biblioteca sAPI. Este generador se debe definir para cada lenguaje de programación. Para este trabajo final se desarrolló solamente el generador de lenguaje C, sin embargo, el mismo se realizó de forma modular para que pueda ser fácilmente adaptado a otros lenguajes. 

Mediante el generador de C, utilizando una instancia de \emph{board} y \emph{drivers} específicos, junto con los módulos sAPI, se genera entonces una biblioteca sAPI en lenguaje C para una plataforma particular.

En las siguientes secciones de detalla el diseño e implementación de cada una de las entidades descriptas. Las mismas se modelaron utilizaron los conceptos del paradigma de la programación orientada a objetos, que son especialmente útiles para describir módulos de software que encapsulan funcionalidad. Estos se describen mediante diagramas UML de clases de forma independiente del lenguaje de programación.
