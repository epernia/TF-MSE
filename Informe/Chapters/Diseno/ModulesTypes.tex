%------------------------------------------------------------------------------
%	SECTION
%------------------------------------------------------------------------------
\section{Modelo de plataforma de hardware}
\label{sec:modelHardware}

En la figura \ref{fig:ModelBoard} se muestra el diagrama que describe una \emph{board} y las partes que la componen.

\begin{figure}[!htbp]
\begin{center}  % [width=14cm,height=8cm] [width=\textwidth]
\includegraphics*[width=14cm]{Figures/Board.pdf}
\par\caption{Arquitectura de una aplicación que utiliza la biblioteca.}\label{fig:ModelBoard}
\end{center}
\end{figure}

Estas partes son:

\begin{itemize}
\item
Circuitos Integrados (en adelante IC, por sus siglas en inglés): son los módulos que describen circuitos integrados necesarios para programar la plataforma, por ejemplo, MCU, sensores y memorias montados en la placa. Contiene una lista de terminales de conexión. 
\item
Conectores: son los conectores físicos de la placa, como ser, pines, borneras, etc.
\item
Conexiones: modela el mapa de conexiones entre circuitos integrados y conectores. Es un diccionario (conjunto de asociaciones clave-valor), donde se tienen asociaciones de terminales de conexión. 
\end{itemize}

