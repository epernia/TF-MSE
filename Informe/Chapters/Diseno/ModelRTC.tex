%------------------------------------------------------------------------------
\subsection{Módulo RTC}

El módulo RTC modela un periférico Reloj de Tiempo Real. Este periférico permite mantener la hora y fecha en un sistema.

\titulo{Propiedades de RTC}

\begin{itemize}
\item
\emph{dateAndTime}, contiene la fecha/hora actual, del tipo \emph{DateAndTime\_t}.
\item
\emph{alarm}, establece la fecha/hora para un evento de alarma, del tipo \emph{DateAndTime\_t}.
\item
\emph{interval}, del tipo \emph{Time\_t}, establece el intervalo de tiempo para un evento periódico.
\end{itemize}

\pagebreak
\titulo{Métodos de RTC}

Inicialización y restablecimiento de un periférico RTC:

\begin{verbatim}
   • init( DateAndTime_t absTime )
   • deInit()
\end{verbatim}

El parámetro \emph{absTime} establece la fecha/hora actual. 

La lectura del RTC se realiza con el método \emph{read()}, que retorna la fecha/hora actual. Para cambiar la fecha/hora se utiliza el método \emph{write()}:

\begin{verbatim}
   • read() : DateAndTime_t
   • write( DateAndTime_t absTime )
\end{verbatim}

Para establecer una interrupción en un RTC ante cierto evento define los métodos: 

\begin{verbatim}
   • eventCallbackSet( RtcEvent_t evt, Callback_t func )
   • eventCallbackClear( RtcEvent_t evt )
   • interrupt( bool_t enable )
\end{verbatim}

Los eventos posibles son: \emph{ALARM}, que genera una única interrupción cuando se cumple cierto valor absoluto de fecha/hora, y \emph{PERIODIC}, que interrumpe de forma periódica cada cierto lapso de tiempo.
