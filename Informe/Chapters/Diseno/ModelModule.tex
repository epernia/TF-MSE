%------------------------------------------------------------------------------
%	SECTION
%------------------------------------------------------------------------------
\section{Módulos de biblioteca sAPI}
\label{sec:modelLibrary}

Se utilizan las siguientes premisas para guiar el diseño de los módulos de la biblioteca sAPI:

\begin{itemize}
\item
%A nivel de biblioteca:
Utilización de nombres sencillos para facilitar el aprendizaje y uso.
\item
Mantener baja la extensión de la definición de la API.
\item
%Modelado a nivel de SoC:
Dar soporte a la programación de núcleos de procesamiento y periféricos, utilizando los modos más comunes de funcionamiento de los mismos, con un nivel de abstracción suficiente para independizarse del hardware, pero manteniendo la identidad y conceptos de cada cada periférico.
\item
Debe especificar las dependencias entre módulos de la biblioteca, y el uso de recursos físicos en cada implementación particular.
\item
La definición de los módulos se realiza independiente del lenguaje de programación.
\end{itemize}

Para el diseño de la API se llevó a cabo un estudio exhaustivo sobre arquitecturas de núcleos de procesamiento y periféricos, así como de las bibliotecas disponibles. Entre las bibliotecas relevadas se destacan:

\begin{itemize}
\item
MicroPython libraries []. % https://docs.micropython.org/en/latest/library/machine.html
\item
STM32 HAL []. % https://www.st.com/resource/en/user_manual/dm00105879.pdf
\item
RUST Embedded HAL []. % https://docs.rs/embedded-hal/0.2.2/embedded_hal/
\item
Mbed OS drivers API []. % https://os.mbed.com/docs/v5.10/
\item
ChibiOS/HAL []. % http://www.chibios.org/dokuwiki/doku.php?id=chibios:product:hal:start
\item
CMSIS [].
\item
LPCOpen [].
\item
MCUXpresso [].
\item
EMLIB []. % https://siliconlabs.github.io/Gecko_SDK_Doc/efm32hg/html/index.html
\end{itemize}

Se realizó un rediseño completo de la API, mejorando y apliando muchas características, pero provocando un quiebre en la compatibilidad con versiones anteriores de sAPI.

%------------------------------------------------------------------------------
\subsection{Definiciones de abstractas de sAPI}

Para la creación de módulos de sAPI independientes del lenguaje de programación se definen tipos de datos, constantes y modos de acceso de parámetros de funciones abstractos, cuya implementación luego dependerá de lenguaje particular. Además, cada módulo agrega sus propias definiciones.

\titulo{Constantes}

Los nombres de las constantes se escriben en mayúsculas y con guiones bajos para separar palabras múltiples. 

\begin{itemize}
\item
Estados lógicos: \emph{TRUE} y \emph{FALSE}.
\item
Estados funcionales: \emph{ON} y \emph{OFF}.
\item
Estados eléctricos: \emph{HIGH} y \emph{LOW}.
\item
Estados de habilitación: \emph{ENABLE} y \emph{DISABLE}.
\item
Valor nulo: \emph{NULL}.
%\item
%Valor indefinido: \emph{UNDEFINED}.
\end{itemize}

\titulo{Tipos de datos}

Los tipos de datos se indican con el sufijo \emph{\_t} para que se distinga a simple vista los tipos de datos de las variables o parámetros formales de los métodos.

Tipos de datos primitivos:

\begin{itemize}
\item
Booleano: \emph{bool\_t}. 
%Cualquiera de las constantes definidas previamente se considera un valor booleano válido, además de 0 y !0 (cualquier valor distinto de 0).
\item
Entero: \emph{int\_t}.
\item
Flotante: \emph{float\_t}.
\item
Byte: \emph{byte\_t}.
\end{itemize}

Tipo de datos para manejo de cadenas de caracteres:

\begin{itemize}
\item
\emph{String\_t}.
\end{itemize}

Arreglos:

\begin{itemize}
\item
\emph{Array\_t}. Permite guardar elementos ordenados. Longitud fija.
\item
\emph{List\_t}. Permite guardar elementos ordenados. Longitud variable.
\item
\emph{Set\_t}. Permite guardar elementos sin orden. Longitud variable.
\item
\emph{ByteArray\_t}. Arreglo de bytes.
\end{itemize}

%\item
%Enteros con signo, tienen un rango de valores de $-2^{(N-1)}$ a $2^{(N-1)} - 1$, (donde N = número de bits) y formato complemento a 2. Estos son: \emph{int8\_t}, \emph{int16\_t}, \emph{int32\_t} e \emph{int64\_t}.
%\item
%Enteros sin signo, con rango de valores de $0$ a $2^{N} - 1$. Estos son: \emph{uint8\_t} (\emph{byte\_t}), \emph{uint16\_t} (\emph{word\_t}), \emph{uint32\_t} (\emph{lword\_t}) y \emph{uint64\_t} (\emph{dword\_t}).
%\item
%Floatantes: representa valores con punto flotante IEEE-754. Estos son \emph{float32\_t} y \emph{float64\_t}.


%\item
%Vacío: \emph{void\_t}. Representa el nulo de los tipos de datos. Se utiliza para los métodos que no retornan nada.
%\item
%Error: \emph{Error\_t}. Representa valores de error.

Tipos de datos para temporización:

\begin{itemize}
\item
\emph{Time\_t}, representa una duración de tiempo en días horas, minutos, segundos y milisegundos.
\item
\emph{TimeOfDay\_t}, representa un valor de tiempo absoluto en horas, minutos, segundos y milisegundos. El rango es de $00:00:00.000$ a $23:59:59.999$.
\item
\emph{Date\_t}, representa una fecha (año, mes y día).
\item
\emph{DateAndTime\_t}, que contiene ambos valores anteriores.
\end{itemize}

Tipo de datos para manejo de eventos:

\begin{itemize}
\item
\emph{Callback}: \emph{Callback\_t}. Representa a una función a llamar en respuesta a un evento.
\end{itemize}

%\item
%Parámetros de \emph{Callback}: \emph{CallbackParams\_t}. Representa los parámetros de una función a llamar en respuesta a un evento.

\titulo{Modificadores}

Tanto para las variables como parámetros formales se definen los siguientes modificadores:

\begin{itemize}
\item
\emph{in}: la variable o parámetro que se precede con el modificador \emph{in} sólo puede ser leída.
\item
\emph{out}: la variable o parámetro que se precede con el modificador \emph{out} es de solo escritura. 
\item
\emph{inOut}: la variable o parámetro que se precede con el modificador \emph{inOut} puede ser tanto leída como escrita.
\item
\emph{const}: la variable o parámetro que se precede con el modificador \emph{const} no se puede modificar.
\end{itemize}

En caso de no especificar un modificador el comportamiento por defecto es que las variables se comportan como \emph{inOut}, mientras que los parámetros formales de métodos se comportan como \emph{in} dentro del mismo.

%------------------------------------------------------------------------------
\subsection{Módulo de sAPI}

Cada módulo de biblioteca sAPI define:

\begin{itemize}
\item
Nombre del módulo.
\item
Descripción. Se utiliza para la generación de documentación.
\item
Versión del módulo. Se utiliza versionado semántico [].
\item
Autor del módulo. Se debe completar nombres, apellidos y dirección de correo electrónico del autor. 
\item
Licencia. Se puede completar con el texto completo de la licencia del módulo o un enlace a la misma. Los módulos sAPI utilizan la licencia de código abierto \emph{BSD-3clause} [].
\item
Dependencias. Una lista de los módulos de sAPI de los cuales depende el módulo actual. Se puede además exigir que la dependencia sea a una versión específica, a partir de una versión mínima o hasta cierta versión máxima.
\end{itemize}
% falta inherits y métodos de clases

Además, define un conjunto de tipos de datos, constantes, propiedades y métodos.

En los módulos de sAPI se encuentran las siguientes categorías de propiedades (o atributos):

\begin{itemize}
\item
Configuración: corresponde a las propiedades para la configuración de modos de funcionamiento de un módulo. Por ejemplo, \emph{power}, \emph{clockSource}, etc.
\item
Valor: un cierto valor que se puede leer o escribir en un módulo.
\item
Eventos: representa los eventos que controla el módulo.
\end{itemize}
%polling_events (List)
%interrupt_events (List)
%event_callbacks (List)

Todos los módulos poseen los siguientes métodos:

\begin{itemize}
\item
\emph{init} (inicialización): es un método para inicializar el módulo. Permite establecer múltiples parámetros de configuración.
\item
\emph{read} (leer): permite leer el valor característico del módulo.
\item
\emph{write} (escribir): permite escribir un valor característico del módulo.
\item
\emph{deInit} (restablecer configuración): es un método para restablecer el módulo a la configuración por defecto.
\item
\emph{interrupt}: es un método para habilitar/deshabilitar interrupciones del módulo.
\item
\emph{getters y setters} de cada propiedad.
\end{itemize}

Se destaca que en una implementación de la biblioteca para lenguajes no orientados a objetos, en lugar de métodos se tiene funciones y por lo tanto, se deberá agregar en cada una el parámetro \emph{this} que corresponde a una referencia al objeto sobre el cual se ejecuta el método. Por ejemplo, en un lenguaje con objetos como \emph{JavaScript} tendremos,

\begin{verbatim}
   • uart0.init( 9600 )
     <objeto>.<método>( <parámetros> )
\end{verbatim}  

mientras que en un lenguaje sin objetos como \emph{C},

\begin{verbatim}
   • uartInit( UART0, 9600 )
     <función>( <objeto>, <parámetros> )
\end{verbatim}

donde además, por problemas de espacio de nombres, se debe agregar un prefijo con el nombre del periférico.

%------------------------------------------------------------------------------

%------------------------------------------------------------------------------
\subsection{Módulo GPIO}

El módulo GPIO modela tanto un único pin de entrada/salida de propósito general (pin), así como a un conjunto de pines (puerto). 

\titulo{Tipos de datos}

\begin{itemize}
\item
\emph{Pin\_t}, modela un pin.
\item
\emph{Port\_t}, representa un puerto.
\item
\emph{PortValue\_t}, representa el valor de todos los pines del puerto.
\end{itemize}

La cantidad de pines que forman un puerto depende de la arquitectura.

\titulo{Propiedades de pin}

\begin{itemize}
\item
\emph{mode}, del tipo \emph{PinMode\_t}, cuyos valores posibles se listan en la tabla \ref{tab:PinModeT}.


\begin{table}[h]
	\centering	
	\begin{tabular}{l l}   
		\toprule
		\textbf{Valor} 	    & \textbf{Descripción}  \\
		\midrule
		DISABLE	 & Deshabilitado \\		
		INPUT	    & Pin configurado como entrada	\\
		OUTPUT	 & Pin configurado como salida, modo \emph{push-pull}	\\
		OUTPUT\_PUSHPULL & Pin configurado como salida, modo \emph{push-pull}	\\
		OUTPUT\_OPEN\_DRAIN	 & Pin como salida, en modo drenador abierto	\\
		\bottomrule
		\hline
	\end{tabular}
	\caption[Valores posibles del tipo \emph{PinMode\_t}.]{Valores posibles del tipo \emph{PinMode\_t} y su descripción.}
	\label{tab:PinModeT}
\end{table}

\item
\emph{pull}, del tipo \emph{PinPull\_t}. Los valores posibles se listan en la tabla \ref{tab:PullT}.


\begin{table}[h]
	\centering	
	\begin{tabular}{l l}   
		\toprule
		\textbf{Valor} 	    & \textbf{Descripción} \\
		\midrule
		NONE	       & Modo por defecto, pin flotante \\		
		PULL\_UP	    & Pin con resistencia de \emph{pull-up} \\
		PULL\_DOWN	 & Pin con resistencia de \emph{pull-down} \\
		PULL\_BOTH	 & Pin con ambas resistencias \\
		\bottomrule
		\hline
	\end{tabular}
	\caption[Valores posibles del tipo \emph{PinPull\_t}.]{Valores posibles del tipo \emph{PinPull\_t} y su descripción.}
	\label{tab:PullT}
\end{table}

\item
\emph{value}, del tipo \emph{bool\_t}. Representa el valor a escribir o leer de un pin.
\end{itemize}

\titulo{Métodos de pin}

Los métodos básicos de configuración y uso de un pin son:

\begin{verbatim}
   • init( PintInit_t init )
   • read() : bool_t
   • write( bool_t value )
   • deInit()
\end{verbatim}

%Speed. %Specifies the speed for the selected pins. %Possible values are:
%GPIO_SPEED_LOW
%GPIO_SPEED_MEDIUM
%GPIO_SPEED_FAST
%GPIO_SPEED_HIGH

El parámetro \emph{init} se debe completar con un valor posible de \emph{mode}, y opcionalmente, un valor de \emph{pull} separados con $|$, por ejemplo:

\begin{verbatim}
   • init( INPUT )
   • init( INPUT | PULL_UP )
\end{verbatim}

Contiene además el método \emph{toggle()} intercambia el valor del pin si el mismo está configurado como salida.:

\begin{verbatim}
   • toggle()
\end{verbatim}

Para establecer una interrupción en un pin ante cierto evento define los métodos: 

\begin{verbatim}
   • eventCallbackSet( PinEvent_t evt, Callback_t func )
   • eventCallbackClear( PinEvent_t evt )
   • interrupt( bool_t enable ) : PinStatus_t
 \end{verbatim} 

\emph{eventCallbackSet()} establece el evento ante el cual se interrumpe, junto a una función de \emph{callback} que se ejecutará ante su ocurrencia. \emph{eventCallbackClear()} permite remover el \emph{callback}. En la tabla \ref{tab:PinEvtT} se listan los eventos del tipo \emph{PinEvent\_t}.

\begin{table}[h]
	\centering	
	\begin{tabular}{l l}   
		\toprule
		\textbf{Valor} & \textbf{Descripción} \\
		\midrule
		HIGH\_LEVEL  & Interrupción por nivel alto \\		
		LOW\_LEVEL   & Interrupción por nivel bajo \\
		RISING\_EDGE & Interrupción por flanco ascendente \\
		FALLING\_EDGE & Interrupción por flanco descendente \\
		BOTH\_EDGES & Interrupción por flanco ascendente y descendente \\
		\bottomrule
		\hline
	\end{tabular}
	\caption[Valores posibles del tipo \emph{PinEvent\_t}.]{Valores posibles del tipo \emph{PinEvent\_t} y su descripción.}
	\label{tab:PinEvtT}
\end{table}

\emph{interrupt()} permite habilitar o deshabilitar la interrupción de hardware del pin ante la ocurrencia del evento. Según la plataforma particular, existe la posibilidad que soporte interrupción en algunos pines fijos, o que exista una cierta cantidad de interrupciones de pin y se pueda elegir a qué pines se le asignan, es por eso que el método \emph{interrupt()} retorna un valor del tipo \emph{PinStatus\_t}, cuyos valores aceptados se exponen en la tabla \ref{tab:PinStatusT}. 

\begin{table}[h]
	\centering	
	\begin{tabular}{l l}   
		\toprule
		\textbf{Valor} & \textbf{Descripción} \\
		\midrule
		STATUS\_OK & Se pudo configurar la interrupción en el pin \\		
		ERROR\_NOT\_SUPPORT & El pin no soporta capacidad de interrupción \\
		ERROR\_NO\_MORE\_AVAILABLE & Si ya se configuraron todas las interrupciones\\
		 & de hardware disponibles \\
		\bottomrule
		\hline
	\end{tabular}
	\caption[Valores posibles del tipo \emph{PinStatus\_t}.]{Valores posibles del tipo \emph{PinStatus\_t} y su descripción.}
	\label{tab:PinStatusT}
\end{table}

\titulo{Propiedades y métodos de puerto}

Puerto define la propiedad \emph{value}, del tipo \emph{PortValue\_t}. Representa el valor a escribir o leer en un puerto.

Para la configuración y uso de un puerto se definen los métodos:

\begin{verbatim}
   • init( List_t config )
   • read() : PortValue_t
   • write( PortValue_t value )
   • deInit()
\end{verbatim}

El método \emph{init()} de puerto requiere una lista de configuraciones. Los métodos \emph{read()} y \emph{write()} permiten leer y escribir el puerto completo.

Se agregan además los métodos:

\begin{verbatim}
   • pins(): List_t
   • set( PortValue_t mask )
   • reset( PortValue_t mask )
\end{verbatim}

El método \emph{pins()}, devuelve una lista de pines que componen el puerto. Los métodos \emph{set()} y \emph{reset()} permiten escribir el puerto afectado por una máscara de bits que se le pasa como parámetro.

% Para implementacion ver:

% https://github.com/epernia/sAPI/blob/develop/sapi1Project/libs/sapi_soc_peripherals/inc/sapi_gpio.h

% https://github.com/epernia/sAPI/blob/develop/sapi1Project/libs/sapi_soc_peripherals/src/sapi_gpio.c

% https://github.com/epernia/sAPI/blob/develop/toMergeIn_sapi1Project/sapi_soc_peripherals/PERIPH/GPIO/gpioInterruptLPCOpenExample.c.txt

% Ejemplos de micropython de uart interrupt de gigliotti



%------------------------------------------------------------------------------
\subsection{Módulo ADC}

El módulo ADC modela un periférico Conversor Analógico-Digital. Este periférico contiene varios canales de conversión multiplexados los cuales se pueden configurar individualmente como entradas analógicas.

\titulo{Propiedades de ADC}

\begin{itemize}
\item
\emph{conversionRate}, del tipo \emph{int\_t}, representa la tasa de conversión en Hertz. Los valores posibles dependen de cada plataforma. Se pueden utilizar alternativamente los valores genéricos: \emph{LOW}, \emph{MEDIUM}, \emph{HIGH} y \emph{VERY\_HIGH}.
\item
\emph{conversionMode}, del tipo \emph{AdcConvMode\_t}, con valores: \emph{SINGLE} (activado por software desde el programa de usuario), \emph{CONTINUOUS} (conversion periódica disparada por hardware a tasa \emph{conversionRate}).% y \emph{EXTERNAL\_TRIGGERED} (Timer Match signal o GPIO).
\item
\emph{voltageReferece}, del tipo \emph{AdcVRef\_t} cuyos valores posibles son: \emph{VCC}, \emph{INTERNAL} y \emph{EXTERNAL}.
\item
\emph{resolution}, del tipo \emph{AdcRes\_t}, cuyos valores también dependen de la plataforma pero se pueden utilizar en su lugar los valores genéricos.
\item
\emph{channelsMode}, del tipo \emph{AdcChannelsMode\_t}, con valores: \emph{SIGLE} y \emph{DIFFERENTIAL}.
\item
\emph{location}, del tipo \emph{Location\_t}. Son las posibles ubicaciones del periférico con respecto a pines físicos del SoC. Si bien dependen de la arquitectura, se definen los valores genéricos \emph{LOCATION0} a \emph{LOCATION7} para cada plataforma.
\end{itemize}

\titulo{Métodos de ADC}

Inicialización y restablecimiento de un periférico ADC:

\begin{verbatim}
   • init()
   • init( int_t conversionRate, AdcInit_t init )
   • deInit()
\end{verbatim}

Los valores posibles del parámetro \emph{init} se forman con valores de:

\begin{verbatim}
   • conversionMode | voltageReferece | resolution | 
     channelsMode | location
\end{verbatim}

Todos los valores de configuración del método \emph{init()} son opcionales y los que no se aplican se inicializan por defecto con:

\begin{verbatim}
   • conversionRate = VERY_HIGH
   • voltageReferece = VCC
   • conversionMode = SINGLE
   • resolution = VERY_HIGH
   • channelMode = SINGLE
   • loacation = LOCATION0
\end{verbatim}

Habilitación/deshabilitación de un canal particular del ADC (configuración de una entrada analógica):

\begin{verbatim}
   • channel( AdcChannel_t channel, bool_t enable )
\end{verbatim}

Para conocer los canales disponibles de un ADC existe el método 
\emph{channelsGet()} que devuelve la lista de canales del ADC. Esta lista se reduce a la mitad si el modo \emph{channelsMode} es \emph{DIFFERENTIAL)}

\begin{verbatim}
   • channelsGet() : List_t
\end{verbatim}

Si el ADC tiene configurado \emph{conversionMode = CONTINUOUS} muestrea automáticamente los canales habilitados. El método\emph{startConversion()} comienza el muestreo automático, para parar el muestreo se utiliza \emph{stopConversion()}. Si en cambio \emph{conversionMode = SINGLE}, la lectura se debe lanzar mediante el método \emph{startConversion()} que recibe un canal como parámetro. 
Una conversión se puede abortar con \emph{stopConversion()}.

\begin{verbatim}
   • startConversion( AdcChannel_t channel )
   • stopConversion()
\end{verbatim}

Existen 2 formas de leer los valores convertidos del ADC, por encuesta, o por interrupción. En el primer caso se debe chequear si finalizó la conversión con el método \emph{status()}, que retorna un un valor del tipo \emph{AdcStatus\_t}. Si se realizó la conversión correctamente se puede leer el valor convertido con el método \emph{read()}.

\begin{verbatim}
   • status() : AdcStatus_t
   • read() : AdcValue_t
\end{verbatim}

Los valores posibles de \emph{AdcStatus\_t} son: 

%\emph{CONVERTING}, \emph{CONVERSION\_COMPLETE}, \emph{CONVERSION\_COMPLETE\_OVERRUN}, \emph{CONVERSION\_ADC\_ERROR}, \emph{CONVERSION- \_TIMEOUT}. 

\begin{verbatim}
   • READY
   • BUSY
   • CONVERSION_COMPLETE
   • ERROR
   • ERROR_TIMEOUT
\end{verbatim}

Para leer un canal por interrupción se debe primero establecer una función de \emph{callback} al evento \emph{CONVERSION\_COMPLETE}, otra al evento \emph{ERROR} y luego activar la interrupción del periférico. Esto se realiza con los métodos:

\begin{verbatim}
   • eventCallbackSet( ChannelEvent_t evt, Callback_t func )
   • eventCallbackClear( ChannelEvent_t evt )
   • interrupt( bool_t enable )
\end{verbatim}

Los eventos del ADC son del tipo \emph{AdcEvent\_t}. Al completar la conversión se ejecutará una de las dos funciones.

\titulo{Métodos de Canal de ADC}

Como alternativa, existen métodos que aplican a cierto canal de ADC y proveen un mayor nivel de abstracción:

\begin{verbatim}
   • read() : AdcValue_t
   • readSync( AdcValue_t data, Time_t timeout )
               : AdcStatus_t
   • readAsync( Callback_t sucess, Callback_t error )
\end{verbatim}

Tanto \emph{read()} como \emph{readSync()} realizan una lectura bloqueante, en consecuencia, es necesario establecer un tiempo de \textit{time out} para que no bloquee indefinidamente. \emph{read()} fija automáticamente el \emph{timeout} al doble del \emph{conversionRate}, siempre retorna un valor, que a veces puede ser inválido. \emph{readSync()} devuelve el estado de conversión, para que el usuario chequee si valor convertido es válido, en ese caso lo carga en la variable \emph{data}, que recibe como parámetro. Si se establece en 0 el valor de \emph{timeout} en el método con sufijo \emph{Sync} entonces el tiempo de \texttt{timeout} será considerado infinito. Lo mismo sucede con todos los métodos que cuenten con dicho sufijo que se describen que se describen en esta sección.

El método \emph{readAsync()} realiza una lectura no bloqueante. Recibe como parámetros dos funciones de \emph{callback}, una que lanza en caso de conversión exitosa (si esto ocurre le pasa como parámetro el valor convertido) y otra en caso de error (le pasa el estado de conversión).

Este módulo no define método \emph{write()}.

% Ver de agregar Analog Comparator como evento: Comparación del valor convertido contra un valor programado, para mayor que, igual que o menor que.
%    ADC_GREATER_THAN: Modo comparación por mayor que.
%    ADC_LESS_THAN: Modo comparación por menor que.

%----------------------------------------------------------------------------------------
%	SECTION
%----------------------------------------------------------------------------------------
\section{Análisis del software}
 
La idea de esta sección es resaltar los problemas encontrados, los criterios utilizados y la justificación de las decisiones que se hayan tomado.

Se puede agregar código o pseudocódigo dentro de un entorno lstlisting con el siguiente código:

\begin{verbatim}
\begin{lstlisting}[caption= "un epígrafe descriptivo"]

	las líneas de código irían aquí...
	
\end{lstlisting}
\end{verbatim}

A modo de ejemplo:

\begin{lstlisting}[caption=Pseudocódigo del lazo principal de control.]  % Start your code-block

#define MAX_SENSOR_NUMBER 3
#define MAX_ALARM_NUMBER  6
#define MAX_ACTUATOR_NUMBER 6

uint32_t sensorValue[MAX_SENSOR_NUMBER];		
FunctionalState alarmControl[MAX_ALARM_NUMBER];	//ENABLE or DISABLE
state_t alarmState[MAX_ALARM_NUMBER];						//ON or OFF
state_t actuatorState[MAX_ACTUATOR_NUMBER];			//ON or OFF

void vControl() {

	initGlobalVariables();
	
	period = 500 ms;
		
	while(1) {

		ticks = xTaskGetTickCount();
		
		updateSensors();
		
		updateAlarms();
		
		controlActuators();
		
		vTaskDelayUntil(&ticks, period);
	}
}
\end{lstlisting}





%------------------------------------------------------------------------------
\subsection{Módulo UART}

El módulo UART modela un periférico de comunicación serie transmisor/receptor asincrónico universal.

\titulo{Propiedades de UART}

\begin{itemize}
\item
\emph{baudRate}, del tipo \emph{UartDataBits\_t}.
\item
\emph{dataBits}, del tipo \emph{UartDataBits\_t}.
\item
\emph{parity}, del tipo \emph{UartParity\_t}.
\item
\emph{stopBits}, del tipo \emph{UartStopBits\_t}.
\item
\emph{flowControl}, del tipo \emph{UartFlowCtrl\_t}.
\end{itemize}

En la tabla \ref{tab:uartProperties} se listan los posibles valores de configuración de estas propiedades. 

\begin{table}[h]
	\centering	
	\begin{tabular}{l l l l l}   
		\toprule
		\textbf{baudRate} & \textbf{dataBits} & \textbf{parity} & \textbf{stopBits} & \textbf{flowControl} \\
		\midrule
		\textbf{int\_t} & \textbf{UartDataBits\_t} & \textbf{UartParity\_t} & \textbf{UartStopBits\_t} & \textbf{UartFlowCtrl\_t} \\
		\midrule
      1200   & DATABITS5 & NONE & STOPBITS1          & NONE     \\
      2400   & DATABITS6 & EVEN & STOPBITS2          & CTS      \\
      4800   & DATABITS7 & ODD  & STOPBITS1\_5 (1.5) & RTS      \\
      9600   & DATABITS8 &      &                    & RTS\_CTS \\
      19200  & DATABITS9 &      &                    &          \\
      38400  &           &      &                    &          \\
      57600  &           &      &                    &          \\
      115200 &           &      &                    &          \\
		\bottomrule
		\hline
	\end{tabular}
	\caption[Valores posibles de las propiedades de UART]{Valores posibles de las propiedades de UART}
	\label{tab:uartProperties}
\end{table}

\begin{itemize}
\item
\emph{enableTransmitter}, del tipo \emph{bool\_t}, se utiliza para habilitar/deshabilitar el transmisor de la UART.
\item
\emph{enableReceiver}, del tipo \emph{bool\_t}, permite habilitar/deshabilitar el receptor de la UART.
\item
\emph{location}, del tipo \emph{Location\_t}.
\item
\emph{transmitValue}, del tipo \emph{byte\_t}. Representa el valor a transmitir.
\item
\emph{receiveValue}, del tipo \emph{byte\_t}. Representa el último valor recibido. 
\end{itemize}

\titulo{Métodos de UART}

Inicialización y restablecimiento de un periférico UART:

\begin{verbatim}
   • init()
   • init( int_t baudRate )
   • init( int_t baudRate, UartInit init )
   • deInit()
\end{verbatim}

Los valores posibles del parámetro \emph{init} se forman con valores de:

\begin{verbatim}
   • dataBits | parity | stopBits | flowControl |
     enableTransmitter | enableReceiver | location
\end{verbatim}

Todos los valores del método \emph{init()} son opcionales. En caso de no inicializarlos los valores por defecto son:

\begin{verbatim}
   • baudRate = 9600
   • dataBits = DATABITS8
   • parity = NONE
   • stopBits = STOPBITS1
   • flowControl = NONE
   • loacation = LOCATION0
   • enableTransmitter = TRUE
   • enableReceiver = TRUE
\end{verbatim}

Se puede transmitir o recibir un byte usando el módulo UART tanto por encuesta como por interrupción mediante los métodos:

\begin{verbatim}
   • status() : UartStatus_t
   • send( const byte_t value )
   • receive() : byte_t
   • eventCallbackSet( UartEvent_t evt, Callback_t func )
   • eventCallbackClear( UartEvent_t evt )
   • interrupt( bool_t enable )
\end{verbatim}
  
Los posibles valores de \emph{UartStatus\_t} son:

%\emph{TRANSMIT\_COMPLETE}, \emph{RECEIVE\_COMPLETE}, \emph{TRANSMIT\_DATA\_OVERRUN}, \emph{RECEIVE\_FRAME\_ERROR}, \emph{RECEIVE\_DATA\_OVERRUN},  \emph{RECEIVE\_PARITY\_ERROR}, \emph{TRANSMIT\_TIMEOUT} y \emph{RECEIVE\_TIMEOUT}

\begin{verbatim}
   • TRANSMITTER_READY
   • TRANSMIT_COMPLETE
   • RECEIVE_COMPLETE
   • ERROR_TRANSMIT_DATA_OVERRUN
   • ERROR_RECEIVE_FRAMER
   • ERROR_RECEIVE_DATA_OVERRUN
   • ERROR_RECEIVE_PARITY
   • ERROR_TIMEOUT
\end{verbatim}

Eventos (tipo \emph{UartEvent\_t}):

%\emph{TRANSMIT\_COMPLETE}, \emph{RECEIVE\_COMPLETE}, \emph{TRANSMIT\_ERROR}, \emph{RECEIVE\_ERROR} 

\begin{verbatim}
   • TRANSMITTER_READY
   • TRANSMIT_COMPLETE
   • RECEIVE_COMPLETE
   • ERROR_TRANSMIT
   • ERROR_RECEIVE
\end{verbatim}

Existe además el método \emph{sendBreak()} que envía una \emph{break condition} al bus. Esto significa que mantiene la línea de transmisión en nivel bajo durante un tiempo mayor al requerido para enviar un caracter.

\begin{verbatim}
   • sendBreak()
\end{verbatim}

\titulo{Métodos de UART de alto nivel}

Métodos de lectura de byte:

\begin{verbatim}
   • readByte() : byte_t
   • readByteSync( inOut byte_t value, Time_t timout )
                   : UartStatus_t
   • readByteAsync( Callback_t sucess, Callback_t error )
\end{verbatim}

Métodos de escritura de byte:

\begin{verbatim}
   • writeByte( const byte_t value )
   • writeByteSync( const byte_t value, Time_t timout )
                    : UartStatus_t
   • writeByteAsync( const byte_t value, 
                     Callback_t sucess,
                     Callback_t error 
                   )
\end{verbatim}

Métodos de lectura de \emph{String}:

\begin{verbatim}
   • readString( inOut String_t data,
                 inOut size_t dataSize,
                 String_t terminator
               )
   • readStringSync( inOut String_t data,
                     inOut size_t dataSize,
                     String_t terminator,
                     Time_t timout 
                   ) : UartStatus_t
   • readStringAsync( inOut String_t data,
                      inOut size_t dataSize,
   	                  String_t terminator,
                      Callback_t sucess, 
                      Callback_t error 
                    ) 
\end{verbatim}

Métodos de escritura de \emph{String}:                  
                
\begin{verbatim}  
   • writeString( const String_t data
                  inOut size_t dataSize
                )
   • writeStringSync( const String_t data,
                      inOut size_t dataSize,
                      Time_t timeout
                    ) : UartStatus_t
   • writeStringAsync( const String_t value, 
                      inOut size_t dataSize,
                      Callback_t sucess,
                      Callback_t error
                     )
\end{verbatim}

Métodos de lectura de arreglo de bytes:

\begin{verbatim}
   • readByteArray( inOut ByteArray_t data,
                    inOut size_t dataSize
                  )
\end{verbatim}

\pagebreak

\begin{verbatim}
   • readByteArraySync( inOut ByteArray_t data,
                        inOut size_t dataSize,
                        Time_t timeout
                      ) : UartStatus_t
   • readByteArrayAsync( inOut ByteArray_t data, 
                         inOut size_t dataSize,
                         Callback_t sucess,
                         Callback_t error 
                       )
\end{verbatim}

Métodos de escritura de arreglo de bytes:

\begin{verbatim}
  • writeByteArray( const ByteArray_t data,
                    inOut size_t dataSize
                  )
  • writeByteArraySync( const ByteArray_t data,
                        inOut size_t dataSize,
                        Time_t timeout
                      ) : UartStatus_t
  • writeByteArrayAsync( const ByteArray_t data, 
                         inOut size_t dataSize,
                         Callback_t sucess,
                         Callback_t error
                       )
\end{verbatim}

En todos los métodos de lectura se le pasa como primer parámetro la variable donde escribirá los datos recibidos. En la lectura de \emph{String} se le debe pasar un parámetro \texttt{terminator} que es un \texttt{String} que indica como debe ser el fin de cadena a buscar. Cuando se lee un \emph{String} o \emph{Byte Array} se almacena la cantidad de datos leídos en la variable \emph{dataSize} que se le pasa como parámetro.

%------------------------------------------------------------------------------
\subsection{Módulo SPI}

Modela un bus de serie para interfaz con periféricos (bus SPI). El periférico SPI proporciona comunicación serial sincrónica \emph{full duplex} entre dispositivos maestros y esclavos. Se usa comúnmente para la comunicación con periféricos externos como memorias flash, sensores, relojes en tiempo real (RTC), etc.

\titulo{Propiedades de SPI}

\begin{itemize}
\item
\emph{mode}, del tipo \emph{SpiMode\_t}, que representa el modo de funcionamiento del periférico SPI, con valores \emph{MASTER} o \emph{SLAVE}.
\item
\emph{clockFrequency}, del tipo \emph{int\_t}, es la frecuencia de reloj (en Hz) del BUS SPI (solo tiene sentido en modo \emph{MASTER}).
\item
\emph{clockPolarity}, del tipo \emph{SpiClockParity\_t}, define el nivel cuando se considera activo el clock, con valores \emph{POLARITY\_LOW} o \emph{POLARITY\_HIGH}.
\item
\emph{clockPhase}, del tipo \emph{SpiClockPhase\_t}, establece en que flanco de reloj se muestrean los datos, cuyos valores posibles son \emph{PHASE\_EDGE1} o \emph{PHASE\_EDGE2}.
\item
\emph{dataBits}, del tipo \emph{SpiDataBits\_t}, que puede tomar los valores \emph{DATABITS8} o \emph{DATABITS16}.
\item
\emph{dataOrder}, del tipo \emph{SpiDataOrder\_t}, representa el orden con el que se mueven los bits en el registro de desplazamiento, es decir si se mueven primero el bit menos significativo o el más significativo: \emph{MSB\_FIRST} o \emph{LSB\_FIRST}.
\item
\emph{location}, del tipo \emph{Location\_t}.
\item
\emph{transmitValue}, del tipo \emph{ByteArray\_t}. Representa el valor a transmitir.
\item
\emph{receiveValue}, del tipo \emph{ByteArray\_t}. Representa el último valor recibido. 
\end{itemize}

\titulo{Métodos de SPI}

Inicialización y restablecimiento de un periférico SPI:

\begin{verbatim}
   • init()
   • init( SpiInit_t init )
   • init( SpiInit_t init, int_t clockFrequency )
   • deInit()
\end{verbatim}

Los valores posibles del parámetro \emph{init} se forman con valores de:

\begin{verbatim}
   • mode | clockPolarity | clockPhase | dataOrder |
     dataBits | location
\end{verbatim}

Los valores de configuración son opcionales, con valores por defecto:

\begin{verbatim}
   • mode = MASTER
   • clockFrequency = 100000
   • clockPolarity = POLARITY_HIGH
   • clockPhase = PHASE_EDGE1
   • dataBits = DATABITS8
   • dataOrder MSB_FIRST
   • location = LOCATION0
\end{verbatim}

En caso de seleccionar modo \emph{MASTER} se debe configurar la frecuencia de reloj y al menos un pin como selector de esclavo. La frecuencia de reloj puede configurarse usando el método \emph{init()} de 2 parámetros, o bien, mediante:

\begin{verbatim}
   • clockFrequencySet( int_t clockFrequency )
\end{verbatim}

Utilizando \emph{initSSPin()} se configura un pin como selector de esclavo:

\begin{verbatim}
   • initSSPin( Pin_t pin )
\end{verbatim}

Luego para elegir que esclavo habilitar o deshabilitar se utiliza:

\begin{verbatim}
   • slaveSelect( Pin_t pin, bool_t enable )
\end{verbatim}

Permite realizar envío, recepción o transferencia de datos (que envía y recibe simultáneamente). Estas operaciones pueden realizarse por encuesta o interrupción con los siguientes métodos:

\begin{verbatim}
   • status() : SpiStatus_t
   • send( const ByteArray_t data )
   • receive( inOut ByteArray_t receive )
   • transfer( const ByteArray_t send, 
               inOut ByteArray_t receive
             ) : SpiStatus_t
   • eventCallbackSet( ChannelEvent_t evt, Callback_t func )
   • eventCallbackClear( ChannelEvent_t evt )
   • interrupt( bool_t enable )
\end{verbatim}

Los posibles valores de \emph{SpiStatus\_t} son:

\begin{verbatim}
   • READY
   • BUSY
   • TRANSFER_COMPLETE
   • ERROR
   • ERROR_TIMEOUT
\end{verbatim}

Eventos:

\begin{verbatim}
   • TRANSFER_COMPLETE
   • TRANSFER_ERROR
\end{verbatim}

\titulo{Métodos de SPI de alto nivel}

% Los métodos de escritura y lectura de a un byte no tienen sentido pues si el tamaño de datos no es 8, entonces siempre es byte array.

Lectura de arreglo de bytes:

\begin{verbatim}
   • readByteArray( inOut ByteArray_t data,
                    inOut size_t dataSize
                  )
   • readByteArraySync( inOut ByteArray_t data,
                        inOut size_t dataSize,
                        Time_t timout
                      ) : SpiStatus_t
   • readByteArrayAsync( inOut ByteArray_t data, 
                         inOut size_t dataSize,
                         Callback_t sucess,
                         Callback_t error 
                       )
\end{verbatim}

Escritura de arreglo de bytes:

\begin{verbatim}
   • writeByteArray( const ByteArray_t data,
                     inOut size_t dataSize
                   )
   • writeByteArraySync( const ByteArray_t data,
                         inOut size_t dataSize,
                         Time_t timout
                       ) : SpiStatus_t
   • writeByteArrayAsync( const ByteArray_t data, 
                          inOut size_t dataSize,
                          Callback_t sucess,
                          Callback_t error 
                        )
\end{verbatim}

Transferencia de arreglo de bytes:

\begin{verbatim}
   • transferByteArray( const ByteArray_t send,
                        inOut size_t sendSize,
                        inOut ByteArray_t receive,
                        inOut size_t dataSize
                      )
   • transferByteArraySync( const ByteArray_t send,
                            inOut size_t sendSize,
                            inOut ByteArray_t receive,
                            inOut size_t dataSize
                          ) : SpiStatus_t
   • transferByteArrayAsync( const ByteArray_t send,
                             inOut size_t sendSize,
                             inOut ByteArray_t receive,
                             Callback_t sucess,
                             Callback_t error 
                           )
\end{verbatim}

% Para la implementación:

% https://github.com/epernia/sAPI/blob/develop/toMergeIn_sapi1Project/sapi_soc_peripherals/inc/sapi_spi.h

%  https://github.com/epernia/sAPI/blob/develop/toMergeIn_sapi1Project/sapi_soc_peripherals/src/sapi_spi.c

%https://www.mikroe.com/ebooks/programming-dspic-microcontrollers-in-pascal/master-and-slave-modes

%------------------------------------------------------------------------------
\subsection{Módulo I2C}

I2C: modela un periférico de comunicación serie I2C.

Events:
I2C


%------------------------------------------------------------------------------
\subsection{Módulo RTC}

El módulo RTC modela un periférico Reloj de Tiempo Real. Este periférico permite mantener la hora y fecha en un sistema.

\titulo{Propiedades de RTC}

\begin{itemize}
\item
\emph{dateAndTime}, contiene la fecha/hora actual, del tipo \emph{DateAndTime\_t}.
\item
\emph{alarm}, establece la fecha/hora para un evento de alarma, del tipo \emph{DateAndTime\_t}.
\item
\emph{interval}, del tipo \emph{Time\_t}, establece el intervalo de tiempo para un evento periódico.
\end{itemize}

\titulo{Métodos de RTC}

Inicialización y restablecimiento de un periférico RTC:

\begin{verbatim}
   • init( DateAndTime_t absTime | )
   • deinit()
\end{verbatim}

El parámetro \emph{absTime} establece la fecha/hora actual. La lectura del RTC se realiza con el método \emph{read()}, que retorna la fecha/hora actual. Para cambiar la fecha/hora se utiliza el método \emph{write()}:

\begin{verbatim}
   • read() : DateAndTime_t
   • write( DateAndTime_t absTime )
\end{verbatim}

Para establecer una interrupción en un RTC ante cierto evento define los métodos: 

\begin{verbatim}
   • interrupt( bool_t enable )
   • eventCallbackSet( ChannelEvent_t evt, Callback_t c )
   • eventCallbackClear( ChannelEvent_t evt )
\end{verbatim}

Los eventos posibles son: \emph{ALARM}, que genera una única interrupción cuando se cumple cierto valor absoluto de fecha/hora, y \emph{PERIODIC}, que interrumpe de forma periódica cada cierto lapso de tiempo.

%------------------------------------------------------------------------------
\subsection{Módulo Timer}
Duración: TIME.

323 Mazidi

TIMER: modela un periférico Timer/Counter.

Time/Counterl Capture Event OOOC
Time/Counterl Compare Match A OOOE
Time/Counterl Compare Match B 001 0
Time/Counterl Overflow 001 2


\emph{Tick\_t}, representa un valor de conteo de un lapso de tiempo adimencional lapso de tiempo en ticks (se almacena internamente como \emph{uint64\_t}).

Como evento posible se define: \emph{CONVERSION\_COMPLETE} que genera una interrupción cuando se completa la conversión. El método \emph{interrupt()} habilita  o deshabilita todas las interrupciones del periférico.

% Para implementacion ver:

% https://github.com/epernia/sAPI/blob/develop/toMergeIn_sapi1Project/sapi_soc_peripherals/inc/sapi_timer.h

% la nueva sapi en la carpeta dev


%De notas.txt:
%
%
%Para TIMER en general:
%----------------------
%
%    Timer(4, prescaler=624, period=13439, mode=UP, div=1)
%
%The information means that this timer is set to run at the peripheral
%clock speed divided by 624+1, 
%
%it will count from 0 up to 13439, at which point it triggers an interrupt, 
%and then starts counting again from 0.
%
%These numbers are set to make the timer trigger at 10 Hz: the source frequency
%of the timer is 84MHz (found by running ``tim.source_freq()``) so we
%get 84MHz / 625 / 13440 = 10Hz.
%
%Timer counter
%-------------
%
%So what can we do with our timer?  The most basic thing is to get the
%current value of its counter::
%
%    >>> tim.counter()
%    21504
%
%This counter will continuously change, and counts up.
%
%---------------------------------------------------------
%
%
%Para TIMER PWM:
%---------------
%
%	enum POLARITY{ ACTIVE_LOW=0, ACTIVE_HIGH=1 };
%
%	float analogFrequency;  //defaults to 100,000 Hz
%	float analogMax;        //defaults to 3.3V
%
%	int setPeriod(unsigned int period_ns);
%	 unsigned int getPeriod();
%	 int setFrequency(float frequency_hz);
%	 float getFrequency();
%	 int setDutyCycle(unsigned int duration_ns);
%	 int setDutyCycle(float percentage);
%	 unsigned int getDutyCycle();
%	 float getDutyCyclePercent();
%
%	 int setPolarity(PWM::POLARITY);
%	 void invertPolarity();
%	 PWM::POLARITY getPolarity();
%
%	virtual void setAnalogFrequency(float frequency_hz) { this->analogFrequency = frequency_hz; }
%	virtual int calibrateAnalogMax(float analogMax); //must be between 3.2 and 3.4
%	virtual int analogWrite(float voltage);
%
%	virtual int run();
%	virtual bool isRunning();
%	virtual int stop();
%
%	virtual ~PWM();
%private:
%	float period_nsToFrequency(unsigned int);
%	unsigned int frequencyToPeriod_ns(float);
%};
%
%---------------------------------------------------------
%
%Para TIMER Waveform generation output signal:
%----------------------------------------------
%
%   // Generate a periodic sine wave in an array of 100 values - using ints
%   unsigned int waveform[100];
%   float gain = 50.0f;
%   float phase = 0.0f;
%   float bias = 50.0f;
%   float freq = 2.0f * 3.14159f / 100.0f;
%   for (i=0; i<100; i++){
%      waveform[i] = (unsigned int)(bias + (gain * sin((i * freq) + phase)));
%   }
%   
%   
%   
%tickConfig( TICK_MS(1) ); o tickStart( TICK_MS(1) );
%tickEnabeling( ENABLE );
%tickSetCallback( function );
%
%typedef tick_t uint64_t;
%
%#define TICK_MS(t) ((typedef)(t))
%#define TICK_S(t)  (((typedef)(t))*1000)





%------------------------------------------------------------------------------
\subsection{Módulo Core}

Events:
Reset
%void ResetISR(void);
%WEAK void NMI_Handler(void);
%WEAK void HardFault_Handler(void);
%WEAK void MemManage_Handler(void);
%WEAK void BusFault_Handler(void);
%WEAK void UsageFault_Handler(void);
%WEAK void SVC_Handler(void);
%WEAK void DebugMon_Handler(void);
%WEAK void PendSV_Handler(void);
%WEAK void SysTick_Handler(void);
%WEAK void IntDefaultHandler(void);


sleep(UntilNextInterrupt)?
sleep(UntilGPIOPinIsHigh)

%------------------------------------------------------------------------------
\subsection{Módulo SoC}


%------------------------------------------------------------------------------
\subsection{Módulo Board}



%void M0CORE_IRQHandler(void) ALIAS(IntDefaultHandler);
%
%void SCT_IRQHandler(void) ALIAS(IntDefaultHandler);
%void RIT_IRQHandler(void) ALIAS(IntDefaultHandler);
%
%void TIMER0_IRQHandler(void) ALIAS(IntDefaultHandler);
%void TIMER1_IRQHandler(void) ALIAS(IntDefaultHandler);
%void TIMER2_IRQHandler(void) ALIAS(IntDefaultHandler);
%void TIMER3_IRQHandler(void) ALIAS(IntDefaultHandler);
%
%void ADC0_IRQHandler(void) ALIAS(IntDefaultHandler);
%
%void I2C0_IRQHandler(void) ALIAS(IntDefaultHandler);
%
%void SPI_IRQHandler(void) ALIAS(IntDefaultHandler);
%
%void UART0_IRQHandler(void) ALIAS(IntDefaultHandler);
%void UART1_IRQHandler(void) ALIAS(IntDefaultHandler);
%void UART2_IRQHandler(void) ALIAS(IntDefaultHandler);
%void UART3_IRQHandler(void) ALIAS(IntDefaultHandler);
%
%void GPIO0_IRQHandler(void) ALIAS(IntDefaultHandler);
%void GPIO1_IRQHandler(void) ALIAS(IntDefaultHandler);
%void GPIO2_IRQHandler(void) ALIAS(IntDefaultHandler);
%void GPIO3_IRQHandler(void) ALIAS(IntDefaultHandler);
%void GPIO4_IRQHandler(void) ALIAS(IntDefaultHandler);
%void GPIO5_IRQHandler(void) ALIAS(IntDefaultHandler);
%void GPIO6_IRQHandler(void) ALIAS(IntDefaultHandler);
%void GPIO7_IRQHandler(void) ALIAS(IntDefaultHandler);
%void GINT0_IRQHandler(void) ALIAS(IntDefaultHandler);
%void GINT1_IRQHandler(void) ALIAS(IntDefaultHandler);


