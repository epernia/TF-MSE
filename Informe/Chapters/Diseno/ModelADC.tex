
%------------------------------------------------------------------------------
\subsection{Módulo ADC}

El módulo ADC modela un periférico Conversor Analógico-Digital. Este periférico contiene varios canales de conversión multiplexados los cuales se pueden configurar individualmente como entradas analógicas.

\titulo{Propiedades de ADC}

\begin{itemize}
\item
\emph{conversionRate}, del tipo \emph{uint32\_t}, representa la tasa de conversión en Hertz. Los valores posibles dependen de cada plataforma. Se pueden utilizar alternativamente los valores los valores del tipo \emph{AdcConvRate\_t} cuyos valores posibles son: \emph{LOW}, \emph{MID} y \emph{HIGH}.
\item
\emph{voltageReferece}, del tipo \emph{AdcVRef\_t} cuyos valores posibles son: \emph{VCC}, \emph{INTERNAL} y \emph{EXTERNAL}.
\item
\emph{conversionMode}, del tipo \emph{AdcConvMode\_t}, con valores: \emph{SINGLE}, \emph{CONTIUOUS} y \emph{EXTERNAL\_TRIGGERED}.
\item
\emph{conversionResolution}, del tipo \emph{AdcConvResolution\_t}, con valores:\emph{LOW}, \emph{MID} y \emph{HIGH}, que si bien dependen de la plataforma, ya están definidos para cada una de ellas.
\item
\emph{channelsMode}, del tipo \emph{AdcChannelsMode\_t}, con valores: \emph{SIGLE} y \emph{DIFFERENTIAL}.
\item
\emph{location}, del tipo \emph{AdcLocation\_t}. Son las posibles ubicaciones del periférico con respecto a pines físicos del SoC. Si bien dependen de la arquitectura, se definen los valores \emph{LOCATION0} a \emph{LOCATION7} para cada plataforma.
\item
\emph{value}, del tipo \emph{uint32\_t}. Representa el último valor de conversión del ADC.
\end{itemize}

\titulo{Métodos de ADC}

Inicialización y restablecimiento de un periférico ADC:

\begin{verbatim}
 • init( uint32_t conversionRate | 
         AdcVRef_t voltageReferece |
         AdcConvMode_t conversionMode |
         AdcConvResolution_t conversionResolution |
         AdcChannelsMode_t channelsMode |
         AdcLocation_t loacation
       ) : AdcStatus_t
 • deinit() : AdcStatus_t
\end{verbatim}

Todos los parámetros del método \emph{init()} son opcionales y si no se aplican se inicializa por defecto con:

\begin{verbatim}
 • conversionRate = HIGH
 • voltageReferece = VCC
 • conversionMode = SINGLE
 • conversionResolution = HIGH
 • channelMode = SINGLE
 • loacation = DEFAULT
\end{verbatim}

Habilitación/deshabilitación de un canal particular del ADC (configuración de una entrada analógica):

\begin{verbatim}
 • channel( AdcChannel_t channel, bool_t enable ) : AdcStatus_t
\end{verbatim}

Define los métodos \emph{currentChannel()}, que devuelve el canal de la conversión actual, y \emph{channels()} que devuelve la lista de canales del ADC. Notar que esta lista se reduce a la mitad si el modo \emph{channelsMode} es \emph{DIFFERENTIAL)}

\begin{verbatim}
 • currentChannel() : AdcChannel_t
 • channels() : List_t
\end{verbatim}

La lectura del conversor ADC se realiza con el método \emph{read()}, que requiere un parámetro \emph{channel)}:

\begin{verbatim}
 • read( AdcChannel_t channel ) : uint32_tt
\end{verbatim}

Si el ADC tiene configurado \emph{conversionMode = SINGLE}, entonces la lectura es bloqueante pues debe esperar a realizar la conversión. Si el ADC está en modo \emph{conversionMode = CONTIUOUS}, entonces devuelve el último valor convertido. Alternativamente existe el método \emph{startConversion()} para realizar una lectura no bloqueante, que se puede utilizar junto a \emph{isConversionComplete()} para encuestar si termino la conversión, o mediante un evento.

\begin{verbatim}
 • startConversion( AdcChannel_t channel ) : uint32_t
 • isConversionComplete() : bool_t
\end{verbatim}

El módulo ADC no define el método \emph{write()}. Para establecer una interrupción en un canal analógico ante cierto evento define los métodos: 

\begin{verbatim}
 • interrupt( bool_t enable )
 • eventCallbackSet( ChannelEvent_t evt, Callback_t c )
 • eventCallbackClear( ChannelEvent_t evt )
\end{verbatim}

Como evento posible se define: \emph{CONVERSION\_COMPLETE} que genera una interrupción cuando se completa la conversión.
% Ver de agregar Analog Comparator como evento
