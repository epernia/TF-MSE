%------------------------------------------------------------------------------
\subsection{Módulo ADC}

El módulo ADC modela un periférico Conversor Analógico-Digital. Este periférico contiene varios canales de conversión multiplexados los cuales se pueden configurar individualmente como entradas analógicas.

\titulo{Propiedades de ADC}

\begin{itemize}
\item
\emph{conversionRate}, del tipo \emph{int\_t}, representa la tasa de conversión en Hertz. Los valores posibles dependen de cada plataforma. Se pueden utilizar alternativamente los valores genéricos: \emph{LOW}, \emph{MEDIUM}, \emph{HIGH} y \emph{VERY\_HIGH}.
\item
\emph{conversionMode}, del tipo \emph{AdcConvMode\_t}, con valores: \emph{SINGLE} (activado por software desde el programa de usuario), \emph{CONTINUOUS} (conversion periódica disparada por hardware a tasa \emph{conversionRate}).% y \emph{EXTERNAL\_TRIGGERED} (Timer Match signal o GPIO).
\item
\emph{voltageReferece}, del tipo \emph{AdcVRef\_t} cuyos valores posibles son: \emph{VCC}, \emph{INTERNAL} y \emph{EXTERNAL}.
\item
\emph{resolution}, del tipo \emph{AdcRes\_t}, cuyos valores también dependen de la plataforma pero se pueden utilizar en su lugar los valores genéricos.
\item
\emph{channelsMode}, del tipo \emph{AdcChannelsMode\_t}, con valores: \emph{SIGLE} y \emph{DIFFERENTIAL}.
\item
\emph{location}, del tipo \emph{Location\_t}. Son las posibles ubicaciones del periférico con respecto a pines físicos del SoC. Si bien dependen de la arquitectura, se definen los valores genéricos \emph{LOCATION0} a \emph{LOCATION7} para cada plataforma.
\end{itemize}

\titulo{Métodos de ADC}

Inicialización y restablecimiento de un periférico ADC:

\begin{verbatim}
   • init()
   • init( int_t conversionRate, AdcInit_t init )
   • deInit()
\end{verbatim}

Los valores posibles del parámetro \emph{init} se forman con valores de:

\begin{verbatim}
   • conversionMode | voltageReferece | resolution | 
     channelsMode | location
\end{verbatim}

Todos los valores de configuración del método \emph{init()} son opcionales y los que no se aplican se inicializan por defecto con:

\begin{verbatim}
   • conversionRate = VERY_HIGH
   • voltageReferece = VCC
   • conversionMode = SINGLE
   • resolution = VERY_HIGH
   • channelMode = SINGLE
   • loacation = LOCATION0
\end{verbatim}

Habilitación/deshabilitación de un canal particular del ADC (configuración de una entrada analógica):

\begin{verbatim}
   • channel( AdcChannel_t channel, bool_t enable )
\end{verbatim}

Para conocer los canales disponibles de un ADC existe el método 
\emph{channelsGet()} que devuelve la lista de canales del ADC. Esta lista se reduce a la mitad si el modo \emph{channelsMode} es \emph{DIFFERENTIAL)}

\begin{verbatim}
   • channelsGet() : List_t
\end{verbatim}

Si el ADC tiene configurado \emph{conversionMode = CONTINUOUS} muestrea automáticamente los canales habilitados. El método\emph{startConversion()} comienza el muestreo automático, para parar el muestreo se utiliza \emph{stopConversion()}. Si en cambio \emph{conversionMode = SINGLE}, la lectura se debe lanzar mediante el método \emph{startConversion()} que recibe un canal como parámetro. 
Una conversión se puede abortar con \emph{stopConversion()}.

\begin{verbatim}
   • startConversion( AdcChannel_t channel )
   • stopConversion()
\end{verbatim}

Existen 2 formas de leer los valores convertidos del ADC, por encuesta, o por interrupción. En el primer caso se debe chequear si finalizó la conversión con el método \emph{status()}, que retorna un un valor del tipo \emph{AdcStatus\_t}. Si se realizó la conversión correctamente se puede leer el valor convertido con el método \emph{read()}.

\begin{verbatim}
   • status() : AdcStatus_t
   • read() : AdcValue_t
\end{verbatim}

Los valores posibles de \emph{AdcStatus\_t} son: 

%\emph{CONVERTING}, \emph{CONVERSION\_COMPLETE}, \emph{CONVERSION\_COMPLETE\_OVERRUN}, \emph{CONVERSION\_ADC\_ERROR}, \emph{CONVERSION- \_TIMEOUT}. 

\begin{verbatim}
   • READY
   • BUSY
   • CONVERSION_COMPLETE
   • ERROR
   • ERROR_TIMEOUT
\end{verbatim}

Para leer un canal por interrupción se debe primero establecer una función de \emph{callback} al evento \emph{CONVERSION\_COMPLETE}, otra al evento \emph{ERROR} y luego activar la interrupción del periférico. Esto se realiza con los métodos:

\begin{verbatim}
   • eventCallbackSet( ChannelEvent_t evt, Callback_t func )
   • eventCallbackClear( ChannelEvent_t evt )
   • interrupt( bool_t enable )
\end{verbatim}

Los eventos del ADC son del tipo \emph{AdcEvent\_t}. Al completar la conversión se ejecutará una de las dos funciones.

\titulo{Métodos de Canal de ADC}

Como alternativa, existen métodos que aplican a cierto canal de ADC y proveen un mayor nivel de abstracción:

\begin{verbatim}
   • read() : AdcValue_t
   • readSync( AdcValue_t data, Time_t timeout )
               : AdcStatus_t
   • readAsync( Callback_t sucess, Callback_t error )
\end{verbatim}

Tanto \emph{read()} como \emph{readSync()} realizan una lectura bloqueante, en consecuencia, es necesario establecer un tiempo de \textit{time out} para que no bloquee indefinidamente. \emph{read()} en fija automáticamente el \emph{timeout} al doble del \emph{conversionRate}, siempre retorna un valor, que a veces puede ser inválido. \emph{readSync()} devuelve el estado de conversión, para que el usuario chequee si valor convertido es válido, en ese caso lo carga en la variable \emph{data}, que recibe como parámetro. Si se establece en 0 el valor de \emph{timeout} en el método con sufijo \emph{Sync} entonces el tiempo de \texttt{timeout} será considerado infinito. Lo mismo sucede con todos los métodos que cuenten con dicho sufijo que se describen que se describen en esta sección.

\emph{readAsync()} realiza una lectura no bloqueante. Recibe como parámetros dos funciones de \emph{callback}, una que lanza en caso de conversión exitosa (si esto ocurre le pasa como parámetro el valor convertido) y otra en caso de error (le pasa el estado de conversión).

Este módulo no define método \emph{write()}.

% Ver de agregar Analog Comparator como evento: Comparación del valor convertido contra un valor programado, para mayor que, igual que o menor que.
%    ADC_GREATER_THAN: Modo comparación por mayor que.
%    ADC_LESS_THAN: Modo comparación por menor que.
