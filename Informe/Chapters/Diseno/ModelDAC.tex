%------------------------------------------------------------------------------
\subsection{Módulo DAC}

El módulo DAC modela un periférico Conversor Digital-Analógico. Este periférico comparte muchas características con el ADC, contando también con uno o más canales de conversión, los cuales se pueden configurar individualmente como salidas analógicas.

\titulo{Propiedades de DAC}

\begin{itemize}
\item
\emph{conversionRate}, del tipo \emph{int\_t}, representa la tasa de conversión en Hertz.
\item
\emph{conversionMode}, del tipo \emph{DacConvMode\_t}.
\item
\emph{voltageReferece}, del tipo \emph{DacVRef\_t}.
\item
\emph{resolution}, del tipo \emph{DacRes\_t}.
\item
\emph{location}, del tipo \emph{DaLocation\_t}.
\end{itemize}

\titulo{Métodos de DAC}

Inicialización y restablecimiento de un periférico DAC:

\begin{verbatim}
   • init()
   • init( int_t conversionRate, DacInit_t init )
   • deInit()
\end{verbatim}

Los valores posibles del parámetro \emph{init} se forman con valores de:

\begin{verbatim}
   • conversionMode | voltageReferece | resolution | location
\end{verbatim}

Todos los valores de inicialización son opcionales y si no se aplican se aplican utiliza por defecto:

\begin{verbatim}
   • conversionRate = VERY_HIGH
   • voltageReferece = VCC
   • conversionMode = SINGLE
   • resolution = VERY_HIGH
   • loacation = LOCATION0
\end{verbatim}

El resto de la API también es muy similar a la del ADC con la diferencia que define métodos para escribir el DAC en lugar de métodos para leer (no define el método \emph{read()}):

\begin{verbatim}
   • channel( DacChannel_t channel, bool_t enable )
   • channelsGet() : List_t
   • write( DacValue_t value )
   • startConversion( DacChannel_t channel )
   • stopConversion()
   • status() : DacStatus_t
   • eventCallbackSet( ChannelEvent_t evt, Callback_t func )
   • eventCallbackClear( ChannelEvent_t evt )
   • interrupt( bool_t enable )
\end{verbatim}

\titulo{Métodos de Canal de DAC}

\begin{verbatim}
   • write( DacValue_t value )
   • writeSync( DacValue_t value, Time_t timout ) 
                : DacStatus_t
   • writeAsync( DacValue_t value, 
                 Callback_t sucess, 
                 Callback_t error 
               )
\end{verbatim}

Se agrega el conjunto de métodos para la generación de señales:

\begin{verbatim}
   • waveGenSet( Array_t samples, Time_t periodicity )
   • waveGenStart()
   • waveGenStop()
\end{verbatim}
