%------------------------------------------------------------------------------
\subsection{Módulo DAC}

El módulo DAC modela un periférico Conversor Digital-Analógico. Este periférico comparte muchas características con el ADC, contando también con uno o más canales de conversión, los cuales se pueden configurar individualmente como salidas analógicas.

\titulo{Propiedades de DAC}

\begin{itemize}
\item
\emph{conversionRate}, del tipo \emph{uint32\_t}, representa la tasa de conversión en Hertz.
\item
\emph{conversionMode}, del tipo \emph{DacConvMode\_t}.
\item
\emph{voltageReferece}, del tipo \emph{DacVRef\_t}.
\item
\emph{resolution}, del tipo \emph{DacRes\_t}.
\item
\emph{location}, del tipo \emph{DaLocation\_t}.
\end{itemize}

\titulo{Métodos de DAC}

Inicialización y restablecimiento de un periférico DAC:

\begin{verbatim}
 • init( uint32_t conversionRate | 
         conversionMode | 
         voltageReferece | 
         resolution |
         location
       )
 • deInit()
\end{verbatim}

Todos los parámetros del método \emph{init()} son opcionales y si no se aplican se inicializa por defecto con:

\begin{verbatim}
 • conversionRate = VERY_HIGH
 • voltageReferece = VCC
 • conversionMode = SINGLE
 • resolution = VERY_HIGH
 • loacation = LOCATION0
\end{verbatim}

El resto de la API también es muy similar a la del ADC con la diferencia que en lugar de métodos para leer (no define el método \emph{read()}) define métodos para escribir el DAC:

\begin{verbatim}
 • channel( DacChannel_t channel, bool_t enable )
 • channelsGet() : List_t
 • write( uint32_t value )
 • startConversion( DacChannel_t channel )
 • stopConversion()
 • conversionStatus() : DacConvStatus_t
 • eventCallbackSet( ChannelEvent_t evt, Callback_t c )
 • eventCallbackClear( ChannelEvent_t evt )
 • interrupt( bool_t enable )
\end{verbatim}

\titulo{Métodos de Canal de DAC}

\begin{verbatim}
 • write( uint32_t value )
 • writeSync( uint32_t value, Time_ms_t timout ) : DacConvStatus_t
 • writeAsync( uint32_t value, 
               Callback_t sucess, 
               Callback_t error )
\end{verbatim}

Se agrega el conjunto de métodos para la generación de señales:

\begin{verbatim}
 • waveGenSet( Array_t samples, Time_ms_t periodicity )
 • waveGenStart()
 • waveGenStop()
\end{verbatim}

