%------------------------------------------------------------------------------
\subsection{Módulo DAC}

El módulo DAC modela un periférico Conversor Digital-Analógico. Este periférico comparte muchas características con el ADC, contando también con uno o más canales de conversión, los cuales se pueden configurar individualmente como salidas analógicas.

\titulo{Propiedades de DAC}

\begin{itemize}
\item
\emph{conversionRate}, del tipo \emph{uint32\_t}, representa la tasa de conversión en Hertz.
\item
\emph{voltageReferece}, del tipo \emph{DacVRef\_t} cuyos valores posibles son: \emph{VCC}, \emph{INTERNAL} y \emph{EXTERNAL}.
\item
\emph{conversionResolution}, del tipo \emph{DacConvResolution\_t}, con valores:\emph{LOW}, \emph{MID} y \emph{HIGH}.
\item
\emph{location}, del tipo \emph{AdcLocation\_t}. Con valores los valores \emph{LOCATION0} a \emph{LOCATION7}.
\item
\emph{value}, del tipo \emph{uint32\_t}. Representa el último valor de conversión del DAC.
\end{itemize}

\titulo{Métodos de DAC}

Inicialización y restablecimiento de un periférico DAC:

\begin{verbatim}
 • init( uint32_t conversionRate | 
         DacVRef_t voltageReferece |
         DacConvResolution_t conversionResolution |
       ) : DacStatus_t
 • deinit() : DacStatus_t
\end{verbatim}

Todos los parámetros del método \emph{init()} son opcionales y si no se aplican se inicializa por defecto con:

\begin{verbatim}
 • conversionRate = HIGH
 • voltageReferece = VCC
 • conversionResolution = HIGH
\end{verbatim}

Habilitación/deshabilitación de un canal particular del DAC (configuración de una salida analógica):

\begin{verbatim}
 • channel( DacChannel_t channel, bool_t enable ) : DacStatus_t
\end{verbatim}

Define los métodos \emph{currentChannel()}, que devuelve el canal de la conversión actual, y \emph{channels()} que devuelve la lista de canales del DAC.

\begin{verbatim}
 • currentChannel() : DacChannel_t
 • channels() : List_t
\end{verbatim}

El módulo DAC no define el método \emph{read()}. La escritura del DAC se realiza con el método \emph{write()} que requiere un parámetro \emph{channel)}

\begin{verbatim}
 • write( DacChannel_t channel ) : uint32_t
\end{verbatim}

La escritura es bloqueante pues debe esperar a que se termine una conversión previa que se esté llevando a cabo antes de comenzar con la actual. Al igual que el ADC, para realizar una escritura no bloqueante existen el método \emph{startConversion()}, que se puede utilizar junto a \emph{isConversionComplete()} para encuestar si termino la conversión, o mediante un evento con los mismos métodos que ADC.

\begin{verbatim}
 • startConversion( DacChannel_t channel ) : uint32_t
 • isConversionComplete() : bool_t
 • interrupt( bool_t enable )
 • eventCallbackSet( ChannelEvent_t evt, Callback_t c )
 • eventCallbackClear( ChannelEvent_t evt )
\end{verbatim}
