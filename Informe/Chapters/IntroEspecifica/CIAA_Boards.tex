%------------------------------------------------------------------------------
%	SECTION
%------------------------------------------------------------------------------
\section{Plataformas del Proyecto CIAA}
\label{sec:ciaaBoards}



En sus comienzos el proyecto CIAA desarrolla una computadora industrial basada en un microcontrolador NXP LPC4337 (JDB 144) Dual-core Cortex-M4+Cortex-M0 a 204MHz, con 1 MB de memoria Flash y 136 KB de memoria SRAM nombrada CIAA-NXP [ ]. En base a este diseño los integrantes del proyecto CIAA diseñan una versión educativa sin las protecciones industriales, nombrada EDU-CIAA-NXP [ ]. En la figura [ ] se presenta esta plataforma.

Esta plataforma incluye todos los periféricos típicos que podemos encontrar en los microcontroladores disponibles en el mercado permitiendo la enseñanza con herramientas modernas.

Mediante la colaboración de la Red Universitaria de Sistemas Embebidos (RUSE) [ ] se han distribuido en 2015 entre 10 y 40 placas en Universidades de Argentina con carreras afines a la electrónica.

Utilizando la EDU-CIAA-NXP como plataforma base para la enseñanza se ha desarrollado, o colaborado en el desarrollo de diferentes herramientas de software y hardware abiertas, las cuales se describen en las siguientes secciones.

--------------------------------------


El proyecto de la Computadora Industrial Abierta Argentina (CIAA) nació en 2013 como una iniciativa conjunta entre el sector académico y el industrial, representados por la ACSE\footnote{Asociación Civil para la investigación, promoción y desarrollo de los Sistemas electrónicos Embebidos. Sitio web: \url{http://www.sase.com.ar/asociacion-civil-sistemas-embebidos}} y CADIEEL\footnote{Cámara Argentina de Industrias Electrónicas, Electromecánicas y Luminotécnicas. Sitio web: \url{http://www.cadieel.org.ar/}}, respectivamente.

\medskip

\noindent Los objetivos del proyecto CIAA son:

\begin{itemize}
\item
Impulsar el desarrollo tecnológico nacional, a partir de sumar valor agregado al trabajo y a los productos y servicios, mediante el uso de sistemas electrónicos, en el marco de la vinculación de las instituciones educativas y el sistema científico-tecnológico con la industria.
\item
Darle visibilidad positiva a la electrónica argentina.
\item
Generar cambios estructurales en la forma en la que se desarrollan y utilizan en nuestro país los conocimientos en el ámbito de la electrónica y de las instituciones y empresas que hacen uso de ella.
\end{itemize}

\noindent Todo esto en el marco de un trabajo libre, colaborativo y articulado entre industria y academia. 

\bigskip

Con esta iniciativa, se han desarrollado en la actualidad varias plataformas de hardware y entornos de programación para utilizarlas.

\medskip

Al momento de la presentación de este trabajo, existen dos versiones de la plataforma CIAA cuyo desarrollo ha sido completado:

\begin{itemize}
\item 
CIAA-NXP, basada en el microcontrolador NXP LPC4337, que ya se comercializa.
\item 
CIAA-FSL, que utiliza, en cambio, el microcontrolador Freescale MK60FX512VLQ15, pero únicamente hay prototipos de esta plataforma. 
\end{itemize}

Además, existe una versión educativa de bajo costo de la CIAA-NXP, nombrada EDU-CIAA-NXP, que ya se distribuyeron alrededor de 1000 unidades y ya hay otras 1000 reservadas en producción. 

\medskip

Debido a estas razones, el trabajo se enfoca en el desarrollo de herramientas para programar las dos plataformas basadas en el microcontrolador NXP LPC4337. Se introducen a continuación las características de las mismas.




Siendo el autor participante de este proyecto desde mediados de 2014, ocupando el rol de Responsable de Software-PLC mediante el aporte al proyecto CIAA de un IDE\footnote{IDE4PLC. Sitio web: \url{http://proyecto-ciaa.com.ar/devwiki/doku.php?id=desarrollo:software-plc}} que permite programar esta plataforma con lenguajes de PLC industriales (IEC-661131-3), se desea agregar en esta oportunidad la posibilidad de programar a esta plataforma con un lenguaje de programación orientado a objetos mediante el desarrollo de un IDE para tal fin.


\subsection{CIAA-NXP}

La CIAA-NXP es la primera y única computadora del mundo que reúne dos cualidades:

\begin{itemize}
\item 
Ser \textbf{Industrial}, ya que su diseño está preparado para las exigencias de confiabilidad, temperatura, vibraciones, ruido electromagnético, tensiones, cortocircuitos, etc., que demandan los productos y procesos industriales.
\item 
Ser \textbf{Abierta}, ya que toda la información sobre su diseño de hardware, firmware, software, etc. está libremente disponible en Internet bajo la Licencia BSD, para que cualquiera la utilice como quiera.
\end{itemize}

\noindent Esta plataforma se compone de:

\begin{itemize}
\item
CPU: Microcontrolador NXP LPC 4337 JDB 144 (Dual-core Cortex-M4 + Cortex-M0 @ 204MHz).
\item
Debugger: USB-to-JTAG FT2232H. Soportado por OpenOCD.
\item
Memorias: 
   \begin{itemize}
   \item
   IS42S16400F - SDRAM. 64Mbit @ 143MHz.
   \item
   S25FL032P0XMFI011 - Flash SPI. 32 Mbit, Quad I/O Fast read: 80 MHz.
   \item
   24AA1025 - EEPROM I2C. 1 Mbit, 400 kHz. Almacenamiento de propósito general, datos de calibración del usuario, etc.
   \item
   24AA025E48 - EEPROM I2C. 2 kbit, 400 kHz. Para implementación de MAC-Address o almacenamiento de propósito general.
   \end{itemize}
\item
Entradas y salidas:   
   \begin{itemize}
   \item
   8 entradas digitales opto-aisladas 24VDC.
   \item
   4 Entradas analógicas 0-10V/4-20mA.
   \item
   4 salidas Open-Drain 24VDC.
   \item
   4 Salidas con Relay DPDT.
   \item
   1 Salida analógica 0-10V/4-20mA.
   \end{itemize}
\item
LV-GPIO:
   \begin{itemize}
   \item
   14 GPIOs.
   \item
   I2C.
   \item
   SPI.
   \item
   4 canales analógicos.
   \item
   Aux. USB.
   \end{itemize}
\item
Interfaces de comunicación:
   \begin{itemize}
   \item
   Ethernet.
   \item
   USB On-The-Go.
   \item
   RS232.
   \item
   RS485.
   \item
   CAN.
   \end{itemize}
\item
Múltiples fuentes de alimentación.

\end{itemize}

\noindent En la figura [\ref{fig:ciaaNxp}] se muestra una fotografía de la plataforma.

\begin{figure}[!htbp]
\begin{center}  %[width=14cm,height=8cm]
\includegraphics*[width=\textwidth]{Figures/CIAA-NXP_Foto.png}
\par\caption{Plataforma CIAA-NXP.}\label{fig:ciaaNxp}
\end{center}
\end{figure}


%http://proyecto-ciaa.com.ar/devwiki/doku.php?id=desarrollo:hardware:ciaa_nxp:ciaa_nxp_inicio

\subsection{EDU-CIAA-NXP}

La plataforma EDU-CIAA-NXP es un desarrollo colaborativo, realizado por miembros de la Red Universitaria de Sistemas Embebidos (RUSE), en el marco del Proyecto CIAA. RUSE se compone de docentes pertenecientes a más de 60 Universidades a lo largo y a lo ancho del país. 

\medskip

\noindent Los propósitos de la plataforma son: 

\begin{itemize}
\item
Proveer una plataforma de desarrollo moderna, económica y de fabricación nacional basada en la CIAA-NXP, que sirva a docentes y a estudiantes en los cursos de sistemas embebidos.
\item
Lograr una amplia inserción en el sistema educativo argentino.
\item
Realizar un aporte eficaz al desarrollo de vocaciones tempranas en electrónica, computación e informática.
\item
Demostrar que las universidades argentinas son capaces de realizar un desarrollo colaborativo exitoso en el área de los sistemas embebidos, cumpliendo con requerimientos de tiempo y forma.
\end{itemize}

\medskip

\noindent Características de la EDU-CIAA-NXP:

\begin{itemize}
\item
CPU: Microcontrolador NXP LPC 4337 JDB 144 (Dual-core Cortex-M4 + Cortex-M0 @ 204MHz).
\item
Debugger: USB-to-JTAG FT2232H. Soportado por OpenOCD.
\item
2 puertos micro-USB (uno para aplicaciones y debug, otro OTG).
\item
6 salidas digitales implementadas con leds (3 normales y uno RGB).
\item
4 entradas digitales con pulsadores.
\item
1 puerto de comunicaciones RS-485 con bornera.
\item
2 conectores de expasión:
   \begin{itemize}
   \item
   P0:
      \begin{itemize}
      \item
      3 entradas analógicas (ADC0 a ADC2).
      \item
      1 salida analógica (DAC0).
      \item
      1 conexión para un teclado de 3 x 4.
      \item
      12 pines genéricos de I/O.
      \end{itemize}
   \item
   P1:
      \begin{itemize}
      \item
      1 puerto Ethernet.
      \item
      1 puerto CAN.
      \item
      1 puerto SPI.
      \item
      1 puerto I2C.
      \item
      12 pines genéricos de I/O.
      \end{itemize}
   \end{itemize}
\end{itemize}

\noindent En la figura [\ref{fig:eduCiaa}] se muestra una fotografía de esta plataforma.

\begin{figure}[!htbp]
\begin{center}  %[width=14cm,height=8cm]
\includegraphics*[width=\textwidth]{Figures/EDU-CIAA-NXP_Foto.png}
\par\caption{Plataforma EDU-CIAA-NXP.}\label{fig:eduCiaa}
\end{center}
\end{figure}

\medskip 

%http://proyecto-ciaa.com.ar/devwiki/doku.php?id=desarrollo:edu-ciaa:edu-ciaa-nxp

% http://proyecto-ciaa.com.ar/devwiki/doku.php?id=start

\subsection{Pico-CIAA}

http://www.proyecto-ciaa.com.ar/devwiki/doku.php?id=desarrollo:hardware:picociaa

Procesador Dual-Core de 32 bits

Tiene un procesador Dual Core de 32-bit ARM Cortex-M4F / Cortex-M0+ @100 MHz con 512kB flash y 104 kB SRAM, incluye debug vía un LPC11U35 programable.

Diagrama en bloques

Posee diversos puertos de comunicación, entre ellos USB, PCIExpress, UART, SPI, I2C y soporte para PWM y entradas y salidas digitales de propósito general.
special

Es la CIAA más pequeña

Con sólo 51 x 30 mm la picoCIAA es la placa más pequeña de la familia CIAA,
lo que la hace ideal para aplicaciones de internet de las cosas (IoT).

Ideal en Single Board Computers

Su interfaz PCI Express permite integrarla fácilmente en SBC de propósito general, para dar soporte de bajo nivel al manejo de dispositivos en tiempo real.

\begin{figure}[!htbp]
\begin{center}  %[width=14cm,height=8cm]
\includegraphics*[width=\textwidth]{Figures/PicoCIAA_Foto.png}
\par\caption{Plataforma PicoCIAA.}\label{fig:picoCiaa}
\end{center}
\end{figure}

\subsection{CIAA-Z3R0}

Esta plataforma es ideal para aplicaciones de bajo consumo y proyectos de robótica educativa. Está diseñada para ser integrada como componente en un diseño mayor debido a su reducido tamaño conectándola mediante tiras de pines, o soldada a través de su borde de agujeros para montaje castellated.

Se puede comprar en Argentina por aproximadamente AR \$400 a inicios de 2018.

Posee la mayoría de los periféricos que se encuentran en la EDU-CIAA-NXP pero utiliza un microcontrolador siete veces más económico que esta última, de la empresa Silicon Labs, modelo EFM32HG322F64 (QFP48) con núcleo ARM Cortex-M0+ a 25MHz, 64KB de memoria Flash y 8KB de memoria SRAM, que es suficiente para que un alumno entre en el mundo de los microcontroladores modernos de 32 bits.

El dispositivo de depuración se debe comprar por separado, sin embargo, mediante un único dispositivo de depuración se pueden programar muchas plataformas CIAAZ3R0.
No es necesario dejar este circuito de depuración en el diseño final

\begin{figure}[!htbp]
\begin{center}  %[width=14cm,height=8cm]
\includegraphics*[width=\textwidth]{Figures/CIAA-Z3R0_Foto.png}
\par\caption{Plataforma CIAA-Z3R0.}\label{fig:ciaaZero}
\end{center}
\end{figure}

\subsection{Arquitectura de Hardware}

Saraza...

\subsection{Bibliotecas disponibles}

Saraza...
