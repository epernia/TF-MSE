%------------------------------------------------------------------------------
%	SECTION
%------------------------------------------------------------------------------
\section{Requerimientos}
\label{sec:requerimientos}

Los requerimientos se establecieron en base a los objetivos expuestos en la sección \ref{sec:objetivosAlance}, reuniones con desarrolladores del Proyecto CIAA y el director del presente trabajo. Además se tuvieron en cuenta las opiniones de alumnos de grado de UNQ, posgrado de FIUBA y cursos CAPSE. Estos son:

\begin{itemize} %[label={}]
\item REQ.1. Fecha de finalización: 19/11/2018.
\item REQ.2. Diseño de la biblioteca.
\begin{itemize}%[label*=\arabic*.]
\item REQ.2.1. Realizar un diseño independiente del hardware y lenguaje de programación, teniendo en cuenta los conceptos familiares al programador de Sistemas Embebidos.
\item REQ.2.2. Debe estar modelada con objetos y contar con una descripción mediante diagramas UML.
\item REQ.2.3. Modelar al menos: CORE, GPIO, ADC, DAC, TIMER, RTC, UART, SPI e I2C.
%\begin{enumerate}
%\item CORE: un núcleo de procesamiento.
%\item GPIO: periférico que consiste en un único pin de entrada/salida de propósito general (\emph{pin}), así como un grupo de pines (\emph{port}).
%\item ADC: periférico conversor analógico-digital.
%\item DAC: periférico conversor digital-analógico.
%\item TIMER: periférico temporizador.
%\item RTC: periférico reloj de tiempo real.
%\item UART: periférico de comunicación serial asincrónico.
%\item SPI: periférico interfaz serie sincrónica.
%\item I2C: periférico de comunicación serie entre circuitos integrados.
%\end{enumerate}
\end{itemize}
\item REQ.3. Implementación de la biblioteca.
\begin{itemize}
\item REQ.3.1. Utilizar un sistema de control de versiones \emph{on line}, desarrollar \emph{tests} unitarios y de integración.
\item REQ.3.2. Programar en lenguaje C la biblioteca para cada plataforma de hardware particular utilizando como plantilla los archivos generados.
\item REQ.3.3. Desarrollar ejemplos de utilización.
\end{itemize}
\item REQ.4. Documentación y difusión.
\begin{itemize}
\item REQ.4.1. Confeccionar un manual de referencia de la biblioteca.
\item REQ.4.2. Desarrollar un tutorial de instalación de las herramientas para utilizar la biblioteca.
\item REQ 4.3. Publicación \emph{on line} del código fuente.
\item REQ 4.4. Informar a la comunidad del Proyecto CIAA y a la comunidad de programadores de Sistemas Embebidos.
\end{itemize}
\end{itemize} %enumerate

% \subsection{Entregables principales del trabajo}
% 
% Se listan a continuación los entregables principales del proyecto:
% 
% 1. Archivos de descripción de la biblioteca mediante diferentes diagramas y código
% independiente del lenguaje de programación.
% 2. Código fuente de la implementación en lenguaje C de la biblioteca para las plataformas de hardware:
%    - CIAA-NXP.
%    - EDU-CIAA-NXP.
%    - CIAA-Z3R0.
%    - PicoCIAA.
% 3. Ejemplos de utilización.
% 4. Informe final.
% 5. Manual de referencia de la API de la biblioteca.
% 6. Tutorial de instalación de las herramientas para utilizar la biblioteca con las plataformas de hardware citadas.

