%------------------------------------------------------------------------------
%	SECTION
%------------------------------------------------------------------------------
\section{Requerimientos}
\label{sec:requerimientos}

%\subsection{Diseño de la biblioteca}

%\subsection{Implementación}

%\subsection{Documentación y difusión}

Los requerimientos se establecieron en base a los objetivos expuestos en la sección \ref{sec:objetivosAlance}, reuniones con desarrolladores del Proyecto CIAA, alumnos (de grado de UNQ, posgrado de FIUBA y cursos CAPSE) y el director. El espíritu de los mismos es proponer una solución que permita la estandarización de la programación de microcontroladores. Estos son:

\begin{enumerate}
   \item Fecha de finalización: 19/11/2018.
   \item Diseño de la biblioteca.
      \begin{enumerate}[1]
         \item Realizar un diseño independiente del hardware y lenguaje de programación, teniendo en cuenta los conceptos familiares al programador de Sistemas Embebidos.
         \item Debe estar modelada con objetos y contar con una descripción mediante diagramas UML.
         \item Debe modelar al menos:
            \begin{enumerate}[1]
               \item CORE: un núcleo de procesamiento.
               \item GPIO: periférico que consiste en un único pin de entrada/salida de propósito general (pin), así como un grupo de pines (port).
               \item ADC: periférico conversor analógico-digital.
               \item DAC: periférico conversor digital-analógico.
               \item TIMER: periférico temporizador.CORE: un núcleo de procesamiento.
               \item RTC: periférico reloj de tiempo real.CORE: un núcleo de procesamiento.
               \item UART: periférico de comunicación serial asincrónico.
               \item SPI: periférico interfaz serie sincrónica.
               \item I2C: periférico de comunicación serie entre circuitos integrados.
            \end{enumerate}
      \end{enumerate}
   \item Implementación de la biblioteca.
      \begin{enumerate}[1]
         \item Utilizar un sistema de control de versiones con repositorios on line.
         \item Programar en lenguaje C la biblioteca para cada plataforma de hardware particular utilizando como plantilla los archivos generados.
         \item Desarrollar ejemplos de utilización para las diferentes plataformas.
      \end{enumerate}
   \item Documentación y difusión.
      \begin{enumerate}[1]
         \item Confeccionar un manual de referencia de la API de la biblioteca.
         \item Desarrollar un tutorial de instalación de las herramientas para utilizar la biblioteca con las plataformas de hardware citadas.
         \item Publicación on line del código fuente.
         \item Informar a la comunidad del Proyecto CIAA y a la comunidad de programadores de
   Sistemas Embebidos.
      \end{enumerate}
\end{enumerate}

% \subsection{Entregables principales del trabajo}
% 
% Se listan a continuación los entregables principales del proyecto:
% 
% 1. Archivos de descripción de la biblioteca mediante diferentes diagramas y código
% independiente del lenguaje de programación.
% 2. Código fuente de la implementación en lenguaje C de la biblioteca para las plataformas de hardware:
%    - CIAA-NXP.
%    - EDU-CIAA-NXP.
%    - CIAA-Z3R0.
%    - PicoCIAA.
% 3. Ejemplos de utilización.
% 4. Informe final.
% 5. Manual de referencia de la API de la biblioteca.
% 6. Tutorial de instalación de las herramientas para utilizar la biblioteca con las plataformas de hardware citadas.

